%&latex -*- latex -*-

\chapter{System Calls}
\label{chapt:syscalls}

Scsh provides (almost) complete access to the basic {\Unix} kernel services:
processes, files, signals and so forth.  These procedures comprise a 
{\Scheme} binding for {\Posix}, with a few of the more standard extensions
thrown in (\eg, symbolic links, \ex{fchown}, \ex{fstat}, sockets).


\section{Errors}
Scsh syscalls never return error codes, and do not use a global
\ex{errno} variable to report errors.
Errors are consistently reported by raising exceptions.
This frees up the procedures to return useful values,
and allows the programmer to assume that 
\emph{if a syscall returns, it succeeded.}
This greatly simplifies the flow of the code from the programmer's point
of view.

Since {\Scheme} does not yet have a standard exception system, the scsh
definition remains somewhat vague on the actual form of exceptions
and exception handlers. When a standard exception system is defined,
scsh will move to it. For now, scsh uses the {\scm} exception system,
with a simple sugaring on top to hide the details in the common case.

System call error exceptions contain the {\Unix} \ex{errno} code reported by
the system call. Unlike C, the \ex{errno} value is a part of the exception
packet, it is \emph{not} accessed through a global variable.

For reference purposes, the {\Unix} \ex{errno} numbers 
are bound to the variables \ex{errno/perm}, \ex{errno/noent}, {\etc}
System calls never return \ex{error/intr}---they
automatically retry.

\begin{dfndesc}
    {errno-error}{errno syscall .\ data}{\noreturn}{procedure}
Raises a {\Unix} error exception for {\Unix} error number \var{errno}.
The \var{syscall} and \var{data} arguments are packaged up in the exception
packet passed to the exception handler.
\end{dfndesc}

\defunx{with-errno-handler*}{handler thunk}{value(s) of thunk}
\begin{dfndescx}
    {with-errno-handler}{handler-spec . body}{\valueofbody}{syntax}
{\Unix} syscalls raise error exceptions by calling \ex{errno-error}.
Programs can use \ex{with-errno-handler*} to establish
handlers for these exceptions.

If a {\Unix} error arises while \var{thunk} is executing, 
\var{handler} is called on two arguments like this:
        \codex{(\var{handler} \var{errno} \var{packet})}
\var{packet} is a list of the form
     $$\var{packet} = \ex{(\var{errno-msg} \var{syscall} . \var{data})},$$
where \var{errno-msg} is the standard {\Unix} error message for the error,
      \var{syscall} is the procedure that generated the error,
and   \var{data} is a list of information generated by the error,
      which varies from syscall to syscall.

If \var{handler} returns, the handler search continues upwards.
\var{Handler} can acquire the exception by invoking a saved continuation.
This procedure can be sugared over with the following syntax:
%
\begin{code}
(with-errno-handler
    ((\var{errno} \var{packet}) \var{clause} \ldots)
  \var{body1}
  \var{body2}
  \ldots)\end{code}
%
This form executes the body forms with a particular errno handler installed.
When an errno error is raised, the handler search machinery will
bind variable \var{errno} to the error's integer code, and variable
\var{packet} to the error's auxiliary data packet.
Then, the clauses will be checked for a match.
The first clause that matches is executed, and its value is the
value of the entire \ex{with-errno-handler} form.
If no clause matches, the handler search continues.

Error clauses have two forms
%
\begin{code}
((\var{errno} \ldots) \var{body} \ldots)
(else \var{body} \ldots)\end{code}
%
In the first type of clause, the \var{errno} forms are integer expressions.
They are evaluated and compared to the error's errno value.
An \ex{else} clause matches any errno value. 
Note that the \var{errno} and \var{data}
variables are lexically visible to the error clauses.

Example:
\begin{code}    
(with-errno-handler 
    ((errno packet) ; Only handle 3 particular errors.
     ((errno/wouldblock errno/again)
      (loop))
     ((errno/acces)
      (format #t "Not allowed access!")
      #f))

  (foo frobbotz)
  (blatz garglemumph))\end{code}
%
It is not defined what dynamic context the handler executes in,
so fluid variables cannot reliably be referenced.

Note that Scsh system calls always retry when interrupted, so that
the \ex{errno/intr} exception is never raised.
If the programmer wishes to abort a system call on an interrupt, he
should have the interrupt handler explicitly raise an exception or
invoke a stored continuation to throw out of the system call.
\end{dfndescx}


\subsection{Interactive mode and error handling}
Scsh runs in two modes: interactive and script mode. It starts up in
interactive mode if the scsh interpreter is started up with no script
argument. Otherwise, scsh starts up in script mode.  The mode determines
whether scsh prints prompts in between reading and evaluating forms, and it
affects the default error handler.  In interactive mode, the default error
handler will report the error, and generate an interactive breakpoint so that
the user can interact with the system to examine, fix, or dismiss from the
error. In script mode, the default error handler causes the scsh process to
exit.

When scsh forks a child with \ex{(fork)}, the child resets to script mode.
This can be overridden if the programmer wishes.

%%%%%%%%%%%%%%%%%%%%%%%%%%%%%%%%%%%%%%%%%%%%%%%%%%%%%%%%%%%%%%%%%%%%%%%%%%%%%%%
\section{I/O}

\subsection{Standard {\RnRS} I/O procedures}
In scsh, most standard {\RnRS} I/O operations (such as \ex{display} or
\ex{read-char}) work on both integer file descriptors and {\Scheme} ports.
When doing I/O with a file descriptor, the I/O operation is done
directly on the file, bypassing any buffered data that may have
accumulated in an associated port.
Note that character-at-a-time operations such as \ex{read-char}
are likely to be quite slow when performed directly upon file
descriptors.

The standard {\RnRS} procedures \ex{read-char}, \ex{char-ready?}, \ex{write},
\ex{display}, \ex{newline},
and \ex{write-char} are all generic, accepting integer file descriptor
arguments as well as ports.
Scsh also mandates the availability of \ex{format}, and further requires
\ex{format} to accept file descriptor arguments as well as ports.

The procedures \ex{peek-char} and \ex{read} do \emph{not} accept
file descriptor arguments, since these functions require the ability to
read ahead in the input stream, a feature not supported by {\Unix} I/O.

%%%%%%%%%%%%%%%%%%%%%%%%%%%%%%%%%%%%%%%%%%%%%%%%%%%%%%%%%%%%%%%%%%%%%%%%%%%%%%%
\subsection{Port manipulation and standard ports}
\defun  {close-after} {port consumer} {value(s) of consumer}
\begin{desc}
    Returns \ex{(\var{consumer} \var{port})}, but closes the port on return.
    No dynamic-wind magic. \remark{Is there a less-awkward name?}
\end{desc}

\defun  {error-output-port}{} {port}
\begin{desc}
This procedure is analogous to \ex{current-output-port}, but produces
a port used for error messages---the scsh equivalent of stderr.
\end{desc}

\defun  {with-current-input-port*}  {port thunk} {value(s) of thunk}
\defunx {with-current-output-port*} {port thunk} {value(s) of thunk}
\defunx {with-error-output-port*}   {port thunk} {value(s) of thunk}
\begin{desc}
These procedures install \var{port} as the current input, current output,
and error output port, respectively, for the duration of a call to
\var{thunk}.
\end{desc}

\dfn  {with-current-input-port}  {port . body} {value(s) of body} {syntax}
\dfnx {with-current-output-port} {port . body} {value(s) of body} {syntax}
\dfnx {with-error-output-port}   {port . body} {value(s) of body} {syntax}
\begin{desc}
These special forms are simply syntactic sugar for the
{\ttt with\=current\=input\=port*} procedure and friends.
\end{desc}

\defun {set-current-input-port!} {port}{\undefined}
\defunx{set-current-output-port!}{port}{\undefined}
\defunx{set-error-output-port!}  {port}{\undefined}
\begin{desc}
These procedures alter the dynamic binding of the current I/O port procedures
to new values.
\end{desc}

\defun {close} {fd/port} {\boolean}
\begin{desc}
    Close the port or file descriptor.

    If \var{fd/port} is a file descriptor, and it has a port allocated to it,
    the port is shifted to a new file descriptor created with \ex{(dup
    fd/port)} before closing \ex{fd/port}. The port then has its revealed
    count set to zero.  This reflects the design criteria that ports are not
    associated with file descriptors, but with open files.

    To close a file descriptor, and any associated port it might have, you
    must instead say one of (as appropriate):
\begin{code}
(close (fdes->inport  fd))
(close (fdes->outport fd))\end{code}

    The procedure returns true if it closed an open port.
    If the port was already closed, it returns false; 
    this is not an error.
\end{desc}

\defun  {stdports->stdio}{} {\undefined}
\defunx {stdio->stdports}{} {\undefined}
\begin{desc}
    These two procedures are used to synchronise Unix' standard I/O
    file descriptors and Scheme's current I/O ports.

    \ex{(stdports->stdio)} causes the standard I/O file descriptors
    (0, 1, and 2) to take their values from the current I/O ports.
    It is exactly equivalent to the series of 
    redirections:\footnote{Why not \ex{move->fdes}? 
                           Because the current output port and error port
                           might be the same port.}
\begin{code}
(dup (current-input-port)  0)
(dup (current-output-port) 1)
(dup (error-output-port)   2)\end{code}
%
    \ex{stdio->stdports} causes the bindings of the current I/O ports
    to be changed to ports constructed over the standard I/O file
    descriptors.
    It is exactly equivalent to the series of assignments
\begin{code}
(set-current-input-port!  (fdes->inport  0))
(set-current-output-port! (fdes->outport 1))
(set-error-output-port!   (fdes->outport 2))\end{code}
However, you are more likely to find the dynamic-extent variant,
\ex{with-stdio-ports*}, below, to be of use in general programming.
\end{desc}

\defun{with-stdio-ports*} {thunk} {value(s) of thunk}
\dfnx {with-stdio-ports} {body \ldots} {value(s) of body}{syntax}
\begin{desc}
    \ex{with-stdio-ports*} binds the standard ports \ex{(current-input-port)}, 
    \ex{(current-output-port)}, and \ex{(error-output-port)} to be ports
    on file descriptors 0, 1, 2, and then calls \var{thunk}.
    It is equivalent to:
\begin{code}
(with-current-input-port (fdes->inport 0)
  (with-current-output-port (fdes->inport 1)
    (with-error-output-port (fdes->outport 2)
      (thunk))))\end{code}
%
The \ex{with-stdio-ports} special form is merely syntactic sugar.
\end{desc}




\subsection{String ports}
{\scm} has string ports, which you can use. Scsh has not committed to the
particular interface or names that {\scm} uses, so be warned that the
interface described herein may be liable to change.

\defun {make-string-input-port} {string} {\port}
\begin{desc}
  Returns a port that reads characters from the supplied string.
\end{desc}

\defun  {make-string-output-port} {} {\port}
\defunx {string-output-port-output} {port} {\str}
\begin{desc}
A string output port is a port that collects the characters given to it into
a string.
The accumulated string is retrieved by applying \ex{string-output-port-output}
to the port.
\end{desc}

\defun {call-with-string-output-port} {procedure} {\str}
\begin{desc}
  The \var{procedure} value is called on a port.  When it returns, 
  \ex{call-with-string-output-port} returns a string containing the 
  characters that were written to that port during the execution
  of \var{procedure}.
\end{desc}

%%%%%%%%%%%%%%%%%%%%%%%%%%%%%%%%%%%%%%%%%%%%%%%%%%%%%%%%%%%%%%%%%%%%%%%%%%%%%%
\subsection{Revealed ports and file descriptors}

The material in this section and the following one is not critical for most
applications.
You may safely skim or completely skip this section on a first reading.

Dealing with {\Unix} file descriptors in a {\Scheme} environment is difficult. 
In {\Unix}, open files are part of the process environment, and are referenced
by small integers called \emph{file descriptors}. Open file descriptors are
the fundamental way I/O redirections are passed to subprocesses, since
file descriptors are preserved across fork's and exec's.

{\Scheme}, on the other hand, uses ports for specifying I/O sources. Ports are
garbage-collected {\Scheme} objects, not integers. Ports can be garbage
collected; when a port is collected, it is also closed. Because file
descriptors are just integers, it's impossible to garbage collect them---you
wouldn't be able to close file descriptor 3 unless there were no 3's in the
system, and you could further prove that your program would never again
compute a 3. This is difficult at best.

If a {\Scheme} program only used {\Scheme} ports, and never actually used
file descriptors, this would not be a problem. But {\Scheme} code
must descend to the file descriptor level in at least two circumstances:
%    
\begin{itemize}    
    \item when interfacing to foreign code
    \item when interfacing to a subprocess.
\end{itemize}
%    
This causes a problem. Suppose we have a {\Scheme} port constructed
on top of file descriptor 2. We intend to fork off a program that
will inherit this file descriptor. If we drop references to the port,
the garbage collector may prematurely close file 2 before we fork
the subprocess. The interface described below is intended to fix this and 
other problems arising from the mismatch between ports and file descriptors.

The {\Scheme} kernel maintains a port table that maps a file descriptor
to the {\Scheme} port allocated for it (or, {\sharpf} if there is no port
allocated for this file descriptor). This is used to ensure that
there is at most one open port for each open file descriptor.

The port data structure for file ports has two fields besides the descriptor:
\var{revealed} and \var{closed?}. 
When a file port is closed with \ex{(close port)}, 
the port's file descriptor is closed, its entry in the port table is cleared, 
and the port's \var{closed?} field is set to true. 

When a file descriptor is closed with \ex{(close fdes)}, any associated
port is shifted to a new file descriptor created with \ex{(dup fdes)}.
The port has its revealed count reset to zero (and hence becomes eligible
for closing on GC). See discussion below.
To really put a stake through a descriptor's heart without waiting for
associated ports to be GC'd, you must say one of
%
\begin{code}
(close (fdes->inport fdes))
(close (fdes->output fdes))\end{code}

The \var{revealed} field is an aid to garbage collection. It is an integer
semaphore. If it is zero, the port's file descriptor can be closed when
the port is collected. Essentially, the \var{revealed} field reflects whether
or not the port's file descriptor has escaped to the {\Scheme} user. If
the {\Scheme} user doesn't know what file descriptor is associated with
a given port, then he can't possibly retain an ``integer handle'' on the
port after dropping pointers to the port itself, so the garbage collector
is free to close the file.

Ports allocated with \ex{open-output-file} and \ex{open-input-file} are
unrevealed ports---\ie, \var{revealed} is initialised to 0.
No one knows the port's file descriptor, so the file descriptor can be closed
when the port is collected.

The functions \ex{fdes->output-port}, \ex{fdes->input-port}, \ex{port->fdes}
are used to shift back and forth between file descriptors and ports.  When
\ex{port->fdes} reveals a port's file descriptor, it increments the port's
\var{revealed} field.  When the user is through with the file descriptor, he
can call \ex{(release-port-handle \var{port})}, which decrements the count. 
The function \ex{(call/fdes fd/port \var{proc})} automates this protocol.
\ex{call/fdes} uses \ex{dynamic-wind} to enforce the protocol.  
If \var{proc} throws out of the \ex{call/fdes} application,
the unwind handler releases the descriptor handle;
if the user subsequently tries to throw \emph{back} into \var{proc}'s
context, the wind handler raises an error. When the user maps a file
descriptor to a port with \ex{fdes->outport} or \ex{fdes->inport}, the port
has its revealed field incremented.

Not all file descriptors are created by requests to make ports.  Some are
inherited on process invocation via \ex{\urlh{http://www.FreeBSD.org/cgi/man.cgi?query=exec&apropos=0&sektion=0&manpath=FreeBSD+4.3-RELEASE&format=html}{exec(2)}}, and are simply part of the
global environment. Subprocesses may depend upon them, so if a port is later
allocated for these file descriptors, is should be considered as a revealed
port. For example, when the {\Scheme} shell's process starts up, it opens ports
on file descriptors 0, 1, and 2 for the initial values of
\ex{(current-input-port)}, \ex{(current-output-port)}, and
\ex{(error-output-port)}. 
These ports are initialised with \var{revealed} set to 1,
so that stdin, stdout, and stderr are not closed even if the user drops the
port.  

Unrevealed file ports have the nice property that they can be closed when all
pointers to the port are dropped. This can happen during gc, or at an
\ex{\urlh{http://www.FreeBSD.org/cgi/man.cgi?query=exec&apropos=0&sektion=0&manpath=FreeBSD+4.3-RELEASE&format=html}{exec()}}---since all memory is dropped at an \ex{\urlh{http://www.FreeBSD.org/cgi/man.cgi?query=exec&apropos=0&sektion=0&manpath=FreeBSD+4.3-RELEASE&format=html}{exec()}}.  No one knows the
file descriptor associated with the port, so the exec'd process certainly
can't refer to it.

This facility preserves the transparent close-on-collect property
for file ports that are used in straightforward ways, yet allows
access to the underlying {\Unix} substrate without interference from
the garbage collector. This is critical, since shell programming
absolutely requires access to the {\Unix} file descriptors, as their
numerical values are a critical part of the process interface.

A port's underlying file descriptor can be shifted around with \ex{\urlh{http://www.FreeBSD.org/cgi/man.cgi?query=dup&apropos=0&sektion=0&manpath=FreeBSD+4.3-RELEASE&format=html}{dup(2)}}
when convenient. That is, the actual file descriptor on top of which a port is
constructed can be shifted around underneath the port by the scsh kernel when
necessary.  This is important, because when the user is setting up file
descriptors prior to a \ex{\urlh{http://www.FreeBSD.org/cgi/man.cgi?query=exec&apropos=0&sektion=0&manpath=FreeBSD+4.3-RELEASE&format=html}{exec(2)}}, he may explicitly use a file descriptor
that has already been allocated to some port. In this case, the scsh kernel
just shifts the port's file descriptor to some new location with \ex{dup},
freeing up its old descriptor.  This prevents errors from happening in the
following scenario.  Suppose we have a file open on port \ex{f}.  Now we want
to run a program that reads input on file 0, writes output to file 1, errors
to file 2, and logs execution information on file 3. We want to run this
program with input from \ex{f}. 
So we write:
%
\begin{code}
(run (/usr/shivers/bin/prog)
     (> 1 output.txt)
     (> 2 error.log)
     (> 3 trace.log)
     (= 0 ,f))\end{code}
%
Now, suppose by ill chance that, unbeknownst to us, when the operating system
opened \ex{f}'s file, it allocated descriptor 3 for it. If we blindly redirect
\ex{trace.log} into file descriptor 3, we'll clobber \ex{f}! However, the
port-shuffling machinery saves us: when the \ex{run} form tries to dup
\ex{trace.log}'s file descriptor to 3, \ex{dup} will notice that file
descriptor 3 is already associated with an unrevealed port (\ie, \ex{f}). So,
it will first move \ex{f} to some other file descriptor. This keeps \ex{f}
alive and well so that it can subsequently be dup'd into descriptor 0 for
\ex{prog}'s stdin.

The port-shifting machinery makes the following guarantee: a port is only
moved when the underlying file descriptor is closed, either by a \ex{\urlh{http://www.FreeBSD.org/cgi/man.cgi?query=close&apropos=0&sektion=0&manpath=FreeBSD+4.3-RELEASE&format=html}{close()}}
or a \ex{\urlh{http://www.FreeBSD.org/cgi/man.cgi?query=dup2&apropos=0&sektion=0&manpath=FreeBSD+4.3-RELEASE&format=html}{dup2()}} operation. Otherwise a port/file-descriptor association is
stable.

Under normal circumstances, all this machinery just works behind the scenes to
keep things straightened out. The only time the user has to think about it is
when he starts accessing file descriptors from ports, which he should almost
never have to do. If a user starts asking what file descriptors have been
allocated to what ports, he has to take responsibility for managing this
information.

\subsection{Port-mapping machinery}

The procedures provided in this section are almost never needed.
You may safely skim or completely skip this section on a first reading.

Here are the routines for manipulating ports in scsh. The important
points to remember are:
\begin{itemize}
    \item A file port is associated with an open file, not a particular file
      descriptor.
    \item The association between a file port and a particular file descriptor
      is never changed \emph{except} when the file descriptor is explicitly
      closed. ``Closing'' includes being used as the target of a \ex{dup2}, so
      the set of procedures below that close their targets are
      \ex{close}, two-argument \ex{dup}, and \ex{move->fdes}.
      If the target file descriptor of one of these routines has an
      allocated port, the port will be shifted to another freshly-allocated
      file descriptor, and marked as unrevealed, thus preserving the port
      but freeing its old file descriptor.
\end{itemize}
These rules are what is necessary to ``make things work out'' with no
surprises in the general case.

\defun  {fdes->inport}  {fd} {port}
\defunx {fdes->outport} {fd} {port}
\defunx {port->fdes}  {port} {\fixnum}
\begin{desc}
    These increment the port's revealed count.
\end{desc}

\defun  {port-revealed} {port} {{\integer} or \sharpf}
\begin{desc}
Return the port's revealed count if positive, otherwise \sharpf.
\end{desc}

\defun{release-port-handle} {port} {\undefined}
\begin{desc}
Decrement the port's revealed count.
\end{desc}

\defun {call/fdes} {fd/port consumer} {value(s) of consumer}
\begin{desc}
  Calls \var{consumer} on a file descriptor; 
  takes care of revealed bookkeeping.
  If \var{fd/port} is a file descriptor, this is just 
  \ex{(\var{consumer} \var{fd/port})}.
  If \var{fd/port} is a port, 
  calls \var{consumer} on its underlying file descriptor.
  While \var{consumer} is running, the port's revealed count is incremented.

  When \ex{call/fdes} is called with port argument, you are not allowed to
  throw into \var{consumer} with a stored continuation, as that would violate
  the revealed-count bookkeeping.
\end{desc}

\defun{move->fdes} {fd/port target-fd} {port or fdes}
\begin{desc}
    Maps fd$\rightarrow$fd and port$\rightarrow$port.

    If \var{fd/port} is a file-descriptor not equal to \var{target-fd}, 
    dup it to \var{target-fd} and close it. Returns \var{target-fd}. 

    If \var{fd/port} is a port, it is shifted to \var{target-fd}, 
    by duping its underlying file-descriptor if necessary. 
    \var{Fd/port}'s original file descriptor is
    closed (if it was different from \var{target-fd}). 
    Returns the port. 
    This operation resets \var{fd/port}'s revealed count to 1.

    In all cases when \var{fd/port} is actually shifted, if there is a port
    already using \var{target-fd}, it is first relocated to some other file
    descriptor.
\end{desc}

%%%%%%%%%%%%%%%%%%%%%%%%%%%%%%%%%%%%%%%%%%%%%%%%%%%%%%%%%%%%%%%%%%%%%%%%%%%%%%%
\subsection{{\Unix} I/O}

\defun {dup}            {fd/port [newfd]} {fd/port}
\defunx{dup->inport}    {fd/port [newfd]} {port}
\defunx{dup->outport}   {fd/port [newfd]} {port}
\defunx{dup->fdes}      {fd/port [newfd]} {fd}
\begin{desc}
These procedures provide the functionality of C's \ex{\urlh{http://www.FreeBSD.org/cgi/man.cgi?query=dup&apropos=0&sektion=0&manpath=FreeBSD+4.3-RELEASE&format=html}{dup()}} and \ex{\urlh{http://www.FreeBSD.org/cgi/man.cgi?query=dup2&apropos=0&sektion=0&manpath=FreeBSD+4.3-RELEASE&format=html}{dup2()}}.
The different routines return different types of values:
\ex{dup->inport}, \ex{dup->outport}, and \ex{dup->fdes} return
input ports, output ports, and integer file descriptors, respectively.
\ex{dup}'s return value depends on on the type of 
\var{fd/port}---it maps fd$\rightarrow$fd and port$\rightarrow$port.

These procedures use the {\Unix} \ex{\urlh{http://www.FreeBSD.org/cgi/man.cgi?query=dup&apropos=0&sektion=0&manpath=FreeBSD+4.3-RELEASE&format=html}{dup()}} syscall to replicate
the file descriptor or file port \var{fd/port}.
If a \var{newfd} file descriptor is given, it is used as the target of
the dup operation, \ie, the operation is a \ex{\urlh{http://www.FreeBSD.org/cgi/man.cgi?query=dup2&apropos=0&sektion=0&manpath=FreeBSD+4.3-RELEASE&format=html}{dup2()}}.
In this case, procedures that return a port (such as \ex{dup->inport})
will return one with the revealed count set to one.
For example, \ex{(dup (current-input-port) 5)} produces
a new port with underlying file descriptor 5, whose revealed count is 1.
If \var{newfd} is not specified, 
then the operating system chooses the file descriptor, 
and any returned port is marked as unrevealed.

If the \var{newfd} target is given, 
and some port is already using that file descriptor,
the port is first quietly shifted (with another \ex{dup}) 
to some other file descriptor (zeroing its revealed count).

Since {\Scheme} doesn't provide read/write ports, \ex{dup->inport} and
\ex{dup->outport} can be useful for getting an output version of an
input port, or \emph{vice versa}.  For example, if \ex{p} is an input
port open on a tty, and we would like to do output to that tty, we can
simply use \ex{(dup->outport p)} to produce an equivalent output port
for the tty. Be sure to open the file with the \ex{open/read+write}
flag for this.
\end{desc}

\defun {seek} {fd/port offset [whence]} {\integer}
\begin{desc}
Reposition the I/O cursor for a file descriptor or port.
\var{whence} is one of \{\ex{seek/set}, \ex{seek/delta}, \ex{seek/end}\},
and defaults to \ex{seek/set}.
If \ex{seek/set}, then \var{offset} is an absolute index into the file;
if \ex{seek/delta}, then \var{offset} is a relative offset from the current
    I/O cursor;
if \ex{seek/end}, then \var{offset} is a relative offset from the end of file.
The \var{fd/port} argument may be a port or an integer file descriptor.
Not all such values are seekable;
this is dependent on the OS implementation.
The return value is the resulting position of the I/O cursor in the I/O stream.
\oops{The current implementation doesn't handle \var{offset} arguments
        that are not immediate integers (\ie, representable in 30 bits).}
\oops{The current implementation doesn't handle buffered ports.}
\end{desc}


\defun {tell} {fd/port} {\integer}
\begin{desc}
Returns the position of the I/O cursor in the the I/O stream.
Not all file descriptors or ports support cursor-reporting; 
this is dependent on the OS implementation.
\end{desc}

\begin{defundesc} {open-file} {fname flags [perms]} {\port}
  \var{Perms} defaults to \cd{#o666}.
  \var{Flags} is an integer bitmask, composed by or'ing together constants
  listed in table~\ref{table:fdes-status-flags} 
  (page~\pageref{table:fdes-status-flags}).
  You must use exactly one of the \ex{open/read}, \ex{open/write}, or
  \ex{open/read+write} flags.
%
  The returned port is an input port if the \var{flags} permit it, 
  otherwise an output port. \RnRS/\scm/scsh do not have input/output ports,
  so it's one or the other. This should be fixed. (You can hack simultaneous
  I/O on a file by opening it r/w, taking the result input port, 
  and duping it to an output port with \ex{dup->outport}.)
\end{defundesc}

\defun{open-input-file}{fname [flags]}\port
\begin{defundescx}{open-output-file}{fname [flags perms]}\port
    These are equivalent to \ex{open-file}, after first setting the
    read/write bits of the \var{flags} argument to \ex{open/read} or
    \ex{open/write}, respectively.
    \var{Flags} defaults to zero for \ex{open-input-file}, 
    and 
        \codex{(bitwise-ior open/create open/truncate)}
    for \ex{open-output-file}.
    These defaults make the procedures backwards-compatible with their
    unary {\RnRS} definitions.
\end{defundescx}

\begin{defundesc} {open-fdes} {fname flags [perms]} \integer
    Returns a file descriptor.
\end{defundesc}

\defun{fdes-flags}{fd/port}{\integer}
\begin{defundescx}{set-fdes-flags}{fd/port \integer}{\undefined}
These procedures allow reading and writing of an open file's flags.
The only such flag defined by {\Posix} is \ex{fdflags/close-on-exec};
your {\Unix} implementation may provide others.

These procedures should not be particularly useful to the programmer,
as the scsh runtime already provides automatic control of the close-on-exec
property.
Unrevealed ports always have their file descriptors marked
close-on-exec, as they can be closed when the scsh process execs a new program.
Whenever the user reveals or unreveals a port's file descriptor, 
the runtime automatically sets or clears the flag for the programmer.
Programmers that manipulate this flag should be aware of these extra, automatic
operations.
\end{defundescx}

\defun{fdes-status}{fd/port}{\integer}
\begin{defundescx}{set-fdes-status}{fd/port \integer}{\undefined}
These procedures allow reading and writing of an open file's status flags
(table~\ref{table:fdes-status-flags}).
%
\begin{table}
\begin{center}
\begin{tabular}{@{}rp{1.5in}>{\ttfamily}l@{}}
& Allowed operations & Status flag \\ \cline{2-3}
\textbf{Open+Get+Set} &
    \parbox[t]{1.5in}{\raggedright
     These flags can be used in \ex{open-file}, \ex{fdes-status},
     and \ex{set-fdes-status} calls.} &
%
    \begin{tabular}[t]{@{}>{\ttfamily}l@{}}
    %% These are gettable and settable
    open/append         \\
    open/non-blocking   \\
    open/async \textrm{(Non-\Posix)} \\
    open/fsync \textrm{(Non-\Posix)}
    \end{tabular}
\\\cline{2-3}
\textbf{Open+Get} &
    \parbox[t]{1.5in}{\raggedright
     These flags can be used in \ex{open-file} and \ex{fdes-status} calls,
     but are ignored by \ex{set-fdes-status}.\strut} &
%
    \begin{tabular}[t]{@{}>{\ttfamily}l@{}}
    %% These are gettable, not settable
    open/read           \\
    open/write          \\
    open/read+write     \\
    open/access-mask
    \end{tabular}
\\\cline{2-3}
\textbf{Open} &
    \parbox[t]{1.5in}{\raggedright
     These flags are only relevant in 
     \ex{open-file} calls; 
     they are ignored by \ex{fdes-status} and \ex{set-fdes-status} calls.} &
%
    \begin{tabular}[t]{@{}>{\ttfamily}l@{}}
    %% These are neither gettable nor settable.
    open/create         \\
    open/exclusive              \\
    open/no-control-tty \\
    open/truncate               
    \end{tabular}
\end{tabular}
\end{center}
\caption{Status flags for \texttt{open-file},
         \texttt{fdes-status} and \texttt{set-fdes-status}.
         Only {\Posix} flags are guaranteed to be present;
         your operating system may define others.
         The \texttt{open/access-mask} value is not an actual flag,
         but a bit mask used to select the field for the \texttt{open/read},
         \texttt{open/write} and \texttt{open/read+write} bits.
        }
\label{table:fdes-status-flags}
\end{table}

Note that this file-descriptor state is shared between file descriptors
created by \ex{dup}---if you create port \var{b} by applying \ex{dup}
to port \var{a}, and change {\var{b}}'s status flags, you will also have
changed {\var{a}}'s status flags.
\end{defundescx}

\begin{defundesc}{pipe}{} {[\var{rport} \var{wport}]}
Returns two ports, the read and write end-points of a {\Unix} pipe.
\end{defundesc}

\defun{read-string}{nbytes [fd/port]} {{\str} or \sharpf}
\dfnix{read-string!} {str [fd/port start end]} {nread or \sharpf}{procedure}
        {read-string"!@\texttt{read-string"!}}
\begin{desc}
  These calls read exactly as much data as you requested, unless
  there is not enough data (eof). 
  \ex{read-string!} reads the data into string \var{str}
  at the indices in the half-open interval $[\var{start},\var{end})$;
  the default interval is the whole string: $\var{start}=0$ and
  $\var{end}=\ex{(string-length \var{string})}$.
  They will persistently retry on partial reads and when interrupted
  until (1) error, (2) eof, or (3) the input request is completely
  satisfied.
  Partial reads can occur when reading from an intermittent source,
  such as a pipe or tty.

  \ex{read-string} returns the string read; \ex{read-string!} returns
  the number of characters read. They both return false at eof.
  A request to read zero bytes returns immediately, with no eof check.

  The values of \var{start} and \var{end} must specify a well-defined
  interval in \var{str}, 
  \ie, $0 \le \var{start} \le \var{end} \le \ex{(string-length \var{str})}$.

  Any partially-read data is included in the error exception packet.
  Error returns on non-blocking input are considered an error.
\end{desc}

\defun {read-string/partial} {nbytes [fd/port]} {{\str} or \sharpf}
\dfnix{read-string!/partial} {str [fd/port start end]} {nread or \sharpf}
        {procedure}{read-string"!/partial@\texttt{read-string"!/partial}}
\begin{desc}
%
  These are atomic best-effort/forward-progress calls. 
  Best effort: they may read less than you request if there is a
  lesser amount of data immediately available (\eg, because you
  are reading from a pipe or a tty).
  Forward progress: if no data is immediately available 
  (\eg, empty pipe), they will block.
  Therefore, if you request an $n>0$ byte read, 
  while you may not get everything you asked for, you will always get something
  (barring eof).

  There is one case in which the forward-progress guarantee is cancelled:
  when the programmer explicitly sets the port to non-blocking I/O.
  In this case, if no data is immediately available, 
  the procedure will not block, but will immediately return a zero-byte read.

  \ex{read-string/partial} reads the data into a freshly allocated string,
  which it returns as its value.
  \ex{read-string!/partial} reads the data into string \var{str}
  at the indices in the half-open interval $[\var{start},\var{end})$;
  the default interval is the whole string: $\var{start}=0$ and
  $\var{end}=\ex{(string-length \var{string})}$.
  The values of \var{start} and \var{end} must specify a well-defined
  interval in \var{str}, 
  \ie, $0 \le \var{start} \le \var{end} \le \ex{(string-length \var{str})}$.
  It returns the number of bytes read.

  A request to read zero bytes returns immediatedly, with no eof check.

  In sum, there are only three ways you can get a zero-byte read:
  (1) you request one, (2) you turn on non-blocking I/O, or (3) you
  try to read at eof.

  These are the routines to use for non-blocking input.
  They are also useful when you wish to efficiently process data
  in large blocks, and your algorithm is insensitive to the block size
  of any particular read operation.
\end{desc}

\defun {select }{rvec wvec evec [timeout]}{[rvec' wvec' evec']}
\defunx{select!}{rvec wvec evec [timeout]}{[nr nw ne]}
\begin{desc}
    The \ex{select} procedure allows a process to block and wait for
    events on multiple I/O channels.  The \var{rvec} and \var{evec}
    arguments are vectors of input ports and integer file descriptors;
    \var{wvec} is a vector of output ports and integer file
    descriptors.  The procedure returns three vectors whose elements
    are subsets of the corresponding arguments.  Every element of
    \var{rvec'} is ready for input; every element of \var{wvec'} is
    ready for output; every element of \var{evec'} has an exceptional
    condition pending.
   
    The \ex{select} call will block until at least one of the I/O
    channels passed to it is ready for operation.  For an input port
    this means that it either has data sitting its buffer or that the
    underlying file descriptor has data waiting.  For an output port
    this means that it either has space available in the associated
    buffer or that the underlying file descriptor can accept output.
    For file descriptors, no buffers are checked, even if they have
    associated ports.
    
    The \var{timeout} value can be used to force the call to time-out
    after a given number of seconds. It defaults to the special value
    \ex{\#f}, meaning wait indefinitely. A zero value can be used to
    poll the I/O channels.
   
    If an I/O channel appears more than once in a given
    vector---perhaps occuring once as a Scheme port, and once as the
    port's underlying integer file descriptor---only one of these two
    references may appear in the returned vector.  Buffered I/O ports
    are handled specially---if an input port's buffer is not empty, or
    an output port's buffer is not yet full, then these ports are
    immediately considered eligible for I/O without using the actual,
    primitive \ex{select} system call to check the underlying file
    descriptor.  This works pretty well for buffered input ports, but
    is a little problematic for buffered output ports.
   
    The \ex{select!} procedure is similar, but indicates the subset of
    active I/O channels by side-effecting the argument vectors.
    Non-active I/O channels in the argument vectors are overwritten
    with {\sharpf} values.  The call returns the number of active
    elements remaining in each vector.  As a convenience, the vectors
    passed in to \ex{select!} are allowed to contain {\sharpf} values
    as well as integers and ports.

    \remark{\texttt{Select} and \texttt{select!} do not
      call their POSIX counterparts directly---there is a POSIX
      \texttt{select} sitting at the very heart of the Scheme 48/scsh
      I/O system, so \emph{all} multiplexed I/O is really
      \texttt{select}-based.  Therefore, you cannot expect a
      performance increase from writing a single-threaded program
      using \texttt{select} and \texttt{select!} instead of writing a
      multi-threaded program where each thread handles one I/O
      connection.
      
      The moral of this story is that \texttt{select} and
      \texttt{select!} make sense in only two situations: legacy code
      written for an older version of scsh, and programs which make
      inherent use of \texttt{select}/\texttt{select!} which do not
      benefit from multiple threads.  Examples are network clients
      that send requests to multiple alternate servers and discard all
      but one of them.
      
      In any case, the \texttt{select-ports} and
      \texttt{select-port-channels} procedures described below
      are usually a preferable alternative to
      \texttt{select}/\texttt{select!}: they are much simpler to use, and
      also have a slightly more efficient implementation.}
\end{desc}

\defun {select-ports}{timeout port \ldots}{ready-ports}
\begin{desc}
  The \ex{select-ports} call will block until at least one of the
  ports passed to it is ready for operation or until the timeout has
  expired.  For an input port this means that it either has data
  sitting its buffer or that the underlying file descriptor has data
  waiting.  For an output port this means that it either has space
  available in the associated buffer or that the underlying file
  descriptor can accept output.
    
  The \var{timeout} value can be used to force the call to time out
  after a given number of seconds.  A value of \ex{\#f} means to wait
  indefinitely.  A zero value can be used to poll the ports.

  \texttt{Select-ports} returns a list of the ports ready for
  operation.  Note that this list may be empty if the timeout expired
  before any ports became ready.
\end{desc}

\defun {select-port-channels}{timeout port \ldots}{ready-ports}
\begin{desc}
  \texttt{Select-port-channels} is like \texttt{select-ports}, except
  that it only looks at the operating system objects the ports refer
  to, ignoring any buffering performed by the ports.

  \remark{\texttt{Select-port-channels} should be used with care: for
    example, if an input port has data in the buffer but no data
    available on the underlying file descriptor,
    \texttt{select-port-channels} will block, even though a read
    operation on the port would be able to complete without blocking.
    
    \texttt{Select-port-channels} is intended for situations where the
    program is not checking for available data, but rather for waiting
    until a port has established a connection---for example, to a
    network port.}
\end{desc}

\begin{defundescx}{write-string}{string [fd/port start end]}\undefined
    This procedure writes all the data requested. 
    If the procedure cannot perform the write with a single kernel call
    (due to interrupts or partial writes),
    it will perform multiple write operations until all the data is written
    or an error has occurred.
    A non-blocking I/O error is considered an error.
    (Error exception packets for this syscall include the amount of
     data partially transferred before the error occurred.)

    The data written are the characters of \var{string} in the half-open
    interval $[\var{start},\var{end})$.
    The default interval is the whole string: $\var{start}=0$ and
    $\var{end}=\ex{(string-length \var{string})}$.
    The values of \var{start} and \var{end} must specify a well-defined
    interval in \var{str}, 
    \ie, $0 \le \var{start} \le \var{end} \le \ex{(string-length \var{str})}$.
    A zero-byte write returns immediately, with no error.

    Output to buffered ports: \ex{write-string}'s efforts end as soon
    as all the data has been placed in the output buffer.
    Errors and true output may not happen until a later time, of course.
\end{defundescx}

\begin{defundescx}{write-string/partial}{string [fd/port start end]}{nwritten}
    This routine is the atomic best-effort/forward-progress analog
    to \ex{write-string}.
    It returns the number of bytes written, which may be less than you
    asked for.
    Partial writes can occur when (1) we write off the physical end of
    the media, (2) the write is interrrupted, or (3) the file descriptor
    is set for non-blocking I/O.

    If the file descriptor is not set up for non-blocking I/O, then
    a successful return from these procedures makes a forward progress
    guarantee---that is, a partial write took place of at least one byte:
    \begin{itemize}
    \item If we are at the end of physical media, and no write takes place,
      an error exception is raised.
      So a return implies we wrote \emph{something}.
    \item If the call is interrupted after a partial transfer, it returns
      immediately. But if the call is interrupted before any data transfer,
      then the write is retried.
    \end{itemize}

    If we request a zero-byte write, then the call immediately returns 0.
    If the file descriptor is set for non-blocking I/O, then the call
    may return 0 if it was unable to immediately write anything
    (\eg, full pipe).
    Barring these two cases, a write either returns $\var{nwritten} > 0$, 
    or raises an error exception.

    Non-blocking I/O is only available on file descriptors and unbuffered
    ports. Doing non-blocking I/O to a buffered port is not well-defined,
    and is an error (the problem is the subsequent flush operation).
    
    \oops{\ex{write-string/partial} is currently not implemented.
      Consider using threads to achive the same functionality.}
\end{defundescx}

\subsection{Buffered I/O}

{\scm} ports use buffered I/O---data is transferred to or from the
OS in blocks. Scsh provides control of this mechanism: the programmer
may force saved-up output data to be transferred to the OS when
he chooses, 
and may also choose which I/O buffering policy to employ for a given
port (or turn buffering off completely). 

It can be useful to turn I/O buffering off in some cases, for example
when an I/O stream is to be shared by multiple subprocesses.
For this reason, scsh allocates an unbuffered port for file descriptor 0
at start-up time.
Because shells frequently share stdin with subprocesses, if the shell
does buffered reads, it might ``steal'' input intended for a subprocess.  For
this reason, all shells, including sh, csh, and scsh, read stdin unbuffered.
Applications that can tolerate buffered input on stdin can reset
\ex{(current-input-port)} to block buffering for higher performance.

\note{So support \texttt{peek-char} a Scheme implementation has to
  maintain a buffer for all input ports. In scsh, for ``unbuffered''
  input ports the buffer size is one. As you cannot request less then
  one character there is no unrequested reading so this can still be
  called ``unbuffered input''.}

\begin{defundesc}{set-port-buffering}{port policy [size]}\undefined
This procedure allows the programmer to assign a particular I/O buffering
policy to a port, and to choose the size of the associated buffer.
It may only be used on new ports, \ie, before I/O is performed on the port.
There are three buffering policies that may be chosen:
        \begin{inset}
        \begin{tabular}{l@{\qquad}l}
        \exi{bufpol/block} & General block buffering (general default) \\
        \exi{bufpol/line}  & Line buffering (tty default) \\
        \exi{bufpol/none}  & Direct I/O---no buffering\footnote{But see the note above}
        \end{tabular}
        \end{inset}
The line buffering policy flushes output whenever a newline is output;
whenever the buffer is full; or whenever an input is read from stdin.
Line buffering is the default for ports open on terminal devices.
\oops{The current implementation doesn't support \ex{bufpol/line}.}

The \var{size} argument requests an I/O buffer of \var{size} bytes.
For output ports, \var{size} must be non-negative, for input ports
\var{size} must be positve. If not given, a reasonable default is
used. For output ports, if given and zero, buffering is turned off
(\ie, $\var{size} = 0$ for any policy is equivalent to $\var{policy} =
\ex{bufpol/none}$). For input ports, setting the size to one
corresponds to unbuffered input as defined above. If given, \var{size}
must be zero respectively one for \ex{bufpol/none}.
\end{defundesc}

\begin{defundesc}{force-output} {[fd/port]}{\undefined}
    This procedure does nothing when applied to an integer file descriptor 
    or unbuffered port.
    It flushes buffered output when applied to a buffered port,
    and raises a write-error exception on error. Returns no value.
\end{defundesc}

\begin{defundesc}{flush-all-ports} {}{\undefined}
    This procedure flushes all open output ports with buffered data.
\end{defundesc}

\subsection{File locking}
\label{sec:filelocking}
Scsh provides {\Posix} advisory file locking.
\emph{Advisory} locks are locks that can be checked by user code, 
but do not affect other I/O operations.
For example, if a process has an exclusive lock on a region of a file,
other processes will not be able to obtain locks on that region of the file,
but they will still be able to read and write the file with no hindrance.
Using advisory locks requires cooperation amongst the agents accessing
the shared resource.

\remark{
Unfortunately, {\Posix} file locks are associated with actual files,
not with associated open file descriptors.
Once a process locks a file, using some file descriptor \var{fd},
the next time \emph{any} file descriptor referencing that file is closed, 
all associated locks are released.
This severely limits the utility of {\Posix} advisory file locks,
and we'd recommend caution when using them.
It is not without reason that the FreeBSD man pages refer to {\Posix}
file locking as ``completely stupid.''

Scsh moves Scheme ports from file descriptor to file descriptor with 
\ex{\urlh{http://www.FreeBSD.org/cgi/man.cgi?query=dup&apropos=0&sektion=0&manpath=FreeBSD+4.3-RELEASE&format=html}{dup()}} and \ex{\urlh{http://www.FreeBSD.org/cgi/man.cgi?query=close&apropos=0&sektion=0&manpath=FreeBSD+4.3-RELEASE&format=html}{close()}} as required by the runtime, 
so it is impossible to keep file locks open across one of these shifts.
Hence we can only offer {\Posix} advisory file locking directly on raw
integer file descriptors; 
regrettably, there are no facilities for locking Scheme ports.

Note that once a Scheme port is revealed in scsh, the runtime will not
shift the port around with \ex{\urlh{http://www.FreeBSD.org/cgi/man.cgi?query=dup&apropos=0&sektion=0&manpath=FreeBSD+4.3-RELEASE&format=html}{dup()}} and \ex{\urlh{http://www.FreeBSD.org/cgi/man.cgi?query=close&apropos=0&sektion=0&manpath=FreeBSD+4.3-RELEASE&format=html}{close()}}.
This means the file-locking procedures can then be applied to the port's
associated file descriptor.
}

{\Posix} allows the user to lock a region of a file with either 
an exclusive or shared lock.
Locked regions are described by the \emph{lock-region} record:
\begin{code}
(define-record lock-region
  exclusive?
  start
  len
  whence
  proc)\end{code}%
\indextt{lock-region?}%
\indextt{lock-region:exclusive?} \indextt{lock-region:whence}%
\indextt{lock-region:start} \indextt{lock-region:end}%
\indextt{lock-region:len} \indextt{lock-region:proc}%
%
\begin{itemize}
\item 
The \ex{exclusive?} field is true if the lock is exclusive; 
false if it is shared.

\item 
The \ex{whence} field is one of the values from the \ex{seek} call:
\ex{seek/set}, \ex{seek/delta}, or \ex{seek/end}, 
and determines the interpretation of the \ex{start} field:
\begin{itemize}
\item If \ex{seek/set}, the \ex{start} value is simply an absolute index
into the file.
\item If \ex{seek/delta}, the \ex{start} value is an offset from the 
file descriptor's current position in the file.
\item If \ex{seek/end}, the \ex{start} value is an offset from the 
end of the file.
\end{itemize}
The region of the file being locked is given by the \ex{start} and \ex{len}
fields;
if \ex{len} is zero, it means ``infinity,'' that is, the region extends
from the starting point through the end of the file, even as the file is
extended by subsequent write operations.

\item 
The \ex{proc} field gives the process object for the process holding the region
lock, when relevant (see \ex{get-lock-region} below).
\end{itemize}

\begin{defundesc}{make-lock-region}{exclusive? start len [whence]}{lock-region}
This procedure makes a lock-region record. 
The \ex{whence} field defaults to \ex{seek/set}.
\end{defundesc}

\defun {lock-region}{fdes lock}{\undefined}
\defunx{lock-region/no-block}{fdes lock}{\boolean}
\begin{desc}
These procedures lock a region of the file referenced by file descriptor
\var{fdes}.
The \ex{lock-region} procedure blocks until the lock is granted;
the non-blocking variant returns a boolean indicating whether or not
the lock was granted.
To take an exclusive (write) lock, you must have the file descriptor
open with write access;
to take a shared (read) lock, you must have the file descriptor
open with read access.
\end{desc}

\begin{defundesc}{get-lock-region}{fdes lock}{lock-region or \sharpf}
Return the first lock region on \var{fdes} that would conflict with 
lock region \var{lock}.
If there is no such lock region, return false.
This procedure fills out the \ex{proc} field of the returned lock region,
and is the only procedure that has anything to do with this field.
(See section~\ref{sec:proc-objects} for a description of process objects.)
Note that if you apply this procedure to a file system that is shared
across multiple operating systems (\ie, an NFS file system), the \ex{proc}
field may be ambiguous.
We note, again, that {\Posix} advisory file locking is not a terribly useful
or well-designed facility.
\end{defundesc}

\begin{defundesc}{unlock-region}{fdes lock}{\undefined}
Release a lock from a file.
\end{defundesc}

\defun{with-region-lock*}{fdes lock thunk}{value(s) of thunk}
\dfnx{with-region-lock}{fdes lock body \ldots}{value(s) of body}{syntax}
\begin{desc}
This procedure obtains the requested lock, and then calls 
\ex{(\var{thunk})}. When \var{thunk} returns, the lock is released.
A non-local exit (\eg, throwing to a saved continuation or raising
an exception) also causes the lock to be released.

After a normal return from \var{thunk}, its return values are returned
by \ex{with-region-lock*}.
The \ex{with-region-lock} special form is equivalent syntactic sugar.
\end{desc}


%%%%%%%%%%%%%%%%%%%%%%%%%%%%%%%%%%%%%%%%%%%%%%%%%%%%%%%%%%%%%%%%%%%%%%%%%%%%%%%

\section{File system}

Besides the following procedures, which allow access to the
computer's file system, scsh also provides a set of procedures
which manipulate file \emph{names}. These string-processing
procedures are documented in section \ref{sec:filenames}.

\defun {create-directory} {fname [perms override?]} {\undefined}
\defunx{create-fifo}      {fname [perms override?]} {\undefined}
\defunx{create-hard-link} {oldname newname [override?]} {\undefined}
\begin{defundescx}
    {create-symlink} {old-name new-name [override?]} {\undefined}

    These procedures create objects of various kinds in the file system.

    The \var{override?} argument controls the action if there is already an
    object in the file system with the new name:
    \begin{optiontable}
    \sharpf     & signal an error (default) \\
    'query      & prompt the user \\
    \textnormal{\emph{other}}& \parbox[t]{0.7\linewidth}{
                  delete the old object (with \ex{delete-file}
                  or \ex{delete-directory,} as appropriate) before
                  creating the new object.}

    \end{optiontable}

    \var{Perms} defaults to \cd{#o777} (but is masked by the current umask).

    \remark{Currently, if you try to create a hard or symbolic link from a
    file to itself, you will error out with \var{override?} false, and simply
    delete your file with \var{override?} true. Catching this will require
    some sort of true-name procedure, which I currently do not have.}
\end{defundescx}
    
\defun {delete-directory} {fname} \undefined
\defunx{delete-file}      {fname} \undefined
\begin{defundescx} {delete-filesys-object} {fname} \undefined
These procedures delete objects from the file system.
The {\ttt delete\=filesys\=object} procedure will delete an object
of any type from the file system: files, (empty) directories, symlinks, fifos,
\etc.

If the object being deleted doesn't exist, \ex{delete-directory} and
\ex{delete-file} raise an error, 
while \ex{delete-filesys-object} simply returns.
\end{defundescx}

\begin{defundescx}{read-symlink}{fname} \str
    Return the filename referenced by symbolic link \ex{fname}.
\end{defundescx}
    
\begin{defundescx} {rename-file} {old-fname new-fname [override?]} \undefined
    If you override an existing object, then \var{old-fname} 
    and \var{new-fname} must type-match---either both directories, 
    or both non-directories. 
    This is required by the semantics of {\Unix} \ex{\urlh{http://www.FreeBSD.org/cgi/man.cgi?query=rename&apropos=0&sektion=0&manpath=FreeBSD+4.3-RELEASE&format=html}{rename()}}.

    \remark{
    There is an unfortunate atomicity problem with the \ex{rename-file} 
    procedure: if you
    specify no-override, but create file \ex{new-fname} sometime between
    \ex{rename-file}'s existence check and the actual rename operation,
    your file will be clobbered with \ex{old-fname}. There is no way to fix
    this problem, given the semantics of {\Unix} \ex{\urlh{http://www.FreeBSD.org/cgi/man.cgi?query=rename&apropos=0&sektion=0&manpath=FreeBSD+4.3-RELEASE&format=html}{rename()}}; 
    at least it is highly unlikely to occur in practice.
    }
\end{defundescx}
    
\defun {set-file-mode}  {fname/fd/port mode} \undefined
\defunx{set-file-owner} {fname/fd/port uid}  {\undefined}
\defunx{set-file-group} {fname/fd/port gid}  {\undefined}
\begin{desc}    
    These procedures set the permission bits, owner id, and group id of a
    file, respectively.
    The file can be specified by giving the file name, or either an
    integer file descriptor or a port open on the file.
    Setting file user ownership usually requires root privileges.
\end{desc}

\defun {set-file-times} {fname [access-time mod-time]} {\undefined}
\begin{desc}
    This procedure sets the access and modified times for the file
    \var{fname} to the supplied values (see section~\ref{sec:time}
    for the scsh representation of time).
    If neither time argument is supplied, they are both taken to be
    the current time. You must provide both times or neither.
    If the procedure completes successfully, the file's time of last
    status-change (\ex{ctime}) is set to the current time.
\end{desc}

\defun {sync-file} {fd/port} \undefined
\defunx{sync-file-system}{}  \undefined
\begin{desc}
    Calling \ex{sync-file}
    causes {\Unix} to update the disk data structures for a given file.
    If \var{fd/port} is a port, any buffered data it may have is first
    flushed. 
    Calling \ex{sync-file-system} synchronises the kernel's entire file
    system with the disk.

    These procedures are not {\Posix}.
    Interestingly enough, \ex{sync\=file\=system} doesn't actually
    do what it is claimed to do. We just threw it in for humor value.
    See the \ex{\urlh{http://www.FreeBSD.org/cgi/man.cgi?query=sync&apropos=0&sektion=0&manpath=FreeBSD+4.3-RELEASE&format=html}{sync(2)}} man page for {\Unix} enlightenment.
\end{desc}

\begin{defundesc} {truncate-file} {fname/fd/port len} \undefined
    The specified file is truncated to \var{len} bytes in length.
\end{defundesc}

\begin{defundesc}{file-info} {fname/fd/port [chase?]} {file-info-record}
    The \ex{file-info} procedure
    returns a record structure containing everything
    there is to know about a file. If the \var{chase?} flag is true
    (the default), then the procedure chases symlinks and reports on
    the files to which they refer. If \var{chase?} is false, then 
    the procedure checks the actual file itself, even if it's a symlink.
    The \var{chase?} flag is ignored if the file argument is a file descriptor
    or port.

    The value returned is a \emph{file-info record}, defined to have the
    following structure:
\begin{code}
(define-record file-info
  type      ; \{block-special, char-special, directory,
            ;     fifo, regular, socket, symlink\}
  device    ; Device file resides on.
  inode     ; File's inode.
  mode      ; File's mode bits: permissions, setuid, setgid
  nlinks    ; Number of hard links to this file.
  uid       ; Owner of file.
  gid       ; File's group id.
  size      ; Size of file, in bytes.
  atime     ; Time of last access.
  mtime     ; Time of last mod.
  ctime)    ; Time of last status change.\end{code}
\indextt{file-info:type}\indextt{file-info:device}\indextt{file-info:inode}%
\indextt{file-info:mode}\indextt{file-info:nlinks}\indextt{file-info:uid}%
\indextt{file-info:gid}\indextt{file-info:size}\indextt{file-info:atime}%
\indextt{file-info:mtime}\indextt{file-info:ctime}%
%
    The uid field of a file-info record is accessed with the procedure
\codex{(file-info:uid x)}
    and similarly for the other fields. 
    The \ex{type} field is a symbol; all other fields are integers.
    A file-info record is discriminated with the \ex{file-info?} predicate.

The following procedures all return selected information about
a file; they are built on top of \ex{file-info}, and are
called with the same arguments that are passed to it.
\begin{inset}
\begin{tabular}{ll}
Procedure & returns \\\hline
\exi{file-type}          & type \\
\exi{file-inode}         & inode \\
\exi{file-mode}          & mode \\
\exi{file-nlinks}        & nlinks \\
\exi{file-owner}         & uid \\
\exi{file-group}         & gid \\
\exi{file-size}          & size \\
\exi{file-last-access}   & atime \\
\exi{file-last-mod}      & mtime \\
\exi{file-last-status-change} & ctime
\end{tabular}
\end{inset}
%
Example:
\begin{code}    
;; All my files in /usr/tmp:
(filter (\l{f} (= (file-owner f) (user-uid)))
        (directory-files "/usr/tmp")))\end{code}

\remark{\ex{file-info} was named \ex{file-attributes} in releases of scsh
        prior to release 0.4. We changed the name to \ex{file-info} for
        consistency with the other information-retrieval procedures in
        scsh: \ex{user-info}, \ex{group-info}, \ex{host-info}, 
        \ex{network-info }, \ex{service-info}, and \ex{protocol-info}.

        The \ex{file-attributes} binding is still supported in the current
        release of scsh, but is deprecated, and may go away in a future
        release.}
\end{defundesc}

\defun  {file-directory?}{fname/fd/port [chase?]}{\boolean}
\defunx {file-fifo?}{fname/fd/port [chase?]}{\boolean}
\defunx {file-regular?}{fname/fd/port [chase?]}{\boolean}
\defunx {file-socket?}{fname/fd/port [chase?]}{\boolean}
\defunx {file-special?}{fname/fd/port [chase?]}{\boolean}
\defunx {file-symlink?}{fname/fd/port}{\boolean}
\begin{desc}
These procedures are file-type predicates that test the
type of a given file.
They are applied to the same arguments to which \ex{file-info} is applied;
the sole exception is \ex{file-symlink?}, which does not take
the optional \var{chase?} second argument.
\begin{inset}
\begin{tabular}{l@{\qquad}l}
\end{tabular}
\end{inset}
For example,
\codex{(file-directory? "/usr/dalbertz")\qquad\evalto\qquad\sharpt}
\end{desc}

There are variants of these procedures which work directly on
\ex{file-info} records:
\defun  {file-info-directory?}{file-info}{\boolean}
\defunx {file-info-fifo?}{file-info}{\boolean}
\defunx {file-info-regular?}{file-info}{\boolean}
\defunx {file-info-socket?}{file-info}{\boolean}
\defunx {file-info-special?}{file-info}{\boolean}
\defunx {file-info-symlink?}{file-info}{\boolean}

The following set of procedures are a convenient means to work on the
permission bits of a file:

\defun {file-not-readable?}  {fname/fd/port} \boolean
\defunx{file-not-writable?}  {fname/fd/port} \boolean
\defunx{file-not-executable?} {fname/fd/port} \boolean
\begin{desc}
    Returns:
    \begin{optiontable}
        \textnormal{Value}      & meaning \\ \hline
        \sharpf                 & Access permitted \\
        'search-denied          & {\renewcommand{\arraystretch}{1}%
                                   \begin{tabular}[t]{@{}l@{}}
                                     Can't stat---a protected directory \\
                                     is blocking access.\end{tabular}} \\
        'permission             & Permission denied. \\
        'no-directory           & Some directory doesn't exist. \\
        'nonexistent            & File doesn't exist.
    \end{optiontable}
%
    A file is considered writeable if either (1) it exists and is writeable
    or (2) it doesn't exist and the directory is writeable.
    Since symlink permission bits are ignored by the filesystem, these
    calls do not take a \var{chase?} flag.

    Note that these procedures use the process' \emph{effective} user
    and group ids for permission checking. {\Posix} defines an \ex{\urlh{http://www.FreeBSD.org/cgi/man.cgi?query=access&apropos=0&sektion=0&manpath=FreeBSD+4.3-RELEASE&format=html}{access()}}
    function that uses the process' real uid and gids. This is handy
    for setuid programs that would like to find out if the actual user
    has specific rights; scsh ought to provide this functionality (but doesn't
    at the current time). 

    There are several problems with these procedures. First, there's an
    atomicity issue. In between checking permissions for a file and then trying
    an operation on the file, another process could change the permissions,
    so a return value from these functions guarantees nothing. Second, 
    the code special-cases permission checking when the uid is root---if
    the file exists, root is assumed to have the requested permission.
    However, not even root can write a file that is on a read-only file system,
    such as a CD ROM. In this case, \ex{file-not-writable?} will lie, saying
    that root has write access, when in fact the opening the file for write
    access will fail.
    Finally, write permission confounds write access and create access.
    These should be disentangled.

    Some of these problems could be avoided if {\Posix} had a real-uid
    variant of the \ex{\urlh{http://www.FreeBSD.org/cgi/man.cgi?query=access&apropos=0&sektion=0&manpath=FreeBSD+4.3-RELEASE&format=html}{access()}} call we could use, but the atomicity
    issue is still a problem. In the final analysis, the only way to
    find out if you have the right to perform an operation on a file
    is to try and open it for the desired operation. These permission-checking
    functions are mostly intended for script-writing, where loose guarantees
    are tolerated.
\end{desc}

\defun  {file-readable?}   {fname/fd/port} \boolean
\defunx {file-writable?}   {fname/fd/port} \boolean
\defunx {file-executable?} {fname/fd/port} \boolean
\begin{desc}
    These procedures are the logical negation of the 
    preceding \ex{file-not-\ldots?} procedures.
    Refer to them for a discussion of their problems and limitations.
\end{desc}

\defun {file-info-not-readable?}  {file-info} \boolean
\defunx{file-info-not-writable?}  {file-info} \boolean
\defunx{file-info-not-executable?} {file-info} \boolean
\defun  {file-info-readable?}   {file-info} \boolean
\defunx {file-info-writable?}   {file-info} \boolean
\defunx {file-info-executable?} {file-info} \boolean

There are variants which work directly on \ex{file-info} records.

\begin{defundesc}{file-not-exists?} {fname/fd/port [chase?]} \object
Returns:
    \begin{optiontable}    
    \sharpf            & Exists. \\
    \sharpt            & Doesn't exist. \\
    'search-denied     & \parbox[t]{0.5\linewidth}{\sloppy\raggedright
                            Some protected directory
                            is blocking the search.}
    \end{optiontable}
\end{defundesc}

\begin{defundesc}{file-exists?} {fname/fd/port [chase?]} \boolean    
    This is simply
    \ex{(not (file-not-exists? \var{fname} \var{[chase?]}))}
\end{defundesc}

\defun {directory-files} {[dir dotfiles?]}      {string list}
\begin{desc}
    Return the list of files in directory \var{dir}, 
    which defaults to the current working directory.
    The \var{dotfiles?} flag (default {\sharpf}) causes dot files to be
    included in the list.
    Regardless of the value of \var{dotfiles?}, the two files \ex{.} and
    \ex{..} are \emph{never} returned.

    The directory \var{dir} is not prepended to each file name in the
    result list. That is, 
        \codex{(directory-files "/etc")}
    returns
        \codex{("chown" "exports" "fstab" \ldots)}
    \emph{not}
        \codex{("/etc/chown" "/etc/exports" "/etc/fstab" \ldots)}
    To use the files in returned list, the programmer can either manually
    prepend the directory:
        \codex{(map (\l{f} (string-append dir "/" f)) files)}
    or cd to the directory before using the file names:
%
\begin{code}
(with-cwd dir
  (for-each delete-file (directory-files)))\end{code}
%    
    or use the \ex{glob} procedure, defined below.

    A directory list can be generated by \ex{(run/strings (ls))}, but this
    is unreliable, as filenames with whitespace in their names will be
    split into separate entries. Using \ex{directory-files} is reliable.
\end{desc}
    
\defun {open-directory-stream} {dir} {directory-stream-record}
\defun {read-directory-stream} {directory-stream-record} {string or \sharpf}
\defun {close-directory-stream} {directory-stream-record} {\undefined}

These functions implement a direct interface to the
\ex{\urlh{http://www.freebsd.org/cgi/man.cgi?query=opendir&apropos=0&sektion=0&manpath=FreeBSD+4.3-RELEASE&format=html}{opendir()}}/
\ex{\urlh{http://www.freebsd.org/cgi/man.cgi?query=readdir&apropos=0&sektion=0&manpath=FreeBSD+4.3-RELEASE&format=html}{readdir()}}/
\ex{\urlh{http://www.freebsd.org/cgi/man.cgi?query=closedir&apropos=0&sektion=0&manpath=FreeBSD+4.3-RELEASE&format=html}{closedir()}}
family of functions for processing directory streams.
\ex{(open-directory-stream dir)} creates a stream of files in the
directory \ex{dir}. \ex{(read-directory-stream directory-stream)}
returns the next file in the stream or \sharpf if no such file exists.
Finally, \ex{(close-directory-stream directory-stream)} closes the
stream.

\defun {glob} {\vari{pat}1 \ldots} {string list}
\begin{desc}
    Glob each pattern against the filesystem and return the sorted list. 
    Duplicates are not removed. Patterns matching nothing are not included 
    literally.\footnote{Why bother to mention such a silly possibility?
    Because that is what sh does.}
    C shell \verb|{a,b,c}| patterns are expanded. Backslash quotes 
    characters, turning off the special meaning of
    \verb|{|, \verb|}|, \cd{*}, \verb|[|, \verb|]|, and \verb|?|. 

    Note that the rules of backslash for {\Scheme} strings and glob patterns
    work together to require four backslashes in a row to specify a
    single literal backslash. Fortunately, it is very rare that a backslash
    occurs in a Unix file name.

    A glob subpattern will not match against dot files unless the first
    character of the subpattern is a literal ``\ex{.}''. 
    Further, a dot subpattern will not match the files \ex{.} or \ex{..} 
    unless it is a constant pattern, as in \ex{(glob "../*/*.c")}.
    So a directory's dot files can be reliably generated
    with the simple glob pattern \ex{".*"}.

    Some examples:
\begin{inset}
\begin{verbatim}    
(glob "*.c" "*.h")
    ;; All the C and #include files in my directory.

(glob "*.c" "*/*.c")
    ;; All the C files in this directory and 
    ;; its immediate subdirectories.

(glob "lexer/*.c" "parser/*.c")
(glob "{lexer,parser}/*.c")
    ;; All the C files in the lexer and parser dirs.

(glob "\\{lexer,parser\\}/*.c")
    ;; All the C files in the strange 
    ;; directory "{lexer,parser}".

(glob "*\\*")
    ;; All the files ending in "*", e.g.
    ;; ("foo*" "bar*")         

(glob "*lexer*")
    ("mylexer.c" "lexer1.notes") 
    ;; All files containing the string "lexer".

(glob "lexer")
    ;; Either ("lexer") or ().\end{verbatim}
\end{inset}
%    
If the first character of the pattern (after expanding braces) is a slash,
the search begins at root; otherwise, the search begins in the current
working directory.

If the last character of the pattern (after expanding braces) is a slash,
then the result matches must be directories, \eg,
\begin{code}
(glob "/usr/man/man?/") \evalto
        ("/usr/man/man1/" "/usr/man/man2/" \ldots)\end{code}

Globbing can sometimes be useful when we need a list of a directory's files
where each element in the list includes the pathname for the file.
Compare:
\begin{code}
(directory-files "../include") \evalto
    ("cig.h" "decls.h" \ldots)

(glob "../include/*") \evalto
    ("../include/cig.h" "../include/decls.h" \ldots)\end{code}
\end{desc}

\defun{glob-quote}{str}\str
\begin{desc}
Returns a constant glob pattern that exactly matches \var{str}.
All wild-card characters in \var{str} are quoted with a backslash.
\begin{code}
(glob-quote "Any *.c files?")
    {\evalto}"Any \\*.c files\\?"\end{code}
\end{desc}


\begin{defundesc}{file-match}{root dot-files? \vari{pat}1 \vari{pat}2 {\ldots} \vari{pat}n}{string list}
   \note{This procedure is deprecated, and will probably either go away or
         be substantially altered in a future release. New code should not
         call this procedure. The problem is that it relies upon
         Posix-notation regular expressions; the rest of scsh has been 
         converted over to the new SRE notation.}

    \ex{file-match} provides a more powerful file-matching service, at the
    expense of a less convenient notation. It is intermediate in
    power between most shell matching machinery and recursive \ex{\urlh{http://www.FreeBSD.org/cgi/man.cgi?query=find&apropos=0&sektion=0&manpath=FreeBSD+4.3-RELEASE&format=html}{find(1)}}.

    Each pattern is a regexp. The procedure searches from \var{root},
    matching the first-level files against pattern \vari{pat}1, the
    second-level files against \vari{pat}2, and so forth.
    The list of files matching the whole path pattern is returned, 
    in sorted order.
    The matcher uses Spencer's regular expression package.

    The files \ex{.} and \ex{..} are never matched. Other dot files are only
    matched if the \var{dot-files?} argument is \sharpt.

    A given \vari{pat}i pattern is matched as a regexp, so it is not forced
    to match the entire file name. \Eg, pattern \ex{"t"} matches any
    file containing a ``t'' in its name, while pattern \verb|"^t$"| matches
    only a file whose entire name is ``\ex{t}''.

    The \vari{pat}i patterns can be more general than stated above. 
    \begin{itemize}
    \item A single pattern can specify multiple levels of the path by
      embedding \ex{/} characters within the pattern. For example,
      the pattern \ex{"a/b/c"} gives a match equivalent to the
      list of patterns \ex{"a" "b" "c"}.

    \item A \vari{pat}i pattern can be a procedure, 
      which is used as a match predicate.
      It will be repeatedly called with a candidate file-name to test.
      The file-name will be the entire path accumulated.
      If the procedure raises an error condition, \ex{file-match} will
      catch the error and treat it as a failed match.
      This keeps \ex{file-match} from being blown out of the water
      by applying tests to dangling symlinks and other similar situations.

    \end{itemize}

    Some examples:
%% UGH. Because we are using code instead of verbatim, we have to
%% double up on backslashes.
\begin{tightleftinset}
\begin{code}
(file-match "/usr/lib" #f "m$" "^tab") \evalto
    ("/usr/lib/term/tab300" "/usr/lib/term/tab300-12" \ldots)
\cb
(file-match "." #f  "^lex|parse|codegen$" "\\\\.c$") \evalto
    ("lex/lex.c" "lex/lexinit.c" "lex/test.c"
     "parse/actions.c" "parse/error.c" parse/test.c"
     "codegen/io.c" "codegen/walk.c")
\cb
(file-match "." #f  "^lex|parse|codegen$/\\\\.c$")
     ;; The same.
\cb
(file-match "." #f  file-directory?)
    ;; Return all subdirs of the current directory.
\cb
(file-match "/" #f  file-directory?) \evalto
    ("/bin" "/dev" "/etc" "/tmp" "/usr")
    ;; All subdirs of root.
\cb
(file-match "." #f  "\\\\.c")
    ;; All the C files in my directory.
\cb
(define (ext extension)
  (\l{fn} (string-suffix? fn extension)))
\cb
(define (true . x) #t)
\cb
(file-match "." #f  "./\\\\.c")
(file-match "." #f  "" "\\\\.c")
(file-match "." #f  true "\\\\.c")
(file-match "." #f  true (ext "c"))
    ;; All the C files of all my immediate subdirs.
\cb
(file-match "." #f "lexer") \evalto
    ("mylexer.c" "lexer.notes") 
    ;; Compare with (glob "lexer"), above.\end{code}
\end{tightleftinset}
    
Note that when \var{root} is the current working directory (\ex{"."}),
when it is converted to directory form, it becomes \ex{""}, and doesn't
show up in the result file-names.

It is regrettable that the regexp wild card char, ``\ex{.}'', 
is such an important file name literal, as dot-file prefix and extension
delimiter.
\end{defundesc}
    
\begin{defundesc}  {create-temp-file}  {[prefix]}         \str
    \ex{Create-temp-file} creates a new temporary file and return its name.
    The optional argument specifies the filename prefix to use, and defaults
    to the value of \ex{"\TMPDIR{}/\var{pid}"} if \TMPDIR{} is set and to
    \ex{"/var/tmp/\var{pid}"} otherwise, where \var{pid} is the current process' id.
    The procedure generates a sequence of filenames that have \var{prefix} as
    a common prefix, looking for a filename that doesn't already exist in the
    file system. When it finds one, it creates it, with permission \cd{#o600}
    and returns the filename. (The file permission can be changed to a more
    permissive permission with \ex{set-file-mode} after being created).

    This file is guaranteed to be brand new. No other process will have it
    open. This procedure does not simply return a filename that is very
    likely to be unused. It returns a filename that definitely did not exist
    at the moment \ex{create-temp-file} created it.

    It is not necessary for the process' pid to be a part of the filename
    for the uniqueness guarantees to hold. The pid component of the default
    prefix simply serves to scatter the name searches into sparse regions, so
    that collisions are less likely to occur. This speeds things up, but does
    not affect correctness.

    Security note: doing I/O to files created this way in \ex{/var/tmp/} is
    not necessarily secure. General users have write access to \ex{/var/tmp/},
    so even if an attacker cannot access the new temp file, he can delete it
    and replace it with one of his own. A subsequent open of this filename
    will then give you his file, to which he has access rights. There are
    several ways to defeat this attack,
    \begin{enumerate}
        \item Use \ex{temp-file-iterate}, below, to return the file descriptor
           allocated when the file is opened. This will work if the file
           only needs to be opened once.
        \item If the file needs to be opened twice or more, create it in a 
           protected directory, \eg, \verb|$HOME|.
        \item Ensure that \ex{/var/tmp} has its sticky bit set. This
           requires system administrator privileges.
    \end{enumerate}
    The actual default prefix used is controlled by the dynamic variable
    \ex{*temp-file-template*}, and can be overridden for increased security.
    See \ex{temp-file-iterate}.
\end{defundesc}

\defunx  {temp-file-iterate} {maker [template]} {\object\+}                     
\defvarx {*temp-file-template*} \str
\begin{desc}
    This procedure can be used to perform certain atomic transactions on
    the file system involving filenames. Some examples:
    \begin{itemize}
        \item Linking a file to a fresh backup temp name.
        \item Creating and opening an unused, secure temp file.
        \item Creating an unused temporary directory.
    \end{itemize}
    
    This procedure uses \var{template} to generate a series of trial
    file names.  \var{Template} is a \ex{format} control string, and
    defaults to \codex{"\TMPDIR{}/\var{pid}.\~a"} if \TMPDIR{} is set
    and \codex{"/var/tmp/\var{pid}.\~a"} otherwise where \var{pid} is
    the current process' process id.  File names are generated by
    calling \ex{format} to instantiate the template's \verb|~a| field
    with a varying string.

    \var{Maker} is a procedure which is serially called on each file name
    generated.  It must return at least one value; it may return multiple
    values. If the first return value is {\sharpf} or if \var{maker} raises the
    \ex{errno/exist} errno exception, \ex{temp-file-iterate} will loop,
    generating a new file name and calling \var{maker} again. If the first
    return value is true, the loop is terminated, returning whatever value(s)
    \var{maker} returned.

    After a number of unsuccessful trials, \ex{temp-file-iterate} may give up
    and signal an error.

    Thus, if we ignore its optional \var{prefix} argument, 
    \ex{create-temp-file} could be defined as:
\begin{code}
(define (create-temp-file)
  (let ((flags (bitwise-ior open/create open/exclusive)))
    (temp-file-iterate 
        (\l{f} 
          (close (open-output-file f flags #o600))
          f))))\end{code}

    To rename a file to a temporary name:
\begin{code}
(temp-file-iterate (\l{backup}
                     (create-hard-link old-file backup)
                     backup)
                   ".#temp.\~a") ; Keep link in cwd.
(delete-file old-file)\end{code}
    Recall that scsh reports syscall failure by raising an error
    exception, not by returning an error code. This is critical to
    to this example---the programmer can assume that if the 
    \ex{temp-file-iterate} call returns, it returns successully.
    So the following \ex{delete-file} call can be reliably invoked,
    safe in the knowledge that the backup link has definitely been established.

    To create a unique temporary directory:
\begin{code}
(temp-file-iterate (\l{dir} (create-directory dir) dir)
                   "/var/tmp/tempdir.\~a")\end{code}
%
    Similar operations can be used to generate unique symlinks and fifos,
    or to return values other than the new filename (\eg, an open file
    descriptor or port).
    
    The default template is in fact taken from the value of the
    dynamic variable \ex{*temp-file-template*}, which itself defaults
    to \ex{"\TMPDIR{}/\var{pid}.\~a"} if \TMPDIR{} is set and
    \ex{"/usr/tmp/\var{pid}.\~a"} otherwise, where \var{pid} is the
    scsh process' pid.  For increased security, a user may wish to
    change the template to use a directory not allowing world write
    access (\eg, his home directory).
\end{desc}

\defun{temp-file-channel}{} {[inp outp]}
\begin{desc}
    This procedure can be used to provide an interprocess communications
    channel with arbitrary-sized buffering.  It returns two values, an input
    port and an output port, both open on a new temp file.  The temp file
    itself is deleted from the {\Unix} file tree before \ex{temp-file-channel}
    returns, so the file is essentially unnamed, and its disk storage is
    reclaimed as soon as the two ports are closed.

    \ex{Temp-file-channel} is analogous to \ex{port-pipe} with two exceptions:
    \begin{itemize}
    \item If the writer process gets ahead of the reader process, it will
       not hang waiting for some small pipe buffer to drain. It will simply
       buffer the data on disk. This is good.

    \item If the reader process gets ahead of the writer process, it will
       also not hang waiting for data from the writer process. It will
       simply see and report an end of file. This is bad.

       In order to ensure that an end-of-file returned to the reader is
       legitimate, the reader and writer must serialise their I/O. The
       simplest way to do this is for the reader to delay doing input
       until the writer has completely finished doing output, or exited.
    \end{itemize}
\end{desc}

\section{Processes}

\defun  {exec} {prog \vari{arg}1 \ldots \vari{arg}n} \noreturn
\defunx {exec-path} {prog \vari{arg}1 \ldots \vari{arg}n} \noreturn
\defunx {exec/env}  {prog env \vari{arg}1 \ldots \vari{arg}n} \noreturn
\defunx {exec-path/env} {prog env \vari{arg}1 \ldots \vari{arg}n} \noreturn
\begin{desc}

The \ex{\ldots/env} variants take an environment specified as a 
string$\rightarrow$string alist.
An environment of {\sharpt} is taken to mean the current process' environment 
(\ie, the value of the external char \ex{**environ}).

[Rationale: {\sharpf} is a more convenient marker for the current environment
 than {\sharpt}, but would cause an ambiguity on Schemes that identify 
 {\sharpf} and \ex{()}.]

The path-searching variants search the directories in the list 
{\ttt exec\=path\=list} for the program.
A path-search is not performed if the program name contains
a slash character---it is used directly. So a program with a name like
\ex{"bin/prog"} always executes the program \ex{bin/prog} in the current working
directory. See \verb|$path| and \verb|exec-path-list|, below.

Note that there is no analog to the C function \ex{\urlh{http://www.FreeBSD.org/cgi/man.cgi?query=execv&apropos=0&sektion=0&manpath=FreeBSD+4.3-RELEASE&format=html}{execv()}}.
To get the effect just do
\codex{(apply exec prog arglist)}

All of these procedures flush buffered output and close unrevealed ports
before executing the new binary.  
To avoid flushing buffered output, see \verb|%exec| below.

Note that the C \ex{\urlh{http://www.FreeBSD.org/cgi/man.cgi?query=exec&apropos=0&sektion=0&manpath=FreeBSD+4.3-RELEASE&format=html}{exec()}} procedure allows the zeroth element of the
argument vector to be different from the file being executed, \eg
%
\begin{inset}
\begin{verbatim}
char *argv[] = {"-", "-f", 0};
exec("/bin/csh", argv, envp);\end{verbatim}
\end{inset}
%
The scsh \ex{exec}, \ex{exec-path}, \ex{exec/env}, and \ex{exec-path/env}
procedures do not give this functionality---element 0 of the arg vector is
always identical to the \ex{prog} argument. In the rare case the user wishes
to differentiate these two items, he can use the low-level \verb|%exec| and
\verb|exec-path-search| procedures.
These procedures never return under any circumstances.
As with any other system call, if there is an error, they raise
an exception.
\end{desc}


\defun {\%exec} {prog arglist env} \undefined
\defunx{exec-path-search} {fname pathlist} {{\str} or \sharpf}
\begin{desc}
The \ex{\%exec} procedure is the low-level interface to the system call.
The \var{arglist} parameter is a list of arguments; 
\var{env} is either a string$\rightarrow$string alist or {\sharpt}.  
The new program's \cd{argv[0]} will be taken from \ex{(car \var{arglist})},
\emph{not} from \var{prog}.
An environment of {\sharpt} means the current process' environment. 
\verb|%exec| does not flush buffered output
(see \ex{flush-all-ports}).

All exec procedures, including \verb|%exec|, coerce the \cd{prog} and \cd{arg}
values to strings using the usual conversion rules: numbers are converted to
decimal numerals, and symbols converted to their print-names.

\ex{exec-path-search} searches the directories of \var{pathlist} looking for
an occurrence of file \ex{fname}. If no executable file is found, it returns
{\sharpf}. If \ex{fname} contains a slash character, the path search is
short-circuited, but the procedure still checks to ensure that the file exists
and is executable---if not, it still returns {\sharpf}.
Users of this procedure should be aware that it invites a potential race 
condition: between checking the file with \ex{exec-path-search} and executing
it with \ex{\%exec}, the file's status might change.
The only atomic way to do the search is to loop over the candidate
file names, exec'ing each one and looping when the exec operation fails.

See \cd{$path} and \ex{exec-path-list}, below.
\end{desc}

\defun  {exit} {[status]} \noreturn
\defunx {\%exit} {[status]} \noreturn
\begin{desc}
These procedures terminate the current process with a given exit status.
The default exit status is 0.
The low-level \verb|%exit| procedure immediately terminates the process
without flushing buffered output.
\end{desc}

\begin{defundesc} {call-terminally} {thunk} \noreturn
    \ex{call-terminally} calls its thunk. When the thunk returns, the process
    exits.  Although \ex{call-terminally} could be implemented as
        \codex{(\l{thunk} (thunk) (exit 0))}
    an implementation can take advantage of the fact that this procedure never
    returns. For example, the runtime can start with a fresh stack and also
    start with a fresh dynamic environment, where shadowed bindings are
    discarded. This can allow the old stack and dynamic environment to be
    collected (assuming this data is not reachable through some live
    continuation).
\end{defundesc}

\begin{defundesc}{suspend}{} \undefined
Suspend the current process with a SIGSTOP signal.
\end{defundesc}

\defun  {fork}   {[thunk or \sharpf] [continue-threads?]} {proc or \sharpf}
\defunx {\%fork} {[thunk or \sharpf] [continue-threads?]} {proc or \sharpf}
\begin{desc}
  \ex{fork} with no arguments or \sharpf{} instead of a thunk is like
  C
  \ex{\urlh{http://www.FreeBSD.org/cgi/man.cgi?query=fork&apropos=0&sektion=0&manpath=FreeBSD+4.3-RELEASE&format=html}{fork()}}.
  In the parent process, it returns the child's \emph{process object}
  (see below for more information on process objects).  In the child
  process, it returns {\sharpf}.

  \ex{fork} with an argument only returns in the parent process, returning
  the child's process object.
  The child process calls \var{thunk} and then exits.
  
  \ex{fork} flushes buffered output before forking, and sets the child
  process to non-interactive. \verb|%fork| does not perform this bookkeeping;
  it simply forks.
  
  The optional boolean argument \var{continue-threads?} specifies
  whether the currently active threads continue to run in the child or
  not.  The default is \sharpf.
\end{desc}

\defun {fork/pipe}  {[thunk] [continue-threads?]} {proc or \sharpf}
\defunx{\%fork/pipe} {[thunk] [continue-threads?]} {proc or \sharpf}
\begin{desc}
    Like \ex{fork} and \ex{\%fork}, but the parent and child communicate via a
    pipe connecting the parent's stdin to the child's stdout. These procedures
    side-effect the parent by changing his stdin.

    In effect, \ex{fork/pipe} splices a process into the data stream
    immediately upstream of the current process.
    This is the basic function for creating pipelines.
    Long pipelines are built by performing a sequence of \ex{fork/pipe} calls.
    For example, to create a background two-process pipe \ex{a | b}, we write:
%
\begin{code}
(fork (\l{} (fork/pipe a) (b)))\end{code}
%
    which returns the process object for \ex{b}'s process.

    To create a background three-process pipe \ex{a | b | c}, we write:
%
\begin{code}
(fork (\l{} (fork/pipe a)
            (fork/pipe b)
            (c)))\end{code}
%    
    which returns the process object for \ex{c}'s process.

    Note that these procedures affect file descriptors, not ports.
    That is, the pipe is allocated connecting the child's file descriptor
    1 to the parent's file descriptor 0.
    \emph{Any previous Scheme port built over these affected file descriptors
    is shifted to a new, unused file descriptor with \ex{dup} before
    allocating the I/O pipe.}
    This means, for example, that the ports bound to \ex{(current-input-port)}
    and \ex{(current-output-port)} in either process are not affected---they 
    still refer to the same I/O sources and sinks as before.
    Remember the simple scsh rule: Scheme ports are bound to I/O sources
    and sinks, \emph{not} particular file descriptors.

    If the child process wishes to rebind the current output port
    to the pipe on file descriptor 1, it can do this using 
    \ex{with-current-output-port} or a related form.
    Similarly, if the parent wishes to change the current input port
    to the pipe on file descriptor 0, it can do this using
    \ex{set-current-input-port!} or a related form.
    Here is an example showing how to set up the I/O ports on both sides
    of the pipe:
\begin{code}
(fork/pipe (\l{}
             (with-current-output-port (fdes->outport 1)
               (display "Hello, world.\\n"))))

(set-current-input-port! (fdes->inport 0))
(read-line)     ; Read the string output by the child.\end{code}
None of this is necessary when the I/O is performed by an exec'd
program in the child or parent process, only when the pipe will
be referenced by Scheme code through one of the default current I/O
ports.
\end{desc}

\defun  {fork/pipe+}  {conns [thunk] [continue-threads?]} {proc or \sharpf}
\defunx {\%fork/pipe+} {conns [thunk] [continue-threads?]} {proc or \sharpf}
\begin{desc}
    Like \ex{fork/pipe}, but the pipe connections between the child and parent
    are specified by the connection list \var{conns}. 
    See the 
            \codex{(|+ \var{conns} \vari{pf}{\!1} \ldots{} \vari{pf}{\!n})}
    process form for a description of connection lists.
\end{desc}

\subsection{Process objects and process reaping}
\label{sec:proc-objects}
Scsh uses \emph{process objects} to represent Unix processes.
They are created by the \ex{fork} procedure, and have the following
exposed structure:
\begin{code}
(define-record proc
        pid)\end{code}
\index{proc}\index{proc?}\index{proc:pid}
The only exposed slot in a proc record is the process' pid, 
the integer id assigned by Unix to the process.
The only exported primitive procedures for manipulating process objects
are \ex{proc?} and \ex{proc:pid}.
Process objects are created with the \ex{fork} procedure.

\begin{defundesc}{pid->proc}{pid [probe?]}{proc}
This procedure maps integer Unix process ids to scsh process objects.
It is intended for use in interactive and debugging code, 
and is deprecated for use in production code.
If there is no process object in the system indexed by the given pid,
\ex{pid->proc}'s action is determined by the \var{probe?} parameter
(default \sharpf): 
\begin{center}
\begin{tabular}{|l|l|}
\hline
\var{probe?}    &       Return                          \\ \hline\hline
\sharpf         &       \emph{signal error condition.}  \\ \hline
\ex{'create}    &       Create new proc object.         \\ \hline
True value      &       \sharpf                         \\ \hline
\end{tabular}
\end{center}
\end{defundesc}

Sometime after a child process terminates, scsh will perform a \ex{wait}
system call on the child in background, caching the process' exit status
in the child's proc object.
This is called ``reaping'' the process.
Once the child has been waited, the Unix kernel can free the storage allocated
for the dead process' exit information, so process reaping prevents the process
table from becoming cluttered with un-waited dead child processes 
(a.k.a. ``zombies'').
This can be especially severe if the scsh process never waits on child
processes at all; if the process table overflows with forgotten zombies,
the OS may be unable to fork further processes.

Reaping a child process moves its exit status information from the kernel
into the scsh process, where it is cached inside the child's process object.
If the scsh user drops all pointers to the process object, it will simply be
garbage collected.
On the other hand, if the scsh program retains a pointer to the process object,
it can use scsh's \ex{wait} system call to synchronise with the child and
retrieve its exit status multiple times (this is not possible with simple
Unix integer pids in C---the programmer can only wait on a pid once).

Thus, process objects allow scsh programmer to do two things not allowed
in other programming environments:
\begin{itemize}
\item Subprocesses that are never waited on are still removed from the
      process table, and their associated exit status data is eventually
      automatically garbage collected.
\item Subprocesses can be waited on multiple times.
\end{itemize}

However, note that once a child has exited, if the scsh programmer
drops all pointers to the child's proc object, the child's exit status
will be reaped and thrown away.
This is the intended behaviour, and it means that integer pids are not
enough to cause a process's exit status to be retained by the scsh runtime.
(This is because it is clearly impossible to GC data referenced by integers.)

As a convenience for interactive use and debugging, all procedures that
take process objects will also accept integer Unix pids as arguments,
coercing them to the corresponding process objects.
Since integer process ids are not reliable ways to keep a child's exit
status from being reaped and garbage collected, programmers are encouraged
to use process objects in production code.

\begin{defundesc}{autoreap-policy}{[policy]}{old-policy}
The scsh programmer can choose different policies for automatic
process reaping.
The policy is determined by applying this procedure to one of the
values \ex{'early}, \ex{'late}, or {\sharpf} (\ie, no autoreap).
\begin{description}
\item [early]
        The child is reaped from the {\Unix} kernel's process table
        into scsh as soon as it dies. This is done by having a
        signal handler for the \ex{SIGCHLD} signal reap the process.

\item [late]
        The child is not autoreaped until it dies \emph{and} the scsh program
        drops all pointers to its process object. That is, the process
        table is cleaned out during garbage collection.

\item [\sharpf]
        If autoreaping is turned off, process reaping is completely under
        control of the programmer, who can force outstanding zombies to
        be reaped by manually calling the \ex{reap-zombies} procedure
        (see below).
\end{description}
Note that under any of the autoreap policies, a particular process $p$ can
be manually reaped into scsh by simply calling \ex{(wait $p$)}.
\emph{All} zombies can be manually reaped with \ex{reap-zombies}.

The \ex{autoreap-policy} procedure returns the policy's previous value.
Calling \ex{autoreap-policy} with no arguments returns the current
policy without no change.
\end{defundesc}


\begin{defundesc}{reap-zombies}{}{\boolean}
This procedure reaps all outstanding exited child processes into scsh.
It returns true if there are no more child processes to wait on, and
false if there are outstanding processes still running or suspended.
\end{defundesc}

\subsubsection{Issues with process reaping}
Reaping a process does not reveal its process group at the time of
death; this information is lost when the process reaped.
This means that a dead, reaped process is \emph{not eligible} as a return
value for a future \ex{wait-process-group} call.
This is not likely to be a problem for most code, as programs almost
never wait on exited processes by process group.
Process group waiting is usually applied to \emph{stopped} processes,
which are never reaped. 
So it is unlikely that this will be a problem for most programs.

%%% Actually, this is *not* a problem if you stick with proc objects, instead
%%% of using pids, so I commented it out.
%
%\paragraph{Pid aliasing}
%Second, once a process has been reaped, its 16-bit process id becomes
%available to Unix for re-use.
%So it is conceivable that a long time in the future, a \ex{fork} operation
%could produce a subprocess with the identical pid, causing \ex{wait}
%operations on the old, dead, reaped child, and the new child to become
%confused.
%This kind of pid aliasing is intrinsic to the nature of Unix's single-use pid
%deallocation policy,
%but is very, very unlikely to happen in practice,
%given the 16-bit size of the pid space.
%Scsh will detect occurences of pid aliasing, 
%in the unlikely event that one occurs.
%When \ex{fork} creates a proc object, it checks to see if the scsh heap
%contains an already existing proc object with the same pid as the newly forked
%process.
%If so, an exception is raised; if not handled by the program, this will stop
%the program, either killing the process or invoking an interactive debugger.

Automatic process reaping is a useful programming convenience.
However, if a program is careful to wait for all children, and does not wish
automatic reaping to happen, the programmer can simply turn process
autoreaping off.

Programs that do not wish to use automatic process reaping should be
aware that some scsh routines create subprocesses but do not return
the child's pid: \ex{run/port*}, and its related procedures and
special forms (\ex{run/strings}, \emph{et al.}).
Automatic process reaping will clean the child processes created by
these procedures out of the kernel's process table.
If a program doesn't use process reaping, it should either avoid these
forms, or use \ex{wait-any} to wait for the children to exit.

\subsection{Process waiting}

\defun  {wait}   {proc/pid [flags]}  {status}
\begin{desc}
    This procedure waits until a child process exits, and returns its
    exit code. The \var{proc/pid} argument is either a process object
    (section \ref{sec:proc-objects}) or an integer process id.
    \ex{Wait} returns the child's exit status code (or suspension code,
    if the \ex{wait/stopped-children} option is used, see below).
    Status values can be queried with the procedures in section 
    \ref{sec:wait-codes}.

    The \var{flags} argument is an integer whose bits specify
    additional options. It is composed by or'ing together the following
    flags:
        \begin{center}
        \begin{tabular}{|l|l|}
        \hline
        Flag                    & Meaning                           \\ \hline \hline
        \ex{wait/poll}          & Return {\sharpf} immediately if
                                  child still active.               \\ \hline
        \ex{wait/stopped-children}      & Wait for suspend as well as exit. \\ \hline
        \end{tabular}
        \end{center}
\end{desc}

\begin{defundesc} {wait-any} {[flags]} {[proc status]}
    The optional \var{flags} argument is as for \ex{wait}.
    This procedure waits for any child process to exit (or stop, if the
        \ex{wait/stopped-children} flag is used)
    It returns the process' process object and status code.
    If there are no children left for which to wait, the two values
        \ex{[{\sharpf} {\sharpt}]} are returned.
    If the \ex{wait/poll} flag is used, and none of the children
        are immediately eligble for waiting, 
        then the values \ex{[{\sharpf} {\sharpf}]} are returned:
        \begin{center}
        \begin{tabular}{|l|l|}
        \hline
        [{\sharpf} {\sharpf}] & Poll, none ready \\ \hline
        [{\sharpf} {\sharpt}] & No children      \\ \hline
        \end{tabular}
        \end{center}

    \ex{Wait-any} will not return a process that has been previously waited
    by any other process-wait procedure (\ex{wait}, \ex{wait-any},
    and \ex{wait-process-group}).
    It will return reaped processes that haven't yet been waited.

    The use of \ex{wait-any} is deprecated.
\end{defundesc}

\begin{defundesc} {wait-process-group} {proc/pid [flags]} {[proc status]}
    This procedure waits for any child whose process group is \var{proc/pid}
    (either a process object or a pid).
    The \var{flags} argument is as for \ex{wait}.

    Note that if the programmer wishes to wait for exited processes
    by process group, the program should take care not to use process
    reaping (section \ref{sec:proc-objects}), as this loses
    process group information. However, most process-group waiting is
    for stopped processes (to implement job control), so this is rarely
    an issue, as stopped processes are not subject to reaping.
\end{defundesc}


\subsection{Analysing process status codes}
\label{sec:wait-codes}
When a child process dies (or is suspended), its parent can call the \ex{wait}
procedure to recover the exit (or suspension) status of the child.
The exit status is a small integer that encodes information
describing how the child terminated.
The bit-level format of the exit status is not defined by {\Posix};
you must use the following three functions to decode one.
However, if a child terminates normally with exit code 0,
{\Posix} does require \ex{wait} to return an exit status that is exactly
zero.
So \ex{(zero? \var{status})} is a correct way to test for non-error, 
normal termination, \eg,
\begin{code}
(if (zero? (run (rcp scsh.tar.gz lambda.csd.hku.hk:)))
    (delete-file "scsh.tar.gz"))\end{code}

\defun {status:exit-val}{status}{{\integer} or \sharpf}
\defunx{status:stop-sig}{status}{{\integer} or \sharpf}
\defunx{status:term-sig}{status}{{\integer} or \sharpf}
\begin{desc}
For a given status value produced by calling \ex{wait},
exactly one of these routines will return a true value.

If the child process exited normally, \ex{status:exit-val} returns the
exit code for the child process (\ie, the value the child passed to \ex{exit} 
or returned from \ex{main}). Otherwise, this function returns false.

If the child process was suspended by a signal, \ex{status:stop-sig}
returns the signal that suspended the child.
Otherwise, this function returns false.

If the child process terminated abnormally, \ex{status:term-sig}
returns the signal that terminated the child.
Otherwise, this function returns false.
\end{desc}

%% Dereleased until we have a more portable implementation.

%\defun{halts?}{proc}\boolean
%\begin{desc}
%This procedure, ported from early T implementations, 
%returns true iff \ex{(\var{proc})} returns at all.
%\remark{The current implementation is a constant function returning {\sharpt},
%        which suffices for all {\Unix} implementations of which we are aware.}
%\end{desc}

\section{Process state}

\defun  {umask}{} \fixnum
\defunx {set-umask} {perms} \undefined
\defunx {with-umask*} {perms thunk} {value(s) of thunk}
\dfnx   {with-umask}  {perms . body} {value(s) of body} {syntax}
\begin{desc}
    The process' current umask is retrieved with \ex{umask}, and set with
    \ex{(set-umask \var{perms})}. Calling \ex{with-umask*} changes the umask
    to \var{perms} for the duration of the call to \var{thunk}.  If the
    program throws out of \var{thunk} by invoking a continuation, the umask is
    reset to its external value. If the program throws back into \var{thunk}
    by calling a stored continuation, the umask is restored to the \var{perms}
    value. The special form \ex{with-umask} is equivalent in effect to
    the procedure \ex{with-umask*}, but does not require the programmer
    to explicitly wrap a \ex{(\l{} \ldots)} around the body of the code
    to be executed.
\end{desc}



\defun  {chdir} {[fname]} \undefined
\defunx {cwd}{} \str
\defunx {with-cwd*} {fname thunk}  {value(s) of thunk}
\dfnx   {with-cwd}  {fname . body} {value(s) of body} {syntax}
\begin{desc}
These forms manipulate the current working directory.
The cwd can be changed with \ex{chdir} 
(although in most cases, \ex{with-cwd} is preferrable).
If \ex{chdir} is called with no arguments, it changes the cwd to
the user's home directory.
The \ex{with-cwd*} procedure calls \ex{thunk} with the cwd temporarily
set to \var{fname}; when \var{thunk} returns, or is exited in a non-local
fashion (\eg, by raising an exception or by invoking a continuation),
the cwd is returned to its original value.
The special form \ex{with-cwd} is simply syntactic sugar for \ex{with-cwd*}.
\end{desc}

\defun {pid}{} \fixnum
\defunx {parent-pid}{} \fixnum
\defunx {process-group} {} \fixnum
\defunx {set-process-group} {[proc/pid] pgrp} \undefined % [not implemented]
\begin{desc}    
\ex{(pid)} and \ex{(parent-pid)} retrieve the process id for the 
current process and its parent.
\ex{(process-group)} returns the process group of the current process.
A process' process-group can be set with \ex{set-process-group}; 
the value \var{proc/pid} specifies the affected process. It may be either
a process object or an integer process id, and defaults to the current
process.
\end{desc}

\defun  {set-priority} {which who priority} \undefined %; priority stuff unimplemented
\defunx {priority} {which who} \fixnum % ; not implemented
\defunx {nice} {[proc/pid delta]} \undefined %; not implemented
\begin{desc}
These procedures set and access the priority of processes. 
I can't remember how \ex{set-priority} and \ex{priority} work, so no
    documentation, and besides, they aren't implemented yet, anyway.
\end{desc}

\defunx {user-login-name}{} \str
\defunx {user-uid}{} \fixnum 
\defunx {user-gid}{} \fixnum
\defunx {user-supplementary-gids}{} {{\fixnum} list}
\defunx {set-uid} {uid} \undefined
\defunx {set-gid} {gid} \undefined
\begin{desc}
These routines get and set the effective and real user and group ids.    
The \ex{set-uid} and \ex{set-gid} routines correspond to the {\Posix}
\ex{\urlh{http://www.FreeBSD.org/cgi/man.cgi?query=setuid&apropos=0&sektion=0&manpath=FreeBSD+4.3-RELEASE&format=html}{setuid()}} and \ex{\urlh{http://www.FreeBSD.org/cgi/man.cgi?query=setgid&apropos=0&sektion=0&manpath=FreeBSD+4.3-RELEASE&format=html}{setgid()}} procedures.
\end{desc}

\defunx {user-effective-uid}{} \fixnum
\defunx {set-user-effective-uid}{\fixnum} \undefined
\defunx {with-user-effective-uid*} {\fixnum{} thunk}  {value(s) of thunk}
\dfnx   {with-user-effective-uid}  {\fixnum{} . body} {value(s) of body} {syntax}
\defunx {user-effective-gid}{} \fixnum
\defunx {set-user-effective-gid}{\fixnum} \undefined
\defunx {with-user-effective-gid*} {\fixnum{} thunk}  {value(s) of thunk}
\dfnx   {with-user-effective-gid}  {\fixnum{} . body} {value(s) of body} {syntax}

\begin{desc}
  These forms manipulate the effective user/group IDs. Possible values
  for setting this resource are either the real user/group ID or the
  saved set-user/group-ID. The \texttt{with-...} forms perform the ususal
  temprary assignment during the execution of the second argument. The
  effective user and group IDs are thread-local.
\end{desc}

\defun {process-times} {} {[{\fixnum} {\fixnum} {\fixnum} \fixnum]}
\begin{desc}
Returns four values:
\begin{tightinset}
\begin{flushleft}
        user CPU time in clock-ticks            \\
        system CPU time in clock-ticks          \\
        user CPU time of all descendant processes        \\
        system CPU time of all descendant processes
\end{flushleft}
\end{tightinset}
Note that CPU time clock resolution is not the same as 
the real-time clock resolution provided by \ex{time+ticks}.
That's Unix.
\end{desc}

\defun{cpu-ticks/sec}{} {integer}
\begin{desc}
Returns the resolution of the CPU timer in clock ticks per second.
This can be used to convert the times reported by \ex{process-times}
to seconds.
\end{desc}


%%%%%%%%%%%%%%%%%%%%%%%%%%%%%%%%%%%%%%%%%%%%%%%%%%%%%%%%%%%%%%%%%%%%%%%%%%%%%%%    
\section{User and group database access}
These procedures are used to access the user and group databases
(\eg, the ones traditionally stored in \ex{/etc/passwd} and \ex{/etc/group}.)

\defun  {user-info}  {uid/name}  {record}
\begin{desc}
    Return a \ex{user-info} record giving the recorded information for a
particular user:
\indextt{user-info}
\indextt{user-info:name}
\indextt{user-info:uid}
\indextt{user-info:gid}
\indextt{user-info:home-dir}
\indextt{user-info:shell}
\begin{code}
(define-record user-info
  name uid gid home-dir shell)\end{code}
The \var{uid/name} argument is either an integer uid or a string user-name.
\end{desc}

\defun  {->uid} {uid/name} \fixnum
\defunx {->username} {uid/name}  \str
\begin{desc}
These two procedures coerce integer uid's and user names to a particular
form.
\end{desc}

\defun  {group-info}  {gid/name}  {record}
\begin{desc}
    Return a \ex{group-info} record giving the recorded information for a
particular group:
\indextt{group-info}
\indextt{group-info:name}
\indextt{group-info:gid}
\indextt{group-info:members}
\begin{code}
(define-record group-info
  name gid members)\end{code}
The \var{gid/name} argument is either an integer gid or a string group-name.
\end{desc}

%%%%%%%%%%%%%%%%%%%%%%%%%%%%%%%%%%%%%%%%%%%%%%%%%%%%%%%%%%%%%%%%%%%%%%%%%%%%%%%
\section{Accessing command-line arguments}

\defvar {command-line-arguments}{{\str} list}
\defunx {command-line}{} {{\str} list}
\begin{desc}
The list of strings \ex{command-line-arguments} contains the arguments
passed to the scsh process on the command line.
Calling \ex{(command-line)} returns the complete \ex{argv}
string list, including the program. So if we run a scsh program
        \codex{/usr/shivers/bin/myls -CF src}
then \ex{command-line-arguments} is
        \codex{("-CF" "src")}
and \ex{(command-line)} returns
    \codex{("/usr/shivers/bin/myls" "-CF" "src")}
\ex{command-line} returns a fresh list each time it is called.
In this way, the programmer can get a fresh copy of the original
argument list if \ex{command-line-arguments} has been modified or is lexically
shadowed.
\end{desc}

\defun  {arg}  {arglist n [default]} \str
\defunx {arg*} {arglist n [default-thunk]} \str
\defunx {argv} {n [default]} \str
\begin{desc}
These procedures are useful for accessing arguments from argument
lists.
\ex{arg} returns the $n^{\rm{th}}$ element of \var{arglist}.
The index is 1-based.
If \var{n} is too large, \var{default} is returned; 
if no \var{default}, then an error is signaled.

\ex{arg*} is similar, except that the \var{default-thunk} is called to generate
the default value.

\ex{(argv \var{n})} is simply \ex{(arg (command-line) (+ \var{n} 1))}. 
The +1 offset ensures that the two forms
%
\begin{code}
(arg command-line-arguments \var{n})
(argv \var{n})\end{code}
%
return the same argument
(assuming the user has not rebound or modified \ex{command-line-arguments}).

Example:
%
\begin{code}
(if (null? command-line-arguments)
    (& (xterm -n ,host -title ,host
              -name ,(string-append "xterm_" host)))
    (let* ((progname (file-name-nondirectory (argv 1)))
           (title (string-append host ":" progname)))
      (& (xterm -n     ,title
                -title ,title
                -e     ,@command-line-arguments))))\end{code}
%
A subtlety: when the scsh interpreter is used to execute a scsh program,
the program name reported in the head of the \ex{(command-line)} list
is the scsh program, {\em not} the interpreter.
For example, if we have a shell script in file \ex{fullecho}:
\begin{code}
#!/usr/local/bin/scsh -s
!#
(for-each (\l{arg} (display arg) (display " "))
          (command-line))\end{code}
and we run the program
\codex{fullecho hello world}
the program will print out
\codex{fullecho hello world}
not
\codex{/usr/local/bin/scsh -s fullecho hello world}

This argument line processing ensures that if a scsh program is subsequently
compiled into a standalone executable or byte-compiled to a heap-image
executable by the {\scm} virtual machine, its semantics will be
unchanged---the arglist processing is invariant. In effect, the
        \codex{/usr/local/bin/scsh -s}
is not part of the program; 
it's a specification for the machine to execute the program on, so it is
not properly part of the program's argument list.

\end{desc}

\section{System parameters}

%\defun  {maximum-fds}{}\fixnum
%\defunx {page-size}{}  \fixnum
\defun {system-name}{} \str
\begin{desc}
Returns the name of the host on which we are executing.
This may be a local name, such as ``solar,'' as opposed to a
fully-qualified domain name such as ``solar.csie.ntu.edu.tw.''
\end{desc}

\defun {uname}{} {uname-record}
\begin{desc}
  Returns a \emph{uname-record} of the following structure:
\begin{code}
(define-record uname
   os-name
   node-name
   release
   version
   machine)\end{code}%

Each of the fields contains a string.

Be aware that POSIX limits the length of all entries to 32 characters,
and that the node name does not necessarily correspond to the
fully-qualified domain name.
\end{desc}

\section{Signal system}

Signal numbers are bound to the variables \ex{signal/hup}, \ex{signal/int},
\ldots. See tables~\ref{table:signals-and-interrupts} and 
\ref{table:uncatchable-signals} for the full list.

\defun  {signal-process}   {proc sig}   \undefined
\defunx {signal-process-group} {prgrp sig} \undefined
\begin{desc}
These two procedures send signals to a specific process, and all the processes
in a specific process group, respectively.
The \var{proc} and \var{prgrp} arguments are either processes 
or integer process ids.
\end{desc}

\defun{itimer}{secs} \undefined
\begin{desc}
  Schedules a timer interrupt in \var{secs} seconds.
\end{desc}
\begin{note}
  As the thread system needs the timer interrupt for its own purpose,
  \ex{itimer} works by spawning a thread which calls the interrupt
  handler for \ex{interrupt/alrm} after the specified time.
\end{note}

\defun{process-sleep}{secs} \undefined
\defunx{process-sleep-until}{time}\undefined
\begin{desc}
  The \ex{sleep} procedure causes the process to sleep for \var{secs}
  seconds.  The \ex{sleep-until} procedure causes the process to sleep
  until \var{time} (see section~\ref{sec:time}).  
  
  \note{The use of these procedures is deprecated as they suspend
    \emph{all} running threads, including the ones Scsh uses for
    administrtive purposes. Consider using the \texttt{sleep}
    procedure from the \texttt{thread} package.}
\end{desc}

\subsubsection{Interrupt handlers}
\label{sec:int_handlers}
Scsh interrupt handlers are complicated by the fact that scsh is implemented on
top of the {\scm} virtual machine, which has its own interrupt system, 
independent of the Unix signal system.
This means that {\Unix} signals are delivered in two stages: first,
{\Unix} delivers the signal to the {\scm} virtual machine, then
the {\scm} virtual machine delivers the signal to the executing Scheme program
as a {\scm} interrupt.
This ensures that signal delivery happens between two vm instructions,
keeping individual instructions atomic.

The {\scm} machine has its own set of interrupts, which includes the
asynchronous {\Unix} signals (table~\ref{table:signals-and-interrupts}).
\begin{table}
\begin{minipage}{\textwidth}
\begin{center}
\begin{tabular}{lll}\hline
Interrupt & Unix signal & OS Variant \\ \hline\hline
\exi{interrupt/alrm}\footnote{Also bound to {\scm} interrupt 
                               \exi{interrupt/alarm}.}
                                        & \exi{signal/alrm}     & \Posix \\
%
\exi{interrupt/int}\footnote{Also bound to {\scm} interrupt 
                               \exi{interrupt/keyboard}.}
                                        & \exi{signal/int}      & \Posix \\
%
\exi{interrupt/memory-shortage} & N/A                   &        \\
\exi{interrupt/chld}            & \exi{signal/chld}    & \Posix \\
\exi{interrupt/cont}            & \exi{signal/cont}    & \Posix \\
\exi{interrupt/hup}             & \exi{signal/hup}     & \Posix \\
\exi{interrupt/quit}            & \exi{signal/quit}    & \Posix \\
\exi{interrupt/term}            & \exi{signal/term}    & \Posix \\
\exi{interrupt/tstp}            & \exi{signal/tstp}    & \Posix \\
\exi{interrupt/usr1}            & \exi{signal/usr1}    & \Posix \\
\exi{interrupt/usr2}            & \exi{signal/usr2}    & \Posix \\
\\
\exi{interrupt/info}            & \exi{signal/info}    & BSD only   \\
\exi{interrupt/io}              & \exi{signal/io}      & BSD + SVR4 \\
\exi{interrupt/poll}            & \exi{signal/poll}    & SVR4 only  \\
\exi{interrupt/prof}            & \exi{signal/prof}    & BSD + SVR4 \\
\exi{interrupt/pwr}             & \exi{signal/pwr}     & SVR4 only  \\
\exi{interrupt/urg}             & \exi{signal/urg}     & BSD + SVR4 \\
\exi{interrupt/vtalrm}          & \exi{signal/vtalrm}  & BSD + SVR4 \\
\exi{interrupt/winch}           & \exi{signal/winch}   & BSD + SVR4 \\
\exi{interrupt/xcpu}            & \exi{signal/xcpu}    & BSD + SVR4 \\
\exi{interrupt/xfsz}            & \exi{signal/xfsz}     & BSD + SVR4 \\
\end{tabular}
\end{center}
\caption{{\scm} virtual-machine interrupts and related {\Unix} signals.
        Only the {\Posix} signals are guaranteed to be defined; however,
        your implementation and OS may define other signals and
        interrupts not listed here.}
\end{minipage}
\label{table:signals-and-interrupts}
\end{table}
%
\begin{table}
\begin{center}
\begin{tabular}{lll}\hline
Unix signal & Type & OS Variant \\ \hline\hline
\exi{signal/stop}       & Uncatchable   & \Posix \\
\exi{signal/kill}       & Uncatchable   & \Posix \\
\\
\exi{signal/abrt}       & Synchronous   & \Posix \\
\exi{signal/fpe}        & Synchronous   & \Posix \\
\exi{signal/ill}        & Synchronous   & \Posix \\
\exi{signal/pipe}       & Synchronous   & \Posix \\
\exi{signal/segv}       & Synchronous   & \Posix \\
\exi{signal/ttin}       & Synchronous   & \Posix \\
\exi{signal/ttou}       & Synchronous   & \Posix \\
\\
\exi{signal/bus}        & Synchronous   & BSD + SVR4 \\
\exi{signal/emt}        & Synchronous   & BSD + SVR4 \\
\exi{signal/iot}        & Synchronous   & BSD + SVR4 \\
\exi{signal/sys}        & Synchronous   & BSD + SVR4 \\
\exi{signal/trap}       & Synchronous   & BSD + SVR4 \\
\end{tabular}
\end{center}
\caption{Uncatchable and synchronous {\Unix} signals. While these signals
         may be sent with \texttt{signal-process} or 
         \texttt{signal-process-group},
         there are no corresponding scsh interrupt handlers.
         Only the {\Posix} signals are guaranteed to be defined; however,
         your implementation and OS may define other signals not listed
         here.}
\label{table:uncatchable-signals}
\end{table}
Note that scsh does \emph{not} support signal handlers for
``synchronous'' {\Unix} signals, such as \ex{signal/ill} or
\ex{signal/pipe} (see table~\ref{table:uncatchable-signals}).
Synchronous occurrences of these signals are better handled by raising
a Scheme exception.  We recommend you avoid using signal handlers
unless you absolutely have to; Section \ref{sec:event-interf-interr}
describes a better interface to signals.
\begin{defundesc}{signal->interrupt}{\integer}{\integer}
The programmer maps from {\Unix} signals to {\scm} interrupts with the
\ex{signal->interrupt} procedure.
If the signal does not have a defined {\scm} interrupt, an errror is signaled.
\end{defundesc}


\begin{defundesc}{interrupt-set}{\zeroormore{\integer}}{\integer}
This procedure builds interrupt sets from its interrupt arguments.
A set is represented as an integer using a two's-complement representation of
the bit set.
\end{defundesc}


\defun{enabled-interrupts}{}{interrupt-set}
\defunx{set-enabled-interrupts}{interrupt-set}{interrupt-set}
\begin{desc}
Get and set the value of the enabled-interrupt set.
Only interrupts in this set have their handlers called when delivered.
When a disabled interrupt is delivered to the {\scm} machine, it is
held pending until it becomes enabled, at which time its handler is invoked.

Interrupt sets are represented as integer bit sets (constructed with
the \ex{interrupt-set} function).
The \ex{set-enabled-interrupts} procedure returns the previous value of
the enabled-interrupt set.
\end{desc}

\dfn  {with-enabled-interrupts} {interrupt-set . body} {value(s) of body} {syntax}
\defunx{with-enabled-interrupts*}{interrupt-set thunk} {value(s) of thunk}
\begin{desc}
Run code with a given set of interrupts enabled.
Note that ``enabling'' an interrupt means enabling delivery from
the {\scm} vm to the scsh program.
Using the {\scm} interrupt system is fairly lightweight, and does not involve
actually making a system call.
Note that enabling an interrupt means that the assigned interrupt handler
is allowed to run when the interrupt is delivered.
Interrupts not enabled are held pending when delivered.

Interrupt sets are represented as integer bit sets (constructed with
the \ex{interrupt-set} function).
\end{desc}


\begin{defundesc}{set-interrupt-handler}{interrupt handler}{old-handler}
Assigns a handler for a given interrupt, 
and returns the interrupt's old handler.
The \var{handler} argument is \ex{\#f} (ignore), \ex{\#t} (default), or a
procedure taking an integer argument; 
the return value follows the same conventions.
Note that the \var{interrupt} argument is an interrupt value, 
not a signal value.
An interrupt is delivered to the {\scm} machine by (1) blocking all interrupts,
and (2) applying the handler procedure to the set of interrupts 
that were enabled prior to the interrupt delivery.
If the procedure returns normally (\ie, it doesn't throw to a continuation), 
the set of enabled interrupts will be returned to its previous value.
(To restore the enabled-interrupt set before throwing out of an interrupt
handler, see \ex{set-enabled-interrupts})

\note{If you set a handler for the \ex{interrupt/chld} interrupt,
      you may break scsh's autoreaping process machinery. See the
      discussion of autoreaping in section~\ref{sec:proc-objects}.}
\end{defundesc}

\begin{defundesc}{interrupt-handler}{interrupt}{handler}
Return the handler for a given interrupt.
Note that the argument is an interrupt value, not a signal value.
A handler is either \ex{\#f} (ignore), \ex{\#t} (default), or a
procedure taking an integer argument.
\end{defundesc}

%         %set-unix-signal-handler
%         %unix-signal-handler

%%%%%%%%%%%%%%%%%%%%%%%%%%%%%%%%%%%%%%%%%%%%%%%%%%%%%%%%%%%%%%%%%%%%%%%%%%%%%%%
\section{Time}
\label{sec:time}

Scsh's time system is fairly sophisticated, particularly with respect
to its careful treatment of time zones.
However, casual users shouldn't be intimidated;
all of the complexity is optional, 
and defaulting all the optional arguments reduces the system
to a simple interface.

\subsection{Terminology}
``UTC'' and ``UCT'' stand for ``universal coordinated time,'' which is the 
official name for what is colloquially referred to as ``Greenwich Mean
Time.''

{\Posix} allows a single time zone to specify \emph{two} different offsets
from UTC: one standard one, and one for ``summer time.''
Summer time is frequently some sort of daylight savings time. 

The scsh time package consistently uses this terminology: we never say
``gmt'' or ``dst;'' we always say ``utc'' and ``summer time.''

\subsection{Basic data types}    
We have two types: \emph{time} and \emph{date}.

\index{time}
A \emph{time} specifies an instant in the history of the universe.
It is location and time-zone independent.\footnote{Physics pedants please note:
    The scsh authors live in a Newtonian universe. We disclaim responsibility
    for calculations performed in non-ANSI standard light-cones.}
A time is a real value
giving the number of elapsed seconds since the Unix ``epoch''
(Midnight, January 1, 1970 UTC).
Time values provide arbitrary time resolution,
limited only by the number system of the underlying Scheme system.

\index{date}
A \emph{date} is a name for an instant in time that is specified
relative to some location/time-zone in the world, \eg:
\begin{tightinset}
    Friday October 31, 1994 3:47:21 pm EST.
\end{tightinset}
Dates provide one-second resolution, 
and are expressed with the following record type:
%
\begin{code}\index{date}
(define-record date     ; A Posix tm struct
  seconds       ; Seconds after the minute [0-59]
  minute        ; Minutes after the hour [0-59]
  hour          ; Hours since midnight [0-23]
  month-day     ; Day of the month [1-31]
  month         ; Months since January [0-11]
  year          ; Years since 1900
  tz-name       ; Time-zone name: #f or a string.
  tz-secs       ; Time-zone offset: #f or an integer.
  summer?       ; Summer (Daylight Savings) time in effect?
  week-day      ; Days since Sunday [0-6]
  year-day)     ; Days since Jan. 1 [0-365]\end{code}
%
If the \ex{tz-secs} field is given, it specifies the time-zone's offset from
UTC in seconds. If it is specified, the \ex{tz-name} and \ex{summer?}
fields are ignored when using the date structure to determine a specific
instant in time.

If the \ex{tz-name} field is given, it is a time-zone string such as 
\ex{"EST"} or \ex{"HKT"} understood by the OS.
Since {\Posix} time-zone strings can specify dual standard/summer time-zones
(e.g., "EST5EDT" specifies U.S. Eastern Standard/Eastern Daylight Time),
the value of the \ex{summer?} field is used to resolve the amiguous
boundary cases. For example, on the morning of the Fall daylight savings
change-over, 1:00am--2:00am happens twice. Hence the date 1:30 am
on this morning can specify two different seconds; 
the \ex{summer?} flag says which one.

A date with $\ex{tz-name} = \ex{tz-secs} = \ex{\#f}$ is a date that
is specified in terms of the system's current time zone.

There is redundancy in the \ex{date} data structure.
For example, the \ex{year-day} field is redundant
with the \ex{month-day} and \ex{month} fields.
Either of these implies the values of the \ex{week-day} field.
The \ex{summer?} and \ex{tz-name} fields are redundant with the \ex{tz-secs}
field in terms of specifying an instant in time.
This redundancy is provided because consumers of dates may want it broken out
in different ways.
The scsh procedures that produce date records fill them out completely.
However, when date records produced by the programmer are passed to
scsh procedures, the redundancy is resolved by ignoring some of the
secondary fields. 
This is described for each procedure below.

\defun{make-date} {s min h mday mon y [tzn tzs summ? wday yday]} {date}
\begin{desc}
    When making a \ex{date} record, the last five elements of the record
    are optional, and default to \ex{\#f}, \ex{\#f}, \ex{\#f}, 0, 
    and 0 respectively.
    This is useful when creating a \ex{date} record to pass as an
    argument to \ex{time}. Other procedures, however, may refuse to work
    with these incomplete \ex{date} records.
\end{desc}

\subsection{Time zones}
    Several time procedures take time zones as arguments. When optional,
    the time zone defaults to local time zone. Otherwise the time zone
    can be one of:
\begin{inset}
\begin{tabular}{lp{0.7\linewidth}}
\ex{\#f}        &       Local time \\
Integer         &       Seconds of offset from UTC. For example,
                        New York City is -18000 (-5 hours), San Francisco
                        is -28800 (-8 hours). \\
String          &       A {\Posix} time zone string understood by the OS
                        (\ie., the sort of time zone assigned to the \ex{\$TZ}
                        environment variable).
\end{tabular}
\end{inset}
    An integer time zone gives the number of seconds you must add to UTC 
    to get time in that zone. It is \emph{not} ``seconds west'' of UTC---that
    flips the sign.

    To get UTC time, use a time zone of either 0 or \ex{"UCT0"}.

\subsection{Procedures}
\defun {time+ticks} {} {[secs ticks]}
\defunx{ticks/sec}  {} \real
\begin{desc}
    The current time, with sub-second resolution.
    Sub-second resolution is not provided by {\Posix},
    but is available on many systems.
    The time is returned as elapsed seconds since the Unix epoch, plus
    a number of sub-second ``ticks.''
    The length of a tick may vary from implementation to implementation;
    it can be determined from \ex{(ticks/sec)}.

    The system clock is not required to report time at the full resolution
    given by \ex{(ticks/sec)}. For example, on BSD, time is reported at
    $1\mu$s resolution, so \ex{(ticks/sec)} is 1,000,000. That doesn't mean
    the system clock has micro-second resolution.

    If the OS does not support sub-second resolution, the \var{ticks} value
    is always 0, and \ex{(ticks/sec)} returns 1.

    \begin{remarkenv}
        I chose to represent system clock resolution as ticks/sec
        instead of sec/tick to increase the odds that the value could
        be represented as an exact integer, increasing efficiency and
        making it easier for Scheme implementations that don't have
        sophisticated numeric support to deal with the quantity.

        You can convert seconds and ticks to seconds with the expression
            \codex{(+ \var{secs} (/ \var{ticks} (ticks/sec)))}
        Given that, why not have the fine-grain time procedure just
        return a non-integer real for time? Following Common Lisp, I chose to
        allow the system clock to report sub-second time in its own units to
        lower the overhead of determining the time.  This would be important
        for a system that wanted to precisely time the duration of some
        event. Time stamps could be collected with little overhead, deferring
        the overhead of precisely calculating with them until after collection.
        
        This is all a bit academic for the {\scm} implementation, where
        we determine time with a heavyweight system call, but it's nice
        to plan for the future.
    \end{remarkenv}
\end{desc}

\defun {date} {} {date-record}
\defunx{date} {[time tz]} {date-record}
\begin{desc}
    Simple \ex{(date)} returns the current date, in the local time zone.

    With the optional arguments, \ex{date} converts the time to the date as
    specified by the time zone \var{tz}.
    \var{Time} defaults to the current time; \var{tz} defaults to local time, 
    and is as described in the time-zone section.

    If the \var{tz} argument is an integer, the date's \ex{tz-name}
    field is a {\Posix} time zone of the form
    ``\ex{UTC+\emph{hh}:\emph{mm}:\emph{ss}}''; 
    the trailing \ex{:\emph{mm}:\emph{ss}} portion is deleted if it is zeroes.

    \oops{The Posix facility for converting dates to times, \ex{\urlh{http://www.FreeBSD.org/cgi/man.cgi?query=mktime&apropos=0&sektion=0&manpath=FreeBSD+4.3-RELEASE&format=html}{mktime()}},
          has a broken design: it indicates an error by returning -1, which
          is also a legal return value (for date 23:59:59 UCT, 12/31/1969).
          Scsh resolves the ambiguity in a paranoid fashion: it always
          reports an error if the underlying Unix facility returns -1.
          We feel your pain.
          }
\end{desc}

\defun {time} {} \integer
\defunx{time} {[date]} \integer
\begin{desc}
    Simple \ex{(time)} returns the current time.
    
    With the optional date argument, \ex{time} converts a date to a time.
    \var{Date} defaults to the current date.

    Note that the input \var{date} record is overconstrained.
    \ex{time} ignores \var{date}'s \ex{week-day} and \ex{year-day} fields. 
    If the date's \ex{tz-secs} field is set, the \ex{tz-name} and
    \ex{summer?} fields are ignored.

    If the \ex{tz-secs} field is \ex{\#f}, then the time-zone is taken
    from the \ex{tz-name} field. A false \ex{tz-name} means the system's
    current time zone. When calculating with time-zones, the date's
    \ex{summer?} field is used to resolve ambiguities:
\begin{tightinset}
\begin{tabular}{ll}    
\ex{\#f} &      Resolve an ambiguous time in favor of non-summer time. \\
true     &      Resolve an ambiguous time in favor of summer time.
\end{tabular}
\end{tightinset}
    This is useful in boundary cases during the change-over. For example,
    in the Fall, when US daylight savings time changes over at 2:00 am,
    1:30 am happens twice---it names two instants in time, an hour apart.

    Outside of these boundary cases, the \ex{summer?} flag is ignored. For
    example, if the standard/summer change-overs happen in the Fall and the
    Spring, then the value of \ex{summer?} is ignored for a January or 
    July date. A January date would be resolved with standard time, and a 
    July date with summer time, regardless of the \ex{summer?} value.

    The \ex{summer?} flag is also ignored if the time zone doesn't have 
    a summer time---for example, simple UTC.
\end{desc}


\defun {date->string} {date} \str
\defunx{format-date}  {fmt date} \str
\begin{desc}
    \ex{Date->string} formats the date as a 24-character string of the
    form:
    \begin{tightinset}
    Sun Sep 16 01:03:52 1973            
    \end{tightinset}
    
    \ex{Format-date} formats the date according to the format string
    \var{fmt}. The format string is copied verbatim, except that tilde
    characters indicate conversion specifiers that are replaced by fields from
    the date record.  Figure \ref{fig:dateconv} gives the full set of
    conversion specifiers supported by \ex{format-date}.

\begin{boxedfigure}{tbp}
    \renewcommand{\arraystretch}{1.25}
    \begin{tabular}{l>{\raggedrightparbox}p{0.9\linewidth}}
      \verb|~~| &   Converted to the \verb|~| character. \\
      \verb|~a| &   abbreviated weekday name \\
      \verb|~A| &   full weekday name \\
      \verb|~b| &   abbreviated month name \\
      \verb|~B| &   full month name \\
      \verb|~c| &   time and date using the time and date representation 
                    for the locale (\verb|~X ~x|) \\
      \verb|~d| &   day of the month as a decimal number (01-31) \\
      \verb|~H| &   hour based on a 24-hour clock
                    as a decimal number (00-23) \\
      \verb|~I| &   hour based on a 12-hour clock
                    as a decimal number (01-12) \\
      \verb|~j| &   day of the year as a decimal number (001-366) \\
      \verb|~m| &   month as a decimal number (01-12) \\
      \verb|~M| &   minute as a decimal number (00-59) \\
      \verb|~p| &   AM/PM designation associated with a 12-hour clock \\
      \verb|~S| &   second as a decimal number (00-61) \\
      \verb|~U| &   week number of the year; 
                    Sunday is first day of week (00-53) \\
      \verb|~w| &   weekday as a decimal number (0-6), where Sunday is 0 \\
      \verb|~W| &   week number of the year;
                    Monday is first day of week (00-53) \\
      \verb|~x| &   date using the date representation for the locale \\
      \verb|~X| &   time using the time representation for the locale \\
      \verb|~y| &   year without century (00-99) \\
      \verb|~Y| &   year with century (\eg 1990) \\
      \verb|~Z| &   time zone name or abbreviation, or no characters
                    if no time zone is determinable
    \end{tabular}

\caption{\texttt{format-date} conversion specifiers}
\label{fig:dateconv}
\end{boxedfigure}
\end{desc}
    
%\defun{utc-offset} {[time tz]} \integer
%\begin{desc}    
%    Returns the offset from UTC of time zone \var{tz} at instant \var{time}. 
%    \var{time} defaults to the current time; \var{tz} defaults to local time, 
%    and is as described in the time-zone section.
%
%    The offset is the number of seconds you add to UTC time to get
%    local time.
%
%    Note: Be aware that other time interfaces (\eg, the BSD C interface)
%    give offsets as seconds \emph{west} of UTC, which flips the sign. The scsh
%    definition is chosen for arithmetic simplicity. It's easy to remember
%    the definition of the offset: what you add to UTC to get local.
%\end{desc}
%
%\defun{time-zone} {[summer? tz]} \str
%\begin{desc}    
%    Returns the name of the time zone as a string. \var{Summer?} is
%    used to choose between the summer name and the standard name 
%    (\eg, ``EST'' and ``EDT'')\@. \var{Summer?} is interpreted as follows:
%    \begin{inset}
%    \begin{tabular}{lp{0.7\linewidth}}
%       Integer          &      A time value. 
%                               The variant in use at that time is returned. \\
%       \ex{\#f}         &      The standard time name is returned. \\
%       \emph{Otherwise} &      The summer time name is returned.
%    \end{tabular}
%    \end{inset}
%    \ex{Summer?} defaults to the case that pertains at the time of the call.
%    It is ignored if the time zone doesn't have a summer variant.
%\end{desc}

\dfni {fill-in-date!}{date}{date}{procedure}
        {fill-in-date"!@\texttt{fill-in-date"!}}
\begin{desc}
This procedure fills in missing, redundant slots in a date record.
In decreasing order of priority:
\begin{itemize}
\itum{year, month, month-day $\Rightarrow$ year-day}
      If the \ex{year}, \ex{month}, and \ex{month-day} fields are all
      defined (are all integers), the \ex{year-day}
      field is set to the corresponding value.
\itum{year, year-day $\Rightarrow$ month, month-day}
      If the \ex{month} and \ex{month-day} fields aren't set, but
      the \ex{year} and \ex{year-day} fields are set, then
      \ex{month} and \ex{month-day} are calculated.
\itum{year, month, month-day, year-day $\Rightarrow$ week-day}
      If either of the above rules is able to determine what day it is,
      the \ex{week-day} field is then set.
\itum{tz-secs $\Rightarrow$ tz-name}
      If \ex{tz-secs} is defined, but \ex{tz-name} is not, it is assigned
      a time-zone name of the form ``\ex{UTC+\emph{hh}:\emph{mm}:\emph{ss}}''; 
      the trailing \ex{:\emph{mm}:\emph{ss}} portion is deleted if it 
      is zeroes.
\itum{tz-name, date, summer? $\Rightarrow$ tz-secs, summer?}
      If the date information is provided up to second resolution,
      \ex{tz-name} is also provided, and \ex{tz-secs} is not set,
      then \ex{tz-secs} and \ex{summer?} are set to their correct values. 
      Summer-time ambiguities are resolved using the original value of
      \ex{summer?}. If the time zone doesn't have a
      summer time variant, then \ex{summer?} is set to \ex{\#f}.
\itum{local time, date, summer? $\Rightarrow$ tz-name, tz-secs, summer?}
      If the date information is provided up to second resolution,
      but no time zone information is provided (both \ex{tz-name} and
      \ex{tz-secs} aren't set), then we proceed as in the above case,
      except the system's current time zone is used.
\end{itemize}
These rules allow one particular ambiguity to escape:
if both \ex{tz-name} and \ex{tz-secs} are set, they are not brought
into agreement. It isn't clear how to do this, nor is it clear which
one should take precedence.

\oops{\ex{fill-in-date!} isn't implemented  yet.}

\end{desc}


\section{Environment variables}

\defun  {setenv} {var val} \undefined
\defunx {getenv} {var}     \str
\begin{desc}
These functions get and set the process environment, stored in the
external C variable \ex{char **environ}.
An environment variable \var{var} is a string.
If an environment variable is set to a string \var{val}, 
then the process' global environment structure is altered with an entry 
of the form \ex{"\var{var}=\var{val}"}.
If \var{val} is {\sharpf}, then any entry for \var{var} is deleted.
\end{desc}

\defun {env->alist}{}      {{\str$\rightarrow$\str} alist}
\begin{desc}
    The \ex{env->alist} procedure converts the entire environment into
    an alist, \eg,
\begin{code}
(("TERM" . "vt100")
 ("SHELL" . "/usr/local/bin/scsh") 
 ("PATH" . "/sbin:/usr/sbin:/bin:/usr/bin")
 ("EDITOR" . "emacs") 
 \ldots)\end{code}
\end{desc}

\defun {alist->env} {alist} \undefined
\begin{desc}
    \var{Alist} must be an alist whose keys are all strings, and whose values
    are all either strings or string lists. String lists are converted to
    colon lists (see below). The alist is installed as the current {\Unix}
    environment (\ie, converted to a null-terminated C vector of
    \ex{"\var{var}=\var{val}"} strings which is assigned to the global
    \ex{char **environ}).

\begin{code}
;;; Note $PATH entry is converted 
;;; to /sbin:/usr/sbin:/bin:/usr/bin.
(alist->env '(("TERM" . "vt100")
              ("PATH" "/sbin" "/usr/sbin" "/bin")
              ("SHELL" . "/usr/local/bin/scsh")))
\end{code}

Note that \ex{env->alist} and \ex{alist->env} are not exact 
inverses---\ex{alist->env} will convert a list value into a single
colon-separated string, but \ex{env->alist} will not parse colon-separated
values into lists. (See the \ex{\$PATH} element in the examples given for
each procedure.)

\end{desc}

The following three functions help the programmer manipulate alist
tables in some generally useful ways. They are all defined using
\ex{equal?} for key comparison.

\begin{defundesc} {alist-delete} {key alist} {alist}
    Delete any entry labelled by value \var{key}.
\end{defundesc}

\begin{defundesc} {alist-update} {key val alist} {alist}
    Delete \var{key} from \var{alist}, then cons on a
    \ex{(\var{key} . \var{val})} entry.
\end{defundesc}

\defun{alist-compress} {alist} {alist}
\begin{desc}
    Compresses \var{alist} by removing shadowed entries.
    Example:
\begin{code}
;;; Shadowed (1 . c) entry removed.
(alist-compress '( (1 . a) (2 . b) (1 . c) (3 . d) ))
    {\evalto} ((1 . a) (2 . b) (3 . d))\end{code}
\end{desc}

\defun  {with-env*} {env-alist-delta thunk} {value(s) of thunk}
\defunx {with-total-env*} {env-alist thunk} {value(s) of thunk}
\begin{desc}
    These procedures call \var{thunk} in the context of an altered
    environment. They return whatever values \var{thunk} returns.
    Non-local returns restore the environment to its outer value;
    throwing back into the thunk by invoking a stored continuation
    restores the environment back to its inner value.

    The \var{env-alist-delta} argument specifies
    a \emph{modification} to the current en\-vi\-ron\-ment---\var{thunk}'s
    environment is the original environment overridden with the
    bindings specified by the alist delta.

    The \var{env-alist} argument specifies a complete environment
    that is installed for \var{thunk}.
\end{desc}

\dfn  {with-env} {env-alist-delta . body} {value(s) of body} {syntax}
\dfnx {with-total-env} {env-alist . body} {value(s) of body} {syntax}
\begin{desc}
    These special forms provide syntactic sugar for \ex{with-env*}
    and {\ttt with\=total\=env*}. 
    The env alists are not evaluated positions, but are implicitly backquoted.
    In this way, they tend to resemble binding lists for \ex{let} and
    \ex{let*} forms.
\end{desc}

Example: These four pieces of code all run the mailer with special 
\cd{$TERM} and \cd{$EDITOR} values.
{\small
\begin{code}
(with-env (("TERM" . "xterm") ("EDITOR" . ,my-editor))
  (run (mail shivers@lcs.mit.edu)))
\cb
(with-env* `(("TERM" . "xterm") ("EDITOR" . ,my-editor))
  (\l{} (run (mail shivers@csd.hku.hk))))
\cb
(run (begin (setenv "TERM" "xterm")      ; Env mutation happens
            (setenv "EDITOR" my-editor) ; in the subshell.
            (exec-epf (mail shivers@research.att.com))))
\cb
;; In this example, we compute an alternate environment ENV2
;; as an alist, and install it with an explicit call to the
;; EXEC-PATH/ENV procedure.
(let* ((env (env->alist))           ; Get the current environment,
       (env1 (alist-update env  "TERM" "xterm"))      ; and compute
       (env2 (alist-update env1 "EDITOR" my-editor))) ; the new env.
  (run (begin (exec-path/env "mail" env2 "shivers@cs.cmu.edu"))))\end{code}}

\subsection{Path lists and colon lists}

When environment variables such as \ex{\$PATH} need to encode a list of
strings (such as a list of directories to be searched), 
the common Unix convention is to separate the list elements with 
colon delimiters.\footnote{\ldots and hope the individual list elements 
don't contain colons themselves.}
To convert between the colon-separated string encoding and the
list-of-strings representation, see the \ex{infix-splitter} function
(section~\ref{sec:field-splitter}) and the string library's
\ex{string-join} function.
For example,
\begin{code}
(define split (infix-splitter (rx ":")))
(split "/sbin:/bin::/usr/bin") {\evalsto} 
    '("/sbin" "/bin" "" "/usr/bin")
(string-join ":" '("/sbin" "/bin" "" "/usr/bin")) {\evalsto}
    "/sbin:/bin::/usr/bin"\end{code}
The following two functions are useful for manipulating these ordered lists,
once they have been parsed from their colon-separated form.

%\remark{An earlier release of scsh provided the \ex{split-colon-list}
%        and \ex{string-list->colon-list} functions. These have been
%       removed from scsh, and are replaced by the more general
%       parsers and unparsers of the field-reader module.}
%
%\defun  {split-colon-list} {string} {{\str} list}
%\defunx {string-list->colon-list} {string-list} \str
%\begin{desc}
%    Many {\Unix} lists, such as the \cd{$PATH} search path, 
%    are stored as ``colon lists.''
%    A colon list is a string containing elements delimited by colon characters.
%    These functions provide conversions between colon lists and true
%    {\Scheme} lists.
%%
%\begin{code}
%(split-colon-list "/foo:/bar::/usr/tmp") \evalto
%    ("/foo" "/bar" "" "/usr/tmp")\end{code}
%%
%    \ex{string-list->colon-list} is the inverse function.
%
%    \ex{with-env*}, \ex{with-total-env*}, and \ex{alist->env} all coerce
%    string lists to colon lists where appropriate.
%\end{desc}

\defun  {add-before} {elt before list} {list}
\defunx {add-after}  {elt after list}  {list}
\begin{desc}
    These functions are for modifying search-path lists, where element order
    is significant.

    \ex{add-before} adds \var{elt} to the list immediately
    before the first occurrence of \var{before} in the list.
    If \var{before} is not in the list, \var{elt} is added to the end
    of the list.

    \ex{add-after} is similar:
    \var{elt} is added after the last occurrence of \var{after}.
    If \var{after} is not found, 
    \var{elt} is added to the beginning of the list.

    Neither function destructively alters the original path-list.
    The result may share structure with the original list.
    Both functions use \ex{equal?} for comparing elements.
\end{desc}

    
%%%%%%%%%%%%%%%%%%%%%%%%%%%%%%%%%%%%%%%%%%%%%%%%%%%%%%%%%%%%%%%%%%%%%%%%%%%%%%%
\subsection{\protect{\tt\$USER}, \protect{\tt\$HOME}, and \protect{\tt\$PATH}}

Like sh and unlike csh, scsh has \emph{no} interactive dependencies on
environment variables.
It does, however, initialise certain internal values at startup time from the
initial process environment, in particular \cd{$HOME} and \cd{$PATH}.
Scsh never uses \cd{$USER} at all.
It computes \ex{(user-login-name)} from the system call \ex{(user-uid)}.

\defvar  {home-directory} \str
\defvarx {exec-path-list} {{\str} list thread-fluid}
\begin{desc}
    Scsh accesses \cd{$HOME} at start-up time, and stores the value in the
    global variable \ex{home-directory}. It uses this value for \ex{\~}
    lookups and for returning to home on \ex{(chdir)}.

    Scsh accesses \cd{$PATH} at start-up time, colon-splits the path list, and
    stores the value in the thread fluid \ex{exec-path-list}. This list is
    used for \ex{exec-path} and \ex{exec-path/env} searches.

    To access, rebind or side-effect thread-fluid cells, you must open
    the \ex{thread-fluids} package.
\end{desc}

%%%%%%%%%%%%%%%%%%%%%%%%%%%%%%%%%%%%%%%%%%%%%%%%%%%%%%%%%%%%%%%%%%%%%%%%%%%%%%%
%&latex -*- latex -*-
% Fix OXTABS footnote bug
% Figures should be dumped out earlier? Pack two to a page?

\section{Terminal device control}
\label{sect:tty}

\newcommand{\fr}[1]{\makebox[0pt][r]{#1}}

Scsh provides a complete set of routines for manipulating terminal
devices---putting them in ``raw'' mode, changing and querying their
special characters, modifying their I/O speeds, and so forth.
The scsh interface is designed both for generality and portability
across different Unix platforms, so you don't have to rewrite your
program each time you move to a new system.
We've also made an effort to use reasonable, Scheme-like names for
the multitudinous named constants involved, so when you are reading
code, you'll have less likelihood of getting lost in a bewildering
maze of obfuscatory constants named \ex{ICRNL}, \ex{INPCK}, \ex{IUCLC},
and \ex{ONOCR}.

This section can only lay out the basic functionality of the terminal
device interface.
For further details, see the termios(3) man page on your system,
or consult one of the standard {\Unix} texts.

\subsection{Portability across OS variants}
Terminal-control software is inescapably complex, ugly, and low-level.
Unix variants each provide their own way of controlling terminal
devices, making it difficult to provide interfaces that are
portable across different Unix systems.
Scsh's terminal support is based primarily upon the {\Posix} termios
interface.
Programs that can be written using only the {\Posix} interface are likely
to be widely portable.

The bulk of the documentation that follows consists of several pages worth
of tables defining different named constants that enable and disable different
features of the terminal driver.
Some of these flags are {\Posix}; others are taken from the two common
branches of Unix development, SVR4 and 4.3+ Berkeley.
Scsh guarantees that the non-{\Posix} constants will be bound identifiers.
\begin{itemize}
\item If your OS supports a particular non-{\Posix} flag, 
      its named constant will be bound to the flag's value.
\item If your OS doesn't support the flag, its named constant
      will be present, but bound to \sharpf.
\end{itemize}
This means that if you want to use SVR4 or Berkeley features in a program,
your program can portably test the values of the flags before using 
them---the flags can reliably be referenced without producing OS-dependent
``unbound variable'' errors.

Finally, note that although {\Posix}, SVR4, and Berkeley cover the lion's
share of the terminal-driver functionality, 
each operating system inevitably has non-standard extensions.
While a particular scsh implementation may provide these extensions,
they are not portable, and so are not documented here.

\subsection{Miscellaneous procedures}
\defun{tty?}{fd/port}{\boolean}
\begin{desc}
Return true if the argument is a tty.
\end{desc}

\defun{tty-file-name}{fd/port}{\str}
\begin{desc}
The argument \var{fd/port} must be a file descriptor or port open on a tty.
Return the file-name of the tty.
\end{desc}

\subsection{The tty-info record type}

The primary data-structure that describes a terminal's mode is
a \ex{tty-info} record, defined as follows:
\index{tty-info record type}
\indextt{tty-info:control-chars}
\indextt{tty-info:input-flags}
\indextt{tty-info:output-flags}
\indextt{tty-info:control-flags}
\indextt{tty-info:local-flags}
\indextt{tty-info:input-speed}
\indextt{tty-info:output-speed}
\indextt{tty-info:min}
\indextt{tty-info:time}
\indextt{tty-info?}
\begin{code}
(define-record tty-info
  control-chars  ; String: Magic input chars
  input-flags    ; Int: Input processing
  output-flags   ; Int: Output processing
  control-flags  ; Int: Serial-line control
  local-flags    ; Int: Line-editting UI
  input-speed    ; Int: Code for input speed
  output-speed   ; Int: Code for output speed
  min            ; Int: Raw-mode input policy
  time)          ; Int: Raw-mode input policy\end{code}

\subsubsection{The control-characters string}
The \ex{control-chars} field is a character string;
its characters may be indexed by integer values taken from 
table~\ref{table:ttychars}.

As discussed above, 
only the {\Posix} entries in table~\ref{table:ttychars} are guaranteed
to be legal, integer indices.
A program can reliably test the OS to see if the non-{\Posix} 
characters are supported by checking the index constants.
If the control-character function is supported by the terminal driver, 
then the corresponding index will be bound to an integer;
if it is not supported, the index will be bound to \sharpf.

To disable a given control-character function, set its corresponding
entry in the \ex{tty-info:control-chars} string to the
special character \exi{disable-tty-char} 
(and then use the \ex{(set-tty-info \var{fd/port} \var{info})} procedure
to update the terminal's state).

\subsubsection{The flag fields}
The \ex{tty-info} record's \ex{input-flags}, \ex{output-flags},
\ex{control-flags}, and \ex{local-flags} fields are all bit sets
represented as two's-complement integers.
Their values are composed by or'ing together values taken from
the named constants listed in tables~\ref{table:ttyin} 
through \ref{table:ttylocal}.

As discussed above, 
only the {\Posix} entries listed in these tables are guaranteed
to be legal, integer flag values.
A program can reliably test the OS to see if the non-{\Posix} 
flags are supported by checking the named constants.
If the feature is supported by the terminal driver, 
then the corresponding flag will be bound to an integer;
if it is not supported, the flag will be bound to \sharpf.

%%%%% I managed to squeeze this into the DEFINE-RECORD's comments.
% Here is a small table classifying the four flag fields by
% the kind of features they determine:
% \begin{center}
% \begin{tabular}{|ll|}\hline
% Field                 & Affects \\ \hline \hline
% \ex{input-flags}      & Processing of input chars \\
% \ex{output-flags}     & Processing of output chars \\
% \ex{control-flags}    & Controlling of terminal's serial line \\
% \ex{local-flags}      & Details of the line-editting user interface \\
% \hline
% \end{tabular}
% \end{center}

%%%
%%% The figures used to go here.
%%%

\subsubsection{The speed fields}
The \ex{input-speed} and \ex{output-speed} fields determine the
I/O rate of the terminal's line.
The value of these fields is an integer giving the speed
in bits-per-second.
The following speeds are supported by {\Posix}:
\begin{center}
\begin{tabular}{rrrr}
0        & 134 & 600  & 4800  \\
50       & 150 & 1200 & 9600  \\
75       & 200 & 1800 & 19200 \\
110      & 300 & 2400 & 38400 \\
\end{tabular}
\end{center}
Your OS may accept others; it may also allow the special symbols
\ex{'exta} and \ex{'extb}.

\subsubsection{The min and time fields}
The integer \ex{min} and \ex{time} fields determine input blocking
behaviour during non-canonical (raw) input; otherwise, they are ignored.
See the termios(3) man page for further details.

Be warned that {\Posix} allows the base system call's representation
of the \ex{tty-info} record to share storage for the \ex{min} field
and the \ex{ttychar/eof} element of the control-characters string,
and for the \ex{time} field and the \ex{ttychar/eol} element
of the control-characters string.
Many implementations in fact do this.

To stay out of trouble, set the \ex{min} and \ex{time} fields only
if you are putting the terminal into raw mode;
set the eof and eol control-characters only if you are putting
the terminal into canonical mode.
It's ugly, but it's {\Unix}.

\subsection{Using tty-info records}

\defun{make-tty-info}{if of cf lf ispeed ospeed min time}
                   {tty-info-record}
\defunx{copy-tty-info}{tty-info-record}{tty-info-record}
\begin{desc}
These procedures make it possible to create new \ex{tty-info} records.
The typical method for creating a new record is to copy one retrieved
by a call to the \ex{tty-info} procedure, then modify the copy as desired.
Note that the \ex{make-tty-info} procedure does not take a parameter
to define the new record's control characters.\footnote{
        Why? Because the length of the string varies from Unix to Unix.
        For example, the word-erase control character (typically control-w)
        is provided by most Unixes, but not part of the {\Posix} spec.}
Instead, it simply returns a \ex{tty-info} record whose control-character
string has all elements initialised to {\Ascii} nul.
You may then install the special characters by assigning to the string.
Similarly, the control-character string in the record produced by
\ex{copy-tty-info} does not share structure with the string in the record
being copied, so you may mutate it freely.
\end{desc}


\defun{tty-info}{[fd/port/fname]}{tty-info-record}
\begin{desc}
The \var{fd/port/fname} parameter is an integer file descriptor or 
Scheme I/O port opened on a terminal device, 
or a file-name for a terminal device; it defaults to the current input port.
This procedure returns a \ex{tty-info} record describing the terminal's
current mode.
\end{desc}

\defun {set-tty-info/now}  {fd/port/fname info}{no-value}
\defunx{set-tty-info/drain}{fd/port/fname info}{no-value}
\defunx{set-tty-info/flush}{fd/port/fname info}{no-value}
\begin{desc}
The \var{fd/port/fname} parameter is an integer file descriptor or 
Scheme I/O port opened on a terminal device, 
or a file-name for a terminal device.
The procedure chosen determines when and how the terminal's mode is altered:
\begin{center}
\begin{tabular}{|ll|} \hline
Procedure & Meaning \\ \hline \hline
\ex{set-tty-info/now}   & Make change immediately. \\
\ex{set-tty-info/drain} & Drain output, then change. \\
\ex{set-tty-info/flush} & Drain output, flush input, then change. \\ \hline
\end{tabular}
\end{center}
\oops{If I had defined these with the parameters in the reverse order,
      I could have made \var{fd/port/fname} optional. Too late now.}
\end{desc}

\subsection{Other terminal-device procedures}
\defun{send-tty-break}{[fd/port/fname duration]}{no-value}
\begin{desc}
The \var{fd/port/fname} parameter is an integer file descriptor or 
Scheme I/O port opened on a terminal device, 
or a file-name for a terminal device; it defaults to the current output port.
Send a break signal to the designated terminal.
A break signal is a sequence of continuous zeros on the terminal's transmission
line.

The \var{duration} argument determines the length of the break signal.
A zero value (the default) causes a break of between 
0.25 and 0.5 seconds to be sent;
other values determine a period in a manner that will depend upon local
community standards.
\end{desc}

\defun{drain-tty}{[fd/port/fname]}{no-value}
\begin{desc}
The \var{fd/port/fname} parameter is an integer file descriptor or 
Scheme I/O port opened on a terminal device, 
or a file-name for a terminal device; it defaults to the current output port.

This procedure waits until all the output written to the
terminal device has been transmitted to the device.
If \var{fd/port/fname} is an output port with buffered I/O
enabled, then the port's buffered characters are flushed before
waiting for the device to drain.
\end{desc}

\defun {flush-tty/input} {[fd/port/fname]}{no-value}
\defunx{flush-tty/output}{[fd/port/fname]}{no-value}
\defunx{flush-tty/both}  {[fd/port/fname]}{no-value}
\begin{desc}
The \var{fd/port/fname} parameter is an integer file descriptor or 
Scheme I/O port opened on a terminal device, 
or a file-name for a terminal device; it defaults to the current input
port (\ex{flush-tty/input} and \ex{flush-tty/both}),
or output port (\ex{flush-tty/output}).

These procedures discard the unread input chars or unwritten
output chars in the tty's kernel buffers. 
\end{desc}

\defun {start-tty-output}{[fd/port/fname]} {no-value}
\defunx{stop-tty-output} {[fd/port/fname]} {no-value}
\defunx{start-tty-input} {[fd/port/fname]} {no-value}
\defunx{stop-tty-input}  {[fd/port/fname]} {no-value}
\begin{desc}
These procedures can be used to control a terminal's input and output flow.
The \var{fd/port/fname} parameter is an integer file descriptor or 
Scheme I/O port opened on a terminal device, 
or a file-name for a terminal device; it defaults to the current input
or output port.

The \ex{stop-tty-output} and \ex{start-tty-output} procedures suspend
and resume output from a terminal device.
The \ex{stop-tty-input} and \ex{start-tty-input} procedures transmit
the special STOP and START characters to the terminal with the intention
of stopping and starting terminal input flow.
\end{desc}

% --- Obsolete ---
% \defun {encode-baud-rate}{speed}{code}
% \defunx{decode-baud-rate}{code}{speed}
% \begin{desc}
% These procedures can be used to map between the special codes
% that are legal values for the \ex{tty-info:input-speed} and
% \ex{tty-info:output-speed} fields, and actual integer bits-per-second speeds.
% The codes are the values bound to the
% \ex{baud/4800}, \ex{baud/9600}, and other named constants defined above.
% For example:
% \begin{code}
% (decode-baud-rate baud/9600) {\evalto} 9600
% 
% ;;; These two expressions are identical:
% (set-tty-info:input-speed ti baud/14400)
% (set-tty-info:input-speed ti (encode-baud-rate 14400))\end{code}
% \end{desc}

%%%%%%%%%%%%%%%%%%%%%%%%%%%%%%%%%%%%%%%%%%%%%%%%%%%%%%%%%%%%%%%%%%%%%%%%%%%%%%%
\subsection{Control terminals, sessions, and terminal process groups}

\defun{open-control-tty}{tty-name [flags]}{port}
\begin{desc}
This procedure opens terminal device \var{tty-name} as the process'
control terminal 
(see the \ex{termios} man page for more information on control terminals).
The \var{tty-name} argument is a file-name such as \ex{/dev/ttya}.
The \var{flags} argument is a value suitable as the second argument
to the \ex{open-file} call; it defaults to \ex{open/read+write}, causing
the terminal to be opened for both input and output.

The port returned is an input port if the \var{flags} permit it, 
otherwise an output port. 
\RnRS/\scm/scsh do not have input/output ports,
so it's one or the other. 
However, you can get both read and write ports open on a terminal
by opening it read/write, taking the result input port,
and duping it to an output port with \ex{dup->outport}.

This procedure guarantees to make the opened terminal the
process' control terminal only if the process does not have
an assigned control terminal at the time of the call.
If the scsh process already has a control terminal, the results are undefined.

To arrange for the process to have no control terminal prior to calling
this procedure, use the \ex{become-session-leader} procedure.

%\oops{The control terminal code was added just before release time
%      for scsh release 0.4. Control terminals are one of the less-standardised
%      elements of Unix. We can't guarantee that the terminal is definitely
%      attached as a control terminal; we were only able to test this out
%      on HP-UX. If you intend to use this feature on your OS, you should
%      test it out first. If your OS requires the use of the \ex{TIOCSCTTY}
%      \ex{ioctl}, uncomment the appropriate few lines of code in the
%      file \ex{tty1.c} and send us email.}
\end{desc}

\defun{become-session-leader}{}{\integer}
\begin{desc}
This is the C \ex{setsid()} call.
{\Posix} job-control has a three-level hierarchy:
session/process-group/process. 
Every session has an associated control terminal.
This procedure places the current process into a brand new session,
and disassociates the process from any previous control terminal.
You may subsequently use \ex{open-control-tty} to open a new control
terminal.

It is an error to call this procedure if the current process is already
a process-group leader.
One way to guarantee this is not the case is only to call this procedure
after forking.
\end{desc}


\defun {tty-process-group}{fd/port/fname}{\integer}
\defunx{set-tty-process-group}{fd/port/fname pgrp}{\undefined}
\begin{desc}
This pair of procedures gets and sets the process group of a given
terminal.
\end{desc}

\defun{control-tty-file-name}{}{\str}
\begin{desc}
Return the file-name of the process' control tty.
On every version of Unix of which we are aware, this is just the string
\ex{"/dev/tty"}.
However, this procedure uses the official Posix interface, so it is more
portable than simply using a constant string.
\end{desc}


%%%%%%%%%%%%%%%%%%%%%%%%%%%%%%%%%%%%%%%%%%%%%%%%%%%%%%%%%%%%%%%%%%%%%%%%%%%%%%%
\subsection{Pseudo-terminals}
Scsh implements an interface to Berkeley-style pseudo-terminals.

\defun{fork-pty-session}{thunk}{[process pty-in pty-out tty-name]}
\begin{desc}
This procedure gives a convenient high-level interface to pseudo-terminals.
It first allocates a pty/tty pair of devices, and then forks a child
to execute procedure \var{thunk}.
In the child process
\begin{itemize}
\item Stdio and the current I/O ports are bound to the terminal device.
\item The child is placed in its own, new session
            (see \ex{become\=session\=leader}).
\item The terminal device becomes the new session's controlling terminal
            (see \ex{open-control-tty}).
\item The \ex{(error-output-port)} is unbuffered.
\end{itemize}

The \ex{fork-pty-session} procedure returns four values:
the child's process object, two ports open on the controlling pty device,
and the name of the child's corresponding terminal device.
\end{desc}

\defun{open-pty}{}{pty-inport tty-name}
\begin{desc}
This procedure finds a free pty/tty pair, and opens the pty device
with read/write access.
It returns a port on the pty, 
and the name of the corresponding terminal device.

The port returned is an input port---Scheme doesn't allow input/output
ports.
However, you can easily use \ex{(dup->outport \var{pty-inport})}
to produce a matching output port.
You may wish to turn off I/O buffering for this output port.
\end{desc}


\defun {pty-name->tty-name}{pty-name}{tty-name}
\defunx{tty-name->pty-name}{tty-name}{pty-name}
\begin{desc}
These two procedures map between corresponding terminal and pty controller
names.
For example,
\begin{code}
(pty-name->tty-name "/dev/ptyq3") {\evalto} "/dev/ttyq3"
(tty-name->pty-name "/dev/ttyrc") {\evalto} "/dev/ptyrc"\end{code}

\remark{This is rather Berkeley-specific. SVR4 ptys are rare enough that
        I've no real idea if it generalises across the Unix gap. Experts
        are invited to advise. Users feel free to not worry---the predominance
        of current popular Unix systems use Berkeley ptys.}
\end{desc}

\defunx{make-pty-generator}{}{\proc}
\begin{desc}
This procedure returns a generator of candidate pty names.
Each time the returned procedure is called, it produces a
new candidate.
Software that wishes to search through the set of available ptys
can use a pty generator to iterate over them.
After producing all the possible ptys, a generator returns {\sharpf}
every time it is called.
Example:
\begin{code}
(define pg (make-pty-generator))
(pg) {\evalto} "/dev/ptyp0"
(pg) {\evalto} "/dev/ptyp1"
        \vdots
(pg) {\evalto} "/dev/ptyqe"
(pg) {\evalto} "/dev/ptyqf"    \textit{(Last one)}
(pg) {\evalto} {\sharpf}
(pg) {\evalto} {\sharpf}
        \vdots\end{code}
\end{desc}


% Flag tables
%%%%%%%%%%%%%%%%%%%%%%%%%%%%%%%%%%%%%%%%%%%%%%%%%%%%%%%%%%%%%%%%%%%%%%%%%%%%%%%

% Control-chars indices
%%%%%%%%%%%%%%%%%%%%%%%
\begin{table}[p]
\begin{center}
\begin{tabular}{|lll|} \hline
Scsh & C & Typical char \\
\hline\hline
{\Posix} & & \\
\exi{ttychar/delete-char}       & \ex{ERASE}    & del \\
\exi{ttychar/delete-line}       & \ex{KILL}     & \verb|^U| \\
\exi{ttychar/eof}               & \ex{EOF}      & \verb|^D| \\
\exi{ttychar/eol}               & \ex{EOL}      & \\
\exi{ttychar/interrupt}         & \ex{INTR}     & \verb|^C| \\
\exi{ttychar/quit}              & \ex{QUIT}     & \verb|^\| \\
\exi{ttychar/suspend}           & \ex{SUSP}     & \verb|^Z| \\
\exi{ttychar/start}             & \ex{START}    & \verb|^Q| \\
\exi{ttychar/stop}              & \ex{STOP}     & \verb|^S| \\

\hline\hline
{SVR4 and 4.3+BSD} & & \\
\exi{ttychar/delayed-suspend}   & \ex{DSUSP}    & \verb|^Y| \\
\exi{ttychar/delete-word}       & \ex{WERASE}   & \verb|^W| \\
\exi{ttychar/discard}           & \ex{DISCARD}  & \verb|^O| \\
\exi{ttychar/eol2}              & \ex{EOL2}     & \\
\exi{ttychar/literal-next}      & \ex{LNEXT}    & \verb|^V| \\
\exi{ttychar/reprint}           & \ex{REPRINT}  & \verb|^R| \\

\hline\hline
{4.3+BSD} & & \\
\exi{ttychar/status}            & \ex{STATUS}   & \verb|^T| \\
\hline
\end{tabular}
\end{center}
\caption{Indices into the \protect\ex{tty-info} record's 
         \protect\var{control-chars} string,
         and the character traditionally found at each index.
         Only the indices for the {\Posix} entries are guaranteed to
         be non-\sharpf.}
\label{table:ttychars}
\end{table}

% Input flags
%%%%%%%%%%%%%
\begin{table}[p]
\begin{center}\small
\begin{tabular}{|lll|} \hline
Scsh & C & Meaning \\ 
\hline\hline
\Posix & & \\
\exi{ttyin/check-parity}
                        & \ex{INPCK}    & Check parity. \\
\exi{ttyin/ignore-bad-parity-chars}
                        & \ex{IGNPAR}   & Ignore chars with parity errors. \\
\exi{ttyin/mark-parity-errors}
                        & \ex{PARMRK}   & Insert chars to mark parity errors.\\
\exi{ttyin/ignore-break}
                        & \ex{IGNBRK}   & Ignore breaks. \\
\exi{ttyin/interrupt-on-break}
                        & \ex{BRKINT}   & Signal on breaks. \\
\exi{ttyin/7bits}
                        & \ex{ISTRIP}   & Strip char to seven bits. \\
\exi{ttyin/cr->nl}
                        & \ex{ICRNL}    & Map carriage-return to newline. \\
\exi{ttyin/ignore-cr}
                        & \ex{IGNCR}    & Ignore carriage-returns. \\
\exi{ttyin/nl->cr}
                        & \ex{INLCR}    & Map newline to carriage-return. \\
\exi{ttyin/input-flow-ctl}
                        & \ex{IXOFF}    & Enable input flow control. \\
\exi{ttyin/output-flow-ctl}
                        & \ex{IXON}     & Enable output flow control. \\

\hline\hline
{SVR4 and 4.3+BSD} & & \\
\exi{ttyin/xon-any}             & \ex{IXANY} & Any char restarts after stop. \\
\exi{ttyin/beep-on-overflow}    & \ex{IMAXBEL} & Ring bell when queue full. \\

\hline\hline
{SVR4} & & \\
\exi{ttyin/lowercase}           & \ex{IUCLC} & Map upper case to lower case. \\
\hline
\end{tabular}
\end{center}
\caption{Input-flags. These are the named flags for the \protect\ex{tty-info} 
         record's \protect\var{input-flags} field.
         These flags generally control the processing of input chars.
         Only the {\Posix} entries are guaranteed to be non-\sharpf.
         }
\label{table:ttyin}
\end{table}

% Output flags
%%%%%%%%%%%%%%
\begin{table}[p]
\begin{center}%\small
\begin{tabular}{|lll|} \hline
Scsh & C & Meaning \\ \hline\hline

\multicolumn{3}{|l|}{\Posix} \\
\exi{ttyout/enable}  & \ex{OPOST} & Enable output processing. \\

\hline\hline
\multicolumn{3}{|l|}{SVR4 and 4.3+BSD} \\
\exi{ttyout/nl->crnl}   & \ex{ONLCR} & Map nl to cr-nl. \\

\hline\hline
\multicolumn{3}{|l|}{4.3+BSD} \\
\exi{ttyout/discard-eot}    & \ex{ONOEOT}       & Discard EOT chars. \\
\exi{ttyout/expand-tabs}    & \ex{OXTABS}\footnote{
                        Note this is distinct from the SVR4-equivalent
                        \ex{ttyout/tab-delayx} flag defined in 
                        table~\ref{table:ttydelays}.}
                      & Expand tabs. \\

\hline\hline
\multicolumn{3}{|l|}{SVR4} \\
\exi{ttyout/cr->nl}             & \ex{OCRNL} & Map cr to nl. \\
\exi{ttyout/nl-does-cr}         & \ex{ONLRET}& Nl performs cr as well. \\
\exi{ttyout/no-col0-cr}         & \ex{ONOCR} & No cr output in column 0. \\
\exi{ttyout/delay-w/fill-char}  & \ex{OFILL} & Send fill char to delay. \\
\exi{ttyout/fill-w/del}         & \ex{OFDEL} & Fill char is {\Ascii} DEL. \\
\exi{ttyout/uppercase}          & \ex{OLCUC} & Map lower to upper case. \\
\hline
\end{tabular}
\end{center}
\caption{Output-flags. These are the named flags for the \protect\ex{tty-info}
         record's \protect\var{output-flags} field.
         These flags generally control the processing of output chars.
         Only the {\Posix} entries are guaranteed to be non-\sharpf.}
\label{table:ttyout}
\end{table}

% Delay flags
%%%%%%%%%%%%%
\begin{table}[p]
\begin{tabular}{r|ll|} \cline{2-3}
& Value & Comment \\ \cline{2-3}
{Backspace delay}       & \exi{ttyout/bs-delay}         & Bit-field mask \\
                        & \exi{ttyout/bs-delay0}        & \\
                        & \exi{ttyout/bs-delay1}        & \\

\cline{2-3}
{Carriage-return delay} & \exi{ttyout/cr-delay}         & Bit-field mask \\
                        & \exi{ttyout/cr-delay0}        & \\
                        & \exi{ttyout/cr-delay1}        & \\
                        & \exi{ttyout/cr-delay2}        & \\
                        & \exi{ttyout/cr-delay3}        & \\

\cline{2-3}
{Form-feed delay}       & \exi{ttyout/ff-delay}         & Bit-field mask \\
                        & \exi{ttyout/ff-delay0}        & \\
                        & \exi{ttyout/ff-delay1}        & \\

\cline{2-3}
{Horizontal-tab delay}  & \exi{ttyout/tab-delay}        & Bit-field mask \\
                        & \exi{ttyout/tab-delay0}       & \\
                        & \exi{ttyout/tab-delay1}       & \\
                        & \exi{ttyout/tab-delay2}       & \\
                        & \exi{ttyout/tab-delayx}       & Expand tabs \\

\cline{2-3}
{Newline delay}         & \exi{ttyout/nl-delay}         & Bit-field mask \\
                        & \exi{ttyout/nl-delay0}        & \\
                        & \exi{ttyout/nl-delay1}        & \\ 

\cline{2-3}
{Vertical tab delay}    & \exi{ttyout/vtab-delay}       & Bit-field mask \\
                        & \exi{ttyout/vtab-delay0}      & \\
                        & \exi{ttyout/vtab-delay1}      & \\

\cline{2-3}
{All}                   & \exi{ttyout/all-delay}  & Total bit-field mask \\
\cline{2-3}
\end{tabular}

\caption{Delay constants. These are the named flags for the
         \protect\ex{tty-info} record's \protect\var{output-flags} field.
         These flags control the output delays associated with printing
         special characters.
         They are non-{\Posix}, and have non-{\sharpf} values
         only on SVR4 systems.}
\label{table:ttydelays}
\end{table}

% Control flags
%%%%%%%%%%%%%%%
\begin{table}[p]
\begin{center}%\small
\begin{tabular}{|lll|} \hline
Scsh & C & Meaning \\

\hline\hline
\multicolumn{3}{|l|}{\Posix} \\
\exi{ttyc/char-size}    & \ex{CSIZE}    & Character size mask \\
\exi{ttyc/char-size5}   & \ex{CS5}      & 5 bits \\
\exi{ttyc/char-size6}   & \ex{CS6}      & 6 bits \\
\exi{ttyc/char-size7}   & \ex{CS7}      & 7 bits \\
\exi{ttyc/char-size8}   & \ex{CS8}      & 8 bits \\
\exi{ttyc/enable-parity}& \ex{PARENB}   & Generate and detect parity. \\
\exi{ttyc/odd-parity}   & \ex{PARODD}   & Odd parity. \\
\exi{ttyc/enable-read}  & \ex{CREAD}    & Enable reception of chars. \\
\exi{ttyc/hup-on-close} & \ex{HUPCL}    & Hang up on last close. \\
\exi{ttyc/no-modem-sync}& \ex{LOCAL}    & Ignore modem lines. \\
\exi{ttyc/2-stop-bits}  & \ex{CSTOPB}   & Send two stop bits. \\

\hline\hline
\multicolumn{3}{|l|}{4.3+BSD} \\
\exi{ttyc/ignore-flags}         & \ex{CIGNORE}  & Ignore control flags. \\
\exi{ttyc/CTS-output-flow-ctl}  & \verb|CCTS_OFLOW| & CTS flow control of output \\
\exi{ttyc/RTS-input-flow-ctl}   & \verb|CRTS_IFLOW| & RTS flow control of input \\
\exi{ttyc/carrier-flow-ctl}     & \ex{MDMBUF} & \\
\hline
\end{tabular}
\end{center}

\caption{Control-flags. These are the named flags for the \protect\ex{tty-info}
         record's \protect\var{control-flags} field.
         These flags generally control the details of the terminal's
         serial line.
         Only the {\Posix} entries are guaranteed to be non-\sharpf.}
\label{table:ttyctl}
\end{table}

% Local flags
%%%%%%%%%%%%%
\begin{table}[p]
\begin{center}\small
\begin{tabular}{|lll|} \hline
Scsh & C & Meaning \\

\hline\hline
\multicolumn{3}{|l|}{\Posix} \\
\exi{ttyl/canonical}    & \ex{ICANON}    & Canonical input processing. \\
\exi{ttyl/echo}         & \ex{ECHO}      & Enable echoing. \\
\exi{ttyl/echo-delete-line} & \ex{ECHOK}   & Echo newline after line kill. \\
\exi{ttyl/echo-nl}      & \ex{ECHONL}    & Echo newline even if echo is off. \\
\exi{ttyl/visual-delete}& \ex{ECHOE}     & Visually erase chars. \\
\exi{ttyl/enable-signals} & \ex{ISIG}    & Enable \verb|^|C, \verb|^|Z signalling. \\
\exi{ttyl/extended}     & \ex{IEXTEN}    & Enable extensions. \\
\exi{ttyl/no-flush-on-interrupt}
                        & \ex{NOFLSH}    &  Don't flush after interrupt. \\
\exi{ttyl/ttou-signal}  & \ex{ITOSTOP}   & \ex{SIGTTOU} on background output. \\

\hline\hline
\multicolumn{3}{|l|}{SVR4 and 4.3+BSD} \\
\exi{ttyl/echo-ctl}             & \ex{ECHOCTL}  
                                & Echo control chars as ``\verb|^X|''. \\
\exi{ttyl/flush-output}         & \ex{FLUSHO}   & Output is being flushed. \\
\exi{ttyl/hardcopy-delete}      & \ex{ECHOPRT}  & Visual erase for hardcopy. \\
\exi{ttyl/reprint-unread-chars} & \ex{PENDIN}   & Retype pending input. \\
\exi{ttyl/visual-delete-line}   & \ex{ECHOKE}   & Visually erase a line-kill. \\

\hline\hline
\multicolumn{3}{|l|}{4.3+BSD} \\
\exi{ttyl/alt-delete-word}      & \ex{ALTWERASE}  & Alternate word erase algorithm \\
\exi{ttyl/no-kernel-status}     & \ex{NOKERNINFO} & No kernel status on \verb|^T|. \\

\hline\hline
\multicolumn{3}{|l|}{SVR4} \\
\exi{ttyl/case-map}     & \ex{XCASE} & Canonical case presentation \\
\hline
\end{tabular}
\end{center}

\caption{Local-flags. These are the named flags for the \protect\ex{tty-info}
         record's \protect\var{local-flags} field.
         These flags generally control the details of the line-editting
         user interface.
         Only the {\Posix} entries are guaranteed to be non-\sharpf.}
\label{table:ttylocal}
\end{table}
%%%%%%%%%%%%%%%%%%%%%%%%%%%%%%%%%%%%%%%%%%%%%%%%%%%%%%%%%%%%%%%%%%%%%%%%%%%%%%%


%%% Local Variables: 
%%% mode: latex
%%% TeX-master: "man"
%%% End: 
