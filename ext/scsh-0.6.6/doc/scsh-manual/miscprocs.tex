%&latex -*- latex -*-

\chapter{Miscellaneous routines}

\section{Integer bitwise ops}
\label{sec:bitwise}
\defun{arithmetic-shift} {i j} \integer
\defunx {bitwise-and} {i j} \integer
\defunx {bitwise-ior} {i j} \integer
\defunx {bitwise-not} {i} \integer
\defunx {bitwise-xor} {i j} \integer
\begin{desc}
    These operations operate on integers representing semi-infinite 
    bit strings, using a 2's-complement encoding.

    \ex{arithmetic-shift} shifts \var{i} by \var{j} bits. 
    A left shift is $j > 0$; a right shift is $j < 0$.
\end{desc}

\section{Password encryption}

\defun {crypt} {key salt} {encrypted value}

Decrypts \var{key} by directly calling the \texttt{crypt} function
using \var{salt} to perturb the hashing algorithm. \var{Salt} must be
a two-character string consisting of digits, alphabetic characters,
``.'' or ``\verb+\+''. The length of \var{key} may be at most eight.

\section{Dot-Locking}
Section \ref{sec:filelocking} already points out that {\Posix}'s file
locks are almost useless in practice. To bypass this restriction other
advisory locking mechanisms, based only on standard file operations,
where invented. One of them is the so-called \emph{dot-locking} scheme
where the lock of \textit{file-name} is represented by the file
\textit{file-name}\texttt{.lock}. Care is taken that only one process
may generate the lock for a given file.

Here is scsh's interface to dot-locking:

\defun {obtain-dot-lock} {file-name [interval retry-number stale-time]} {\boolean}

\begin{desc}
  Tries to obtain the lock for \var{file-name}. If the file is already
  locked, the thread sleeps for \var{interval} seconds (default is 1)
  before it retries. If the lock cannot be obtained after
  \var{retry-number} attempts, the procedure returns \sharpf,
  otherwise \sharpt. The default value of \var{retry-number} is
  \sharpf{} which corresponds to an infinite number of retires.
  
  If \var{stale-time} is non-\sharpf, it specifies the minimum age a
  lock may have (in seconds) before it is considered \textit{stale}.
  \ex{Obtain-dot-lock} attempts to delete stale locks.  If it was
  succcessful obtaining a lock after breaking it, \ex{obtain-dot-lock}
  returns \ex{broken}.  If \var{stale-time} is \sharpf,
  \ex{obtain-dot-lock} never considers a lock stale.  The default for
    \var{stale-time} is 300.
    
    Note that it is possible that \ex{obtain-dot-lock} breaks a lock
    but nevertheless fails to obtain it otherwise.  If it is necessary
    to handle this case specially, use \ex{break-dot-lock} directly
    (see below) rather than specifying a non-\sharpf{} \var{stale-time}
\end{desc}

\defun {break-dot-lock} {file-name} {undefined}

\begin{desc}
  Breaks the lock for \var{file-name} if one exists.  Note that
  breaking a lock does \emph{not} imply a subsequent
  \ex{obtain-dot-lock} will succeed, as another party may have
  acquired the lock between \ex{break-dot-lock} and
  \ex{obtain-dot-lock}.
\end{desc}

\defun {release-dot-lock} {file-name} {\boolean}

\begin{desc}
  Releases the lock for \var{file-name}. On success,
  \ex{release-dot-lock} returns \sharpt, otherwise \sharpf. Note that
  this procedure can also be used to break the lock for
  \var{file-name}.
\end{desc}


\defun{with-dot-lock*} {file-name thunk} {value(s) of thunk}
\dfnx{with-dot-lock} {file-name body \ldots} {value(s) of body}{syntax}
  
\begin{desc}
  The procedure \ex{with-dot-lock*} obtains the requested lock, and
  then calls \ex{(\var{thunk})}. When \var{thunk} returns, the lock is
  released.  A non-local exit (\eg, throwing to a saved continuation
  or raising an exception) also causes the lock to be released.
  
  After a normal return from \var{thunk}, its return values are
  returned by \ex{with-dot-lock*}.  The \ex{with-dot-lock} special
  form is equivalent syntactic sugar.
\end{desc}

\section{Syslog facility}
\label{syslog-facility}

(Note: the functionality presented in this section is still somewhat
experimental and thus subject to interface changes.)

The procedures in this section provide access to the 4.2BSD syslog
facility present in most POSIX systems.  The functionality is in a
structure called \ex{syslog}.  There's an additional structure
\ex{syslog-channels} documented below.  The scsh interface to
the syslog facility differs significantly from that of the Unix
library functionality in order to support multiple simultaneous
connections to the syslog facility.

Log messages carry a variety of parameters beside the text of the
message itself, namely a set of options controlling the output format
and destination, the facility identifying the class of programs the
message is coming from, an identifier specifying the conrete program,
and the level identifying the importance of the message.  Moreover, a
log mask can prevent messages at certain levels to be actually sent to
the syslog daemon.

\subsection*{Log options}

A log option specifies details of the I/O behavior of the syslog
facility.  A syslog option is an element of a finite type (see
the \scm~manual) constructed by the
\ex{syslog-option} macro.  The syslog facility works with sets of
options which are represented as enum sets (see
the \scm~manual).

\dfn{syslog-option}{option-name}{option}{syntax}

\defun{syslog-option?}{x}{boolean}

\defun{make-syslog-options}{list}{options}

\dfn{syslog-options}{option-name \ldots}{options}{syntax}

\defun{syslog-options?}{x}{boolean}

\begin{desc}
\ex{Syslog-option} constructs a log option from the name of an
option.  (The possible names are listed below.)  \ex{Syslog-option?}
is a predicate for log options.  Options are comparable using
\ex{eq?}.  \ex{Make-syslog-options} constructs a set of options
from a list of options.  \ex{Syslog-options} is a macro which
expands into an expression returning a set of options from names.
\ex{Syslog-options?} is a predicate for sets of options.
\end{desc}
%                               
Here is a list of possible names of syslog options:

\begin{description}
\item[\ex{console}]
  If syslog cannot pass the message to syslogd it will attempt to
  write the message to the console.

\item[\ex{delay}]
  Delay opening the connection to syslogd immediately until the first
  message is logged.

\item[\ex{no-delay}]
  Open the connection to syslogd immediately.  Normally
  the open is delayed until the first message is logged.
  Useful for programs that need to manage the order in which
  file descriptors are allocated.

  \noindent\textbf{NOTA BENE:}
  The \ex{delay} and \ex{no-delay} options are included for
  completeness, but do not have the expected effect in the present
  Scheme interface: Because the Scheme interface has to multiplex
  multiple simultaneous connections to the syslog facility over a
  single one, open and close operations on that facility happen at
  unpredictable times.
  
\item[\ex{log-pid}]
  Log the process id with each message: useful for identifying
  instantiations of daemons.
\end{description}

\subsection*{Log facilities}

A log facility identifies the originator of a log message from a
finite set known to the system.  Each originator is identified by a
name:

\dfn{syslog-facility}{facility-name}{facility}{syntax}
\defun{syslog-facility?}{x}{boolean}

\begin{desc}
  \ex{Syslog-facility} is macro that expands into an expression
  returning a facility for a given name.  \ex{Syslog-facility?} is a
  predicate for facilities.  Facilities are comparable via \ex{eq?}.
\end{desc}
%
Here is a list of possible names of syslog facilities:

\begin{description}
\item[\ex{authorization}]
  The authorization system: login, su, getty, etc.

\item[\ex{cron}]
  The cron daemon.

\item[\ex{daemon}]
  System daemons, such as routed, that are not provided for explicitly
  by other facilities.

\item[\ex{kernel}]
  Messages generated by the kernel.

\item[\ex{lpr}]
  The line printer spooling system: lpr, lpc, lpd, etc.

\item[\ex{mail}]
  The mail system.

\item[\ex{news}]
  The network news system.

\item[\ex{user}]
  Messages generated by random user processes.

\item[\ex{uucp}]
  The uucp system.

\item[\ex{local0} \ex{local1} \ex{local2} \ex{local3} \ex{local4} \ex{local5} \ex{local6} \ex{local7}]
  Reserved for local use.
\end{description}
                                
\subsection*{Log levels}

A log level identifies the importance of a message from a fixed set
of possible levels.

\dfn{syslog-level}{level-name}{level}{syntax}
\defun{syslog-level?}{x}{boolean}
%
\begin{desc}
  \ex{Syslog-level} is macro that expands into an expression returning
  a facility for a given name.  \ex{Syslog-level?} is a predicate for
  facilities.  Levels are comparable via \ex{eq?}.
\end{desc}
%
Here is a list of possible names of syslog levels:

\begin{description}
\item[\ex{emergency}]
  A panic condition.  This is normally broadcast to all users.

\item[\ex{alert}]
  A condition that should be corrected immediately, such as a
  corrupted system database.

\item[\ex{critical}]
  Critical conditions, e.g., hard device errors.

\item[\ex{error}]
  Errors.

\item[\ex{warning}]
  Warning messages.

\item[\ex{notice}]
  Conditions that are not error conditions, but should possibly be
  handled specially.

\item[\ex{info}]
  Informational messages.

\item[\ex{debug}]
  Messages that contain information normally of use only when
  debugging a program.
\end{description}

\subsection*{Log masks}

A log masks can mask out log messages at a set of levels.  A log
mask is an enum set of log levels.

\defun{make-syslog-mask}{list}{mask}
\dfn{syslog-mask}{level-name \ldots}{mask}{syntax}
\defvar{syslog-mask-all}{mask}
\defun{syslog-mask-upto}{level}{mask}
\defun{syslog-mask?}{x}{boolean}

\begin{desc}
  \ex{Make-syslog-mask} constructs a mask from a list of levels.
  \ex{Syslog-mask} is a macro which constructs a mask from names of
  levels.  \ex{Syslog-mask-all} is a predefined log mask containing
  all levels.  \ex{Syslog-mask-upto} returns a mask consisting of all
  levels up to and including a certain level, starting with
  \ex{emergency}.
\end{desc}

\subsection*{Logging}

Scheme~48 dynamically maintains implicit connections to the syslog
facility specifying a current identifier, current options, a current
facility and a current log mask.  This implicit connection is held in
a thread fluid (see
Section~\ref{sec:ps_interac}).  Hence, every thread
maintains it own implicit connection to syslog.  Note that the
connection is not implicitly preserved across a \ex{spawn}, but it
is preserved across a \ex{fork-thread}:


\defun{with-syslog-destination}{string options facility mask thunk}{value}
\defun{set-syslog-destination!}{string options facility mask}{undefined}
%
\begin{desc}
  \ex{With-syslog-destination} dynamically binds parameters of the
  implicit connection to the syslog facility and runs \var{thunk}
  within those parameter bindings, returning what \var{thunk}
  returns.  Each of the parameters may be \ex{\#f} in which case the
  previous values will be used.  \ex{Set-syslog-destination!} sets the
  parameters of the implicit connection of the current thread.
\end{desc}

\defun{syslog}{level message}{undefined}
\defun{syslog}{level message [string options syslog-facility]}{undefined}
%
\begin{desc}
  \ex{Syslog} actually logs a message.  Each of the parameters of the
  implicit connection (except for the log mask) can be explicitly
  specified as well for the current call to \ex{syslog}, overriding
  the parameters of the channel.  The parameters revert to their
  original values after the call.
\end{desc}

\subsection*{Syslog channels}
%
The \ex{syslog-channels} structure allows direct manipulation of
syslog channels, the objects that represent connections to the syslog
facility.  Note that it is
not necessary to explicitly open a syslog channel to do logging.

\defun{open-syslog-channel}{string options facility mask}{channel}
\defun{close-syslog-channel}{channel}{undefined}
\defun{syslog}{level message channel}{undefined}
%
\begin{desc}
  \ex{Open-syslog-channel} and \ex{close-syslog-channel} create and
  destroy a connection to the syslog facility, respectively.  The
  specified form of calling \ex{syslog} logs to the specified channel.
\end{desc}


\section{MD5 interface}
\label{sec:md5}

Scsh provides a direct interface to the MD5 functions to compute the
``fingerprint'' or ``message digest'' of a file or string. It uses the
C library written by Colin Plum.

\defun{md5-digest-for-string}{string}{md5-digest}
\begin{desc}
  Calculates the MD5 digest for the given string.
\end{desc}
\defun{md5-digest-for-port}{port [buffer-size]}{md5-digest}
\begin{desc}
  Reads the contents of the port and calculates the MD5 digest for it.
  The optional argument \var{buffer-size} determines the size of the
  port's input buffer in bytes. It defaults to 1024 bytes.
\end{desc}

\defun{md5-digest?}{thing}{boolean}
\begin{desc}
  The type predicate for MD5 digests: \ex{md5-digest?} returns true if
  and only if \var{thing} is a MD5 digest.
\end{desc}
\defun{md5-digest->number}{md5-digest}{number}
\begin{desc}
  Returns the number corresponding to the MD5 digest.
\end{desc}
\defun{number->md5-digest}{number}{md5-digest}
\begin{desc}
  Creates a MD5 digest from a number.
\end{desc}

\defun{make-md5-context}{}{md5-context}
\defunx{md5-context?}{thing}{boolean}
\defunx{update-md5-context!}{md5-context string}\undefined
\defunx{md5-context->md5-digest}{md5-context}{md5-digest}
\begin{desc}
  These procedures provide a low-level interface to the library. A
  \var{md5-context} stores the state of a MD5 computation, it is
  created by \ex{make-md5-context}, its type predicate is
  \ex{md5-context?}. The procedure \ex{update-md5-context!} extends
  the \var{md5-context} by the given string. Finally,
  \ex{md5-context->md5-digest} returns the \var{md5-digest} for the
  \var{md5-context}. With these procedures it is possible to
  incrementally add strings to a \var{md5-context} before computing
  the digest.
\end{desc}

\section{Configuration variables}
\label{sec:configure}

This section describes procedures to access the configuration
parameters used to compile scsh and flags needed to build C extensions
for scsh.

\defun{host}{}{string}
\defunx{machine}{}{string}
\defunx{vendor}{}{string}
\defunx{os}{}{string}
\begin{desc}
  These procedures return the description of the host, scsh was built
  on, as determined by the script \texttt{config.guess}.
\end{desc}
%
\defun{prefix}{}{string}
\defunx{exec-prefix}{}{string}
\defunx{bin-dir}{}{string}
\defunx{lib-dir}{}{string}
\defunx{include-dir}{}{string}
\defunx{man-dir}{}{string}
\begin{desc}
  These procedures return the various directories of
  the scsh installation.
\end{desc}
%
\defun{lib-dirs-list}{}{symbol list}
\begin{desc}
  Returns the default list of library directories. See
  Section~\ref{sec:scsh-switches} for more information about the
  library search facility.
\end{desc}
%
\defun{libs}{}{string}
\defunx{defs}{}{string}
\defunx{cflags}{}{string}
\defunx{cppflags}{}{string}
\defunx{ldflags}{}{string}
\begin{desc}
  The values returned by these procedures correspond to the values
\texttt{make} used to compile scsh's C files.
\end{desc}
%
\defunx{compiler-flags}{}{string}
\begin{desc}
  The procedure \var{compiler-flags} returns flags suitable for
  running the C compiler when compiling a C file that uses scsh's
  foreign function interface.
\end{desc}

\defun{linker-flags}{}{string}
\begin{desc}
  Scsh also comes as a library that can be linked into other programs.
  The procedure \var{linker-flags} returns the appropriate flags to
  link the scsh library to another program.
\end{desc}


%%% Local Variables: 
%%% mode: latex
%%% TeX-master: "man"
%%% End: 
