% Scheme 48 documentation

\documentclass[twoside]{report}
\usepackage{hyperlatex}
\T\usepackage{longtable}
\T\usepackage{myindex}
\W\newcommand{\mainschindex}[1]{\cindex[#1]{{\tt #1}}}
\newcommand{\mainindex}[1]{\cindex[1]{{\rm #1}}}
\newcommand{\schindex}[1]{}
\newcommand{\sharpindex}[1]{}

\newcommand{\codemainindex}[1]{{\mainschindex{#1}\code{#1}}}
\newcommand{\codeindex}[1]{{\mainschindex{#1}\code{#1}}}


\W%% -*-latex-*-
%% Navigation menus for the Scheme 48 report.

\newcommand{\previousnextmenu}{\EmptyP{\HlxBackUrl\HlxForwUrl}{%
    \EmptyP{\HlxBackUrl}
      {Previous: \xlink{\HlxBackTitle}{\HlxBackUrl}}
      {}%
    \EmptyP{\HlxForwUrl}
      {\EmptyP{\HlxBackUrl}{{} | {}}{}%
       Next: \xlink{\HlxForwTitle}{\HlxForwUrl}}
      {}%
    }{}}

\newcommand{\toppanel}{%
    {\em Scheme 48 Manual} | {\link{Contents}{top_node}}%
    \EmptyP{\HlxUpUrl}
      {{} | In Chapter: \xlink{\HlxUpTitle}{\HlxUpUrl}}
      {}%
    \\%
    \previousnextmenu}

\newcommand{\chaptertoppanel}{%
    {\em Scheme 48 Manual} | {\link{Contents}{top_node}}%
    \EmptyP{\HlxPrevUrl}
      {{} | Previous Chapter: \xlink{\HlxPrevTitle}{\HlxPrevUrl}}
      {}%
    \EmptyP{\HlxNextUrl}
      {{} | Next Chapter: \xlink{\HlxNextTitle}{\HlxNextUrl}}
      {}%
    \\%
    \previousnextmenu{}}

\newcommand{\bottommatter}{%
  %\htmlrule
  \html{P}
  \EmptyP{\HlxAddress}
  {\par{}\html{ADDRESS}\HlxAddress\html{/ADDRESS}\\}{}}

\newcommand{\bottompanel}{\previousnextmenu}
\newcommand{\xsubsection}[1]{%
\texonly{\subsection{#1}}%
\htmlonly{\strong{#1}\\}%
}

\newcommand{\evalsto}{%
\texonly{$\rightarrow$}%
\htmlonly{\code{->}}%
}

\newcommand{\cvar}[1]{%
\texonly{{\rm\em{#1}}}%
\htmlonly{\code{\var{#1}}}%
}

%%%%%%%%%%%%%%%% Latex prototypes
\texonly{
\newenvironment{protos}{\list{$\bullet$}
{\leftmargin1.2em\rightmargin0pt\itemsep0pt\parsep0pt\partopsep-2pt}}
{\endlist}

% The following is for prototypes that have return types.
%    (foo int int) -> int

\newcommand{\proto}[3]{%
\protonoindex{{#1}}{{#2}}{{#3}}%
\mainschindex{{#1}}}%

\newcommand{\protonoindex}[3]{\item\noindent\unskip%
\hbox{\spaceskip=0.5em\code{({#1}{\it#2\/})} {$\rightarrow$} {\it#3}}}

\newcommand{\cproto}[1]{\item\noindent\unskip%
\hbox{\spaceskip=0.5em\code{{#1}}}}

\newcommand{\cgcproto}[1]{\item\noindent\unskip%
\hbox{\spaceskip=0.5em\code{{#1}}}\hfill\penalty 0%
\hbox{ }\nobreak\hfill\hbox{\rm (may GC)}}

\newcommand{\protonoresultnoindex}[2]{\item\noindent\unskip%
\hbox{\spaceskip=0.5em\code{(\hbox{#1}{\it#2\/})}}}

\newcommand{\protonoresult}[2]{%
\protonoresultnoindex{{#1}}{{#2}}%
\mainschindex{{#1}}}%

\newcommand{\protoresult}[1]{\newline\unskip%
{\hspace*{2em}\code{{$\rightarrow$} {\it#1}}\hfill}}

% Syntax prototypes

\newcommand{\syntaxprotonoresultnoindex}[2]{\item\noindent\unskip%
\hbox{\spaceskip=0.5em\code{(\hbox{#1}{#2})}}\hfill\penalty 0%
\hbox{ }\nobreak\hfill\hbox{\rm syntax}}

\newcommand{\syntaxprotonoresult}[2]{%
\syntaxprotonoresultnoindex{{#1}}{{#2}}\mainschindex{{#1}}}

\newcommand{\syntaxproto}[3]{\item\noindent\unskip%
\hbox{\spaceskip=0.5em\code{(\hbox{#1}{#2})}} {$\rightarrow$} {\it#3}%
\hfill\penalty 0%
\hbox{ }\nobreak\hfill\hbox{\rm syntax}}

% This can be reduced

\newcommand{\pconstproto}[2]{\item\noindent\unskip%
\hbox{\spaceskip=0.5em\code{#1}}\hfill\penalty 0%
\hbox{ }\nobreak\hfill\hbox{\rm #2}}

% Variable prototype
\newcommand{\constproto}[2]{\pconstproto{{#1}}{{#2}}\mainschindex{{#1}}}

\newcommand{\constprotonoindex}[2]{\pconstproto{{#1}}{{#2}}}

}
%%%%%%%%%%%%%%%% end of Latex proto definitions

%%%%%%%%%%%%%%%% HTML prototypes
\htmlonly{
\newenvironment{protos}{\begin{itemize}}{\end{itemize}}

% The following is for prototypes that have return types.
%    (foo int int) -> int

\newcommand{\protonoindex}[3]{%
%\cindex{\code{#1}}%
\item\noindent\code{({#1}{\var{#2}\/})~-->~{\var{#3}}}}

\newcommand{\proto}[3]{%
\protonoindex{{#1}}{{#2}}{{#3}}%
\mainschindex{{#1}}}%

\newcommand{\protonoresultnoindex}[2]{%
%\cindex{\code{#1}}%
\item\noindent\code{({#1}{\var{#2}\/})}}

\newcommand{\protonoresult}[2]{%
\protonoresultnoindex{{#1}}{{#2}}%
\mainschindex{{#1}}}%

\newcommand{\pconstproto}[2]{%
\item\noindent\prototagstart\code{{#1}}\prototag{{#2}}}

% Variable prototype
\newcommand{\constproto}[2]{\pconstproto{{#1}}{{#2}}\mainschindex{{#1}}}

\newcommand{\constprotonoindex}[2]{\pconstproto{{#1}}{{#2}}}

\newcommand{\cproto}[1]{%
\item\noindent\code{{#1}}}

\newcommand{\cgcproto}[1]{%
\item\noindent\prototagstart\code{{#1}}\prototag{(may GC)}}

\newcommand{\syntaxprotonoresult}[2]{%
\syntaxprotonoresultnoindex{{#1}}{{#2}}\mainschindex{{#1}}}

\newcommand{\syntaxprotonoresultnoindex}[2]{%
\item\noindent\prototagstart\code{({#1}{#2})}\prototag{syntax}}

\newcommand{\protoresult}[1]{%
\\{}\noindent\code{\ \ \ \ -->~{\it{#1}}}}

\newcommand{\syntaxproto}[3]{%
\item\noindent\prototagstart\code{({#1}{#2})~-->~{\var{#3}}}\prototag{syntax}}

\newcommand{\prototagstart}{%
\begin{rawhtml}<table border=0 cellspacing=0 cellpadding=0 width=80%>
<tr> <td>\end{rawhtml}}

\newcommand{\prototag}[1]{%
\begin{rawhtml}</td> <td align=right>\end{rawhtml}%
{#1}%
\begin{rawhtml}</td></tr></table>\end{rawhtml}}

}
%%%%%%%%%%%%%%%% end of HTML proto definitions

\W\newcommand{\langle}{<}
\W\newcommand{\rangle}{>}

\newcommand{\meta}[1]{{\noindent\mbox{\textrm{$\langle$#1$\rangle$}}}}
\newcommand{\hyper}[1]{\meta{#1}}
\newcommand{\hyperi}[1]{\hyper{#1$_1$}}
\newcommand{\hyperii}[1]{\hyper{#1$_2$}}
\newcommand{\hyperj}[1]{\hyper{#1$_i$}}
\newcommand{\hypern}[1]{\hyper{#1$_n$}}
\renewcommand{\var}[1]{\noindent\mbox{\textit{#1}}}
\newcommand{\vari}[1]{\var{#1$_1$}}
\newcommand{\varii}[1]{\var{#1$_2$}}
\newcommand{\variii}[1]{\var{#1$_3$}}
\newcommand{\variv}[1]{\var{#1$_4$}}
\newcommand{\varj}[1]{\var{#1$_j$}}
\newcommand{\varn}[1]{\var{#1$_n$}}

\W\newcommand{\frenchspacing}{}
\W\newcommand{\textnormal}[1]{#1}

\newcommand{\defining}[1]{{\emph{#1}}}

\newcommand{\exprtype}{syntax}

\W\newcommand{\unskip}{}
\W\newcommand{\textrm}[1]{#1}

\newcommand{\dotsfoo}{\ldots\texonly{\thinspace}}

\newcommand{\goesto}{%
\texonly{\hbox{$\longrightarrow$}}%
\htmlonly{\code{-->}}%
}

\newcommand{\xform}{%
\texonly{\hbox{$\Longrightarrow$}}%
\htmlonly{\code{--->}}%
}

\newcommand{\arbno}[1]{{{#1}$^*$}}
\newcommand{\hack}{Scheme~48}

\T\newcommand{\alt}{$\vert$}
\W\newcommand{\alt}{|}

\T\newcommand{\ok}{\discretionary{}{}{}}
\W\newcommand{\ok}{}

\T\newcommand{\syn}[1]{{\noindent\mbox{\textrm{$\langle$#1$\rangle$}}}}
\W\newcommand{\syn}[1]{{\noindent\mbox{$\langle$#1$\rangle$}}}



% Make a few big HTML files, and not a lot of small ones.
\setcounter{htmldepth}{3}

% Put the html code in its own directory.
\htmldirectory{manual}

% Set the html base name.
\htmlname{s48manual}

% Add sections to main menu
\setcounter{htmlautomenu}{2}

% White background
\htmlattributes{BODY}{BGCOLOR="#ffffff"}

\htmltitle{Scheme 48 Manual}

% Suppress navigation panel for first page.
\htmlpanel{0}

%%% End preamble

\begin{document}
\label{top_node}

\section{Optimistic concurrency}

A \cvar{proposal} is a record of reads from and and writes to locations in
 memory.
The \cvar{logging} operations listed below record any values read or
 written in the current proposal.
A reading operation, such as \code{provisional-vector-ref}, first checks to
 see if the current proposal contains a value for the relevent location.
If so, that value is returned as the result of the read.
If not, the current contents of the location are stored in the proposal and
 then returned as the result of the read.
A logging write to a location stores the new value as the current contents of
 the location in the current proposal; the contents of the location itself
 remain unchanged.

\cvar{Committing} to a proposal verifies that any reads logged in
 the proposal are still valid and, if so, performs any writes that
 the proposal contains.
A logged read is valid if, at the time of the commit, the location contains
 the same value it had at the time of the original read (note that this does
 not mean that no change occured, simply that the value now is the same as
 the value then).
If a proposal has an invalid read then the effort to commit fails no change
 is made to the value of any location.

\begin{protos}
\proto{ensure-atomicity}{ thunk}{value(s)}
\protonoresult{ensure-atomicity!}{ thunk}
\end{protos}
\noindent
If there is a proposal in place 
 \code{ensure-atomicity} and \code{ensure-atomicity!}
 simply make a (tail-recursive) call to \cvar{thunk}.
If the current proposal is \code{\#f} they create a new proposal,
 install it, call \cvar{thunk}, and then try to commit to the proposal.
This process repeats, with a new proposal on each iteration, until
 the commit succeeds.
\code{Ensure-atomicity} returns whatever values are returned by \cvar{thunk}
 on its final invocation, while \code{ensure-atomicity!} discards any such
 values and returns nothing.

\begin{protos}
\proto{provisional-car}{ pair}{value}
\proto{provisional-cdr}{ pair}{value}
\protonoresult{provisional-set-car!}{ pair value}
\protonoresult{provisional-set-cdr!}{ pair value}
\proto{provisional-cell-ref}{ cell}{value}
\protonoresult{provisional-cell-set!}{ cell value}
\proto{provisional-vector-ref}{ vector i}{value}
\protonoresult{provisional-vector-set!}{ vector i value}
\proto{provisional-string-ref}{ vector i}{char}
\protonoresult{provisional-string-set!}{ vector i char}
\proto{provisional-byte-vector-ref}{ vector i}{k}
\protonoresult{provisional-byte-vector-set!}{ vector i k}
\end{protos}
\noindent
These are all logging versions of their Scheme counterparts.
Reads are checked when the current proposal is committed and writes are
 delayed until the commit succeeds.
If the current proposal is \code{\#f} these perform exactly as their Scheme
 counterparts.

The following implementation of a simple counter may not function
 if used by multiple threads.
\begin{example}
(define (make-counter)
  (let ((value 0))
    (lambda ()
      (set! value (+ value 1))
      value)))
\end{example}

Here is the same procedure using a proposal to ensure that each
 increment operation happens atomically.
The value of the counter is kept in a vector to allow the use of
 logging operations.
\begin{example}
(define (make-counter)
  (let ((value (make-cell 0)))
    (lambda ()
      (ensure-atomicity
        (lambda ()
          (let ((v (+ (provisional-cell-ref value)
                      1)))
            (provisional-cell-set! value v)
            v))))))
\end{example}

Because \code{ensure-atomicity} creates a new proposal only if there is
 no existing proposal in place, multiple atomic actions can be performed
 simultaneously.
The following procedure increments an arbitrary number of counters at the same
 time.
This works even if the same counter appears multiple times;
 \code{(step-counters! c0 c0)} would add two to the value of counter \code{c0}.
\begin{example}
(define (step-counters! . counters)
  (ensure-atomicity
    (lambda ()
      (for-each (lambda (counter)
                  (counter))
                counters))))
\end{example}

\begin{example}
(define-synchronized-record-type \cvar{tag} \cvar{type-name}
  (\cvar{constructor-name} \cvar{field-tag} \ldots)
  [(\cvar \cvar{field-tag} \ldots)]
  \cvar{predicate-name}
  (\cvar{field-tag} \cvar{accessor-name} [\cvar{modifier-name}])
  \ldots)
\end{example}
This is the same as \code{define-record-type} except all field reads and
 writes are logged in the current proposal.
If the optional list of field tags is present then only those fields will
 be logged.

\begin{protos}
\proto{atomically}{ thunk}{value(s)}
\protonoresult{atomically!}{ thunk}
\end{protos}
\noindent
\code{Atomically} and \code{atomically!} are identical
 to \code{ensure-atomicity} and \code{ensure-atomicity!} except that they
 always install a new proposal before calling \code{thunk}.
The current proposal is saved and then restored after \code{thunk} returns.
\code{Atomically} and \code{atomically!} are useful if \code{thunk} contains
 code that should not be combined with any other operation (example?).

The following procedures give access to the low-leve proposal mechanism for
 use when necessary.
\begin{protos}
\proto{make-proposal}{}{proposal}
\proto{current-proposal}{}{proposal}
\protonoresult{set-current-proposal!}{ proposal}
\proto{with-proposal}{ proposal thunk}{value(s)}
\proto{maybe-commit}{ proposal}{boolean}
\end{protos}
\noindent
\code{With-proposal} saves the current proposal, installs \cvar{proposal} as
 the current proposal, and then calls \cvar{thunk}.
When \cvar{thunk} returns the saved proposal is reinstalled as the current
 proposal
 and the value(s) returned by \cvar{thunk} are returned.

\code{Maybe-commit} verifies that any reads logged in \cvar{proposal} are
 still valid and, if so, performs any writes that \cvar{proposal} contains.
A logged read is valid if, at the time of the commit, the location contains
 the same value it had at the time of the original read (note that this does
 not mean that no change occured, simply that the value now is the same as
 the value then).
\code{Maybe-commit} returns \code{\#t} if the commit succeeds and \code{\#f}
 if it fails.

\end{document}    
