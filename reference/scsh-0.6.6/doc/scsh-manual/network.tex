%&latex -*- latex -*-

\chapter{Networking}

The Scheme Shell provides a BSD-style sockets interface. 
There is not an official standard for a network interface
for scsh to adopt (this is the subject of the forthcoming Posix.8
standard).
However, Berkeley sockets are a \emph{de facto} standard,
being found on most Unix workstations and PC operating systems.

It is fairly straightforward to add higher-level network protocols
such as smtp, telnet, or http on top of the the basic socket-level
support scsh provides.
The Scheme Underground has also released a network library with
many of these protocols as a companion to the current release of scsh.
See this code for examples showing the use of the sockets interface.

\section{High-level interface}

For convenience, and to avoid some of the messy details of the socket
interface, we provide a high level socket interface. These routines
attempt to make it easy to write simple clients and servers without
having to think of many of the details of initiating socket connections.
We welcome suggested improvements to this interface, including better
names, which right now are solely descriptions of the procedure's action.
This might be fine for people who already understand sockets,
but does not help the new networking programmer.

\defun {socket-connect} {protocol-family socket-type . args} {socket}
\begin{desc}
\ex{socket-connect} is intended for creating client applications.
\var{protocol-family} is specified as either the
\ex{protocol-family/internet} or \ex{protocol-family/unix}.
\var{socket-type} is specified as either \ex{socket-type/stream} or
\ex{socket-type/datagram}. See \ex{socket} for a more complete
description of these terms. 

The variable \var{args} list is meant to specify protocol family
specific information. For Internet sockets, this consists of two
arguments: a host name and a port number. For {\Unix} sockets, this
consists of a pathname.

\ex{socket-connect} returns a \ex{socket} which can be used for input
and output from a remote server. See \ex{socket} for a description of
the \emph{socket record}.
\end{desc}

\defun {bind-listen-accept-loop} {protocol-family proc arg} {does-not-return}
\begin{desc}
\ex{bind-listen-accept-loop} is intended for creating server
applications. \var{protocol-family} is specified as either the
\ex{protocol-family/internet} or \ex{protocol-family/unix}.
\var{proc} is a procedure of two arguments: a \ex{socket} and a
{socket-address}. \var{arg} specifies a port number for Internet sockets
or a pathname for {\Unix} sockets. See \ex{socket} for a more complete
description of these terms.

\var{proc} is called with a socket and a socket address each time there
is a connection from a client application. The socket allows
communications with the client.  The socket address specifies the
address of the remote client.

This procedure does not return, but loops indefinitely accepting
connections from client programs.
\end{desc}

\defun {bind-prepare-listen-accept-loop} {protocol-family prepare proc arg} {does-not-return}
\begin{desc}
  Same as \ex{bind-listen-accept-loop} but runs the thunk
  \var{prepare} after binding the address and before entering the
  loop. The typical task of the \var{prepare} procedure is to change
  the user id from the superuser to some unprivileged id once the
  address has been bound.
\end{desc}

\section{Sockets}

\defun  {create-socket} {protocol-family type [protocol]} {socket}
\defunx {create-socket-pair} {type} {[socket$_{1}$ socket$_{2}$]}
\defunx {close-socket}   {socket} \undefined
\begin{desc}

A socket is one end of a network connection. Three specific properties
of sockets are specified at creation time: the protocol-family, type,
and protocol.

The \var{protocol-family} specifies the protocol family to be used with
the socket. This also determines the address family of socket addresses,
which are described in more detail below. Scsh currently supports the
{\Unix} internal protocols and the Internet protocols using the
following constants:
\begin{code}\codeallowbreaks
protocol-family/unspecified
protocol-family/unix
protocol-family/internet\end{code}

The \var{type} specifies the style of communication. Examples that your
operating system probably provides are stream and datagram sockets.
Others maybe available depending on your system. Typical values are:
\begin{code}\codeallowbreaks
socket-type/stream
socket-type/datagram
socket-type/raw\end{code}

The \var{protocol} specifies a particular protocol to use within a
protocol family and type. Usually only one choice exists, but it's
probably safest to set this explicitly. See the protocol database
routines for information on looking up protocol constants.

New sockets are typically created with \ex{create-socket}. However,
\ex{create-socket-pair} can also be used to create a pair of connected
sockets in the \ex{protocol-family/unix} protocol-family. The value of a
returned socket is a \emph{socket record}, defined to have the following
structure:
\begin{code}
(define-record socket
  family                                ; protocol family
  inport                                ; input-port
  outport)                              ; output-port\end{code}

The \ex{family} specifies the protocol family of the socket. The
\ex{inport} and \ex{outport} fields are ports that can be used for input
and output, respectively. For a stream socket, they are only usable
after a connection has been established via \ex{connect-socket} or
\ex{accept-connection}. For a datagram socket, \var{outport} can be
immediately using \ex{send-message}, and \var{inport} can be used after
\ex{bind} has created a local address.

\ex{close-socket} provides a convenient way to close a socket's port. It
is preferred to explicitly closing the inport and outport because using
\ex{close} on sockets is not currently portable across operating systems.

\end{desc}

\defun  {port->socket} {port protocol-family} {socket}
\begin{desc}
  This procedure turns \var{port} into a socket object. The port's
  underlying file descriptor must be a socket with protocol family
  \var{protocol-family}. \ex{port->socket} applies \ex{dup->inport}
  and \ex{dup->outport} to \var{port} to create the ports of the
  socket object.
  
  \ex{port->socket} comes in handy for writing
  servers which run as children of \texttt{inetd}: after receiving a
  connection \texttt{inetd} creates a socket and passes it as
  standard input to its child.
\end{desc}

\section{Socket addresses}

The format of a socket-address depends on the address family of the
socket. Address-family-specific routines are provided to convert
protocol-specific addresses to socket addresses. The value returned by
these routines is a \emph{socket-address record}, defined to have the
following visible structure:
\begin{code}
(define-record socket-address
  family)                               ; address family\end{code}

The \ex{family} is one of the following constants:
\begin{code}\codeallowbreaks
address-family/unspecified
address-family/unix
address-family/internet\end{code}

\defun  {unix-address->socket-address} {pathname} {socket-address}
\begin{desc}
\ex{unix-address->socket-address} returns a \var{socket-address} based
on the string \var{pathname}. There is a system dependent limit on the
length of \var{pathname}.
\end{desc}

\defun  {internet-address->socket-address} {host-address service-port} {socket-address}
\begin{desc}
\ex{internet-address->socket-address} returns a \var{socket-address} based
on an integer \var{host-address} and an integer \var{service-port}.
Besides being a 32-bit host address, an Internet host address can also
be one of the following constants:
\begin{code}\codeallowbreaks
internet-address/any
internet-address/loopback
internet-address/broadcast\end{code}

The use of \ex{internet-address/any} is described below in
\ex{bind-socket}. \ex{internet-address/loopback} is an address that
always specifies the local machine. \ex{internet-address/broadcast} is
used for network broadcast communications.

For information on obtaining a host's address, see the \ex{host-info}
function. 
\end{desc}

\defun  {socket-address->unix-address} {socket-address} {pathname}
\defunx {socket-address->internet-address} {socket-address} {[host-address service-port]}
\begin{desc}

The routines \ex{socket-address->internet-address} and
\ex{socket-address->unix-address} return the address-family-specific addresses.
Be aware that most implementations don't correctly return anything more
than an empty string for addresses in the {\Unix} address-family.
\end{desc}

\section{Socket primitives}

The procedures in this section are presented in the order in which a
typical program will use them. Consult a text on network systems
programming for more information on sockets.\footnote{
Some recommended ones are: 

\begin{itemize}

\item ``Unix Network Programming'' by W. Richard Stevens

\item ``An Introductory 4.3BSD Interprocess Communication Tutorial.''
(reprinted in UNIX Programmer's Supplementary Documents Volume 1, PS1:7)

\item ``An Advanced 4.3BSD Interprocess Communication Tutorial.''
(reprinted in UNIX Programmer's Supplementary Documents Volume 1, PS1:8)

\end{itemize}
}
The last two tutorials are freely available as part of BSD. In the
absence of these, your {\Unix} manual pages for socket might be a good
starting point for information.

\defun {connect-socket} {socket socket-address} \undefined
\begin{desc}
\ex{connect-socket} sets up a connection from a \var{socket}
to a remote \var{socket-address}. A connection has different meanings
depending on the socket type. A stream socket must be connected before
use. A datagram socket can be connected multiple times, but need not be
connected at all if the remote address is specified with each
\ex{send-message}, described below. Also, datagram sockets
may be disassociated from a remote address by connecting to a null
remote address.
\end{desc}
\defun {connect-socket-no-wait} {socket socket-address} \boolean
\defunx {connect-socket-successful?} {socket} \boolean
\begin{desc}
  Just like \ex{connect-socket}, \ex{connect-socket-no-wait} sets up a
  connection from a \var{socket} to a remote \var{socket-address}.
  Unlike \ex{connect-socket}, \ex{connect-socket-no-wait} does not
  block if it cannot establish the connection immediately. Instead it
  will return \sharpf{} at once. In this case a subsequent \ex{select} on
  the output port of the socket will report the output port as ready
  as soon as the operation system has established the connection or as
  soon as setting up the connection led to an error. Afterwards, the
  procedure \ex{connect-socket-successful?} can be used to test
  whether the connection has been established successfully or not.
\end{desc}
\defun {bind-socket} {socket socket-address} \undefined
\begin{desc}
\ex{bind-socket} assigns a certain local \var{socket-address} to a
\var{socket}. Binding a socket reserves the local address. To receive
connections after binding the socket, use \ex{listen-socket} for stream
sockets and \ex{receive-message} for datagram sockets.

Binding an Internet socket with a host address of
\ex{internet-address/any} indicates that the caller does
not care to specify from which local network interface connections are
received. Binding an Internet socket with a service port number of zero
indicates that the caller has no preference as to the port number
assigned.

Binding a socket in the {\Unix} address family creates a socket special
file in the file system that must be deleted before the address can be
reused. See \ex{delete-file}.
\end{desc}

\defun {listen-socket} {socket backlog} \undefined
\begin{desc}
\ex{listen-socket} allows a stream \var{socket} to start receiving connections,
allowing a queue of up to \var{backlog} connection requests. Queued
connections may be accepted by \ex{accept-connection}.
\end{desc}

\defun {accept-connection} {socket} {[new-socket socket-address]}
\begin{desc}
\ex{accept-connection} receives a connection on a \var{socket}, returning
a new socket that can be used for this connection and the remote socket
address associated with the connection.
\end{desc}

\defun  {socket-local-address} {socket} {socket-address}
\defunx {socket-remote-address} {socket} {socket-address}
\begin{desc}
Sockets can be associated with a local address or a remote address or
both. \ex{socket-local-address} returns the local \var{socket-address}
record associated with \var{socket}. \ex{socket-remote-address} returns
the remote \var{socket-address} record associated with \var{socket}.
\end{desc}

\defun {shutdown-socket} {socket how-to} \undefined
\begin{desc}

\ex{shutdown-socket} shuts down part of a full-duplex socket.
The method of shutting done is specified by the \var{how-to} argument,
one of:
\begin{code}\codeallowbreaks
shutdown/receives
shutdown/sends
shutdown/sends+receives\end{code}
\end{desc}

\section{Performing input and output on sockets}

\defun  {receive-message} {socket length [flags]} {[string-or-\sharpf{} socket-address]}
\dfnix  {receive-message!} {socket string [start] [end] [flags]}
        {[count-or-\sharpf{}  socket-address]}{procedure}
        {receive-message"!@\texttt{receive-message"!}}
\defunx {receive-message/partial} {socket length [flags]}
         {[string-or-\sharpf{} socket-address]}
\dfnix {receive-message!/partial} {socket string [start] [end] [flags]}
        {[count-or-\sharpf{} socket-address]}{procedure}
        {receive-message"!/partial@\texttt{receive-message"!/partial}}
\defun  {send-message} {socket string [start] [end] [flags] [socket-address]}
        \undefined
\defunx {send-message/partial} 
        {socket string [start] [end] [flags] [socket-address]} {count}

\begin{desc}
For most uses, standard input and output routines such as
\ex{read-string} and \ex{write-string} should suffice.  However, in some
cases an extended interface is required. The \ex{receive-message} and
\ex{send-message} calls parallel the \ex{read-string} and
\ex{write-string} calls with a similar naming scheme.

One additional feature of these routines is that \ex{receive-message}
returns the remote \var{socket-address} and \var{send-message} takes an
optional remote
\ex{socket-address}. This allows a program to know the source of input
from a datagram socket and to use a datagram socket for output without
first connecting it.

All of these procedures take an optional \var{flags} field. This
argument is an integer bit-mask, composed by or'ing together the
following constants:
\begin{code}\codeallowbreaks
message/out-of-band
message/peek
message/dont-route\end{code}

See \ex{read-string} and \ex{write-string} for a more detailed
description of the arguments and return values.
\end{desc}

\section{Socket options}

\defun  {socket-option} {socket level option} {value}
\defunx {set-socket-option} {socket level option value} \undefined

\begin{desc}
\ex{socket-option} and \ex{set-socket-option} allow the inspection and
modification, respectively, of several options available on sockets. The
\var{level} argument specifies what protocol level is to be examined or
affected. A level of \ex{level/socket} specifies the highest possible
level that is available on all socket types. A specific protocol number
can also be used as provided by \ex{protocol-info}, described below.

There are several different classes of socket options. The first class
consists of boolean options which can be either true or false. Examples
of this option type are:
\begin{code}\codeallowbreaks
socket/debug
socket/accept-connect
socket/reuse-address
socket/keep-alive
socket/dont-route
socket/broadcast
socket/use-loop-back
socket/oob-inline
socket/use-privileged
socket/cant-signal
tcp/no-delay\end{code}

Value options are another category of socket options. Options of this
type are an integer value. Examples of this option type are:
\begin{code}\codeallowbreaks
socket/send-buffer
socket/receive-buffer
socket/send-low-water
socket/receive-low-water
socket/error
socket/type
ip/time-to-live
tcp/max-segment\end{code}

A third option type specifies how long for data to linger after a socket
has been closed. There is only one option of this type:
\ex{socket/linger}. It is set with either \sharpf to disable it or an
integer number of seconds to linger and returns a value of the same type
upon inspection.

The fourth and final option type of this time is a timeout option. There
are two examples of this option type: \ex{socket/send-timeout} and
\ex{socket/receive-timeout}. These are set with a real number of
microseconds resolution and returns a value of the same type upon
inspection.

\end{desc}

\section{Database-information entries}

\defun  {host-info} {name-or-socket-address} {host-info}
\defunx {network-info} {name-or-socket-address} {network-info or \sharpf}
\defunx {service-info} {name-or-number [protocol-name]} {service-info or \sharpf}
\defunx {protocol-info} {name-or-number} {protocol-info or \sharpf}

\begin{desc}

\ex{host-info} allows a program to look up a host entry based on either
its string \var{name} or \var{socket-address}. The value returned by this
routine is a \emph{host-info record}, defined to have the following
structure:
\begin{code}
(define-record host-info
  name                                  ; Host name
  aliases                               ; Alternative names
  addresses)                            ; Host addresses\end{code}

\ex{host-info} could fail and raise an error for one of the following
reasons: 
\begin{code}\codeallowbreaks
herror/host-not-found
herror/try-again
herror/no-recovery
herror/no-data
herror/no-address\end{code}

\ex{network-info} allows a program to look up a network entry based on either
its string \var{name} or \var{socket-address}. The value returned by this
routine is a \emph{network-info record}, defined to have the following
structure:
\begin{code}
(define-record network-info
  name                                  ; Network name
  aliases                               ; Alternative names
  net)                                  ; Network number\end{code}

\ex{service-info} allows a program to look up a service entry based
on either its string \var{name} or integer \var{port}. The value returned
by this routine is a \emph{service-info record}, defined to have the
following structure:
\begin{code}
(define-record service-info
  name                                  ; Service name
  aliases                               ; Alternative names
  port                                  ; Port number
  protocol)                             ; Protocol name\end{code}

\ex{protocol-info} allows a program to look up a protocol entry based
on either its string \var{name} or integer \var{number}. The value returned
by this routine is a \emph{protocol-info record}, defined to have the
following structure:
\begin{code}
(define-record protocol-info
  name                                  ; Protocol name
  aliases                               ; Alternative names
  number)                               ; Protocol number)\end{code}

\ex{network-info}, \ex{service-info} and \ex{protocol-info} return
\sharpf if the specified entity was not found.

\end{desc}
