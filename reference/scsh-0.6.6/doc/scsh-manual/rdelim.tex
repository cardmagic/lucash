%&latex -*- latex -*-

\chapter{Reading delimited strings}
\label{chapt:rdelim}

Scsh provides a set of procedures that read delimited strings from
input ports.
There are procedures to read a single line of text
(terminated by a newline character),
a single paragraph (terminated by a blank line),
and general delimited strings
(terminated by a character belonging to an arbitrary character set).

These procedures can be applied to any Scheme input port.
However, the scsh virtual machine has native-code support for performing
delimited reads on Unix ports, and these input operations should be 
particularly fast---much faster than doing the equivalent character-at-a-time
operation from Scheme code.

All of the delimited input operations described below take a \ex{handle-delim}
parameter, which determines what the procedure does with the terminating 
delimiter character.
There are four possible choices for a \ex{handle-delim} parameter:
\begin{inset}
\begin{tabular}{|l|l|} \hline
        \ex{handle-delim}       & Meaning \\ \hline\hline
        \ex{'trim}              & Ignore delimiter character. \\
        \ex{'peek}              & Leave delimiter character in input stream. \\
        \ex{'concat}            & Append delimiter character to returned value. \\
        \ex{'split}             & Return delimiter as second value. \\
        \hline
\end{tabular}
\end{inset}
The first case, \ex{'trim}, is the standard default for all the routines 
described in this section.
The last three cases allow the programmer to distinguish between strings
that are terminated by a delimiter character, and strings that are 
terminated by an end-of-file.


\begin{defundesc} {read-line} {[port handle-newline]} {{\str} or eof-object}
  Reads and returns one line of text; on eof, returns the eof object. 
  A line is terminated by newline or eof. 

  \var{handle-newline} determines what \ex{read-line} does with the 
  newline or EOF that terminates the line; it takes the general set
  of values described for the general \ex{handle-delim} case above,
  and defaults to \ex{'trim} (discard the newline).
  Using this argument allows one to tell whether or not the last line of
  input in a file is newline terminated.
\end{defundesc}

\defun{read-paragraph} {[port handle-delim]} {{\str} or eof}
\begin{desc}
    This procedure skips blank lines,
    then reads text from a port until a blank line or eof is found.
    A ``blank line'' is a (possibly empty) line composed only of white space.
    The \var{handle-delim} parameter determines how the terminating
    blank line is handled.
    It is described above, and defaults to \ex{'trim}.
    The \ex{'peek} option is not available.
\end{desc}


The following procedures read in strings from ports delimited by characters
belonging to a specific set.
See section~\ref{sec:char-sets} for information on character set manipulation.

\defun{read-delimited}{char-set [port handle-delim]} {{\str} or eof}
\begin{desc}
    Read until we encounter one of the chars in \var{char-set} or eof.
    The \var{handle-delim} parameter determines how the terminating character
    is handled. It is described above, and defaults to \ex{'trim}.

    The \var{char-set} argument may be a charset, a string, a character, or a
    character predicate; it is coerced to a charset.
\end{desc}

\dfni{read-delimited!} {char-set buf [port handle-delim start end]}
        {nchars or eof or \#f}{procedure}
        {read-delimited"!@\texttt{read-delimited"!}}
\begin{desc}
    A side-effecting variant of \ex{read-delimited}.
    
    The data is written into the string \var{buf} at the indices in the
    half-open interval $[\var{start},\var{end})$; the default interval is the
    whole string: $\var{start}=0$ and $\var{end}=\ex{(string-length
    \var{buf})}$.  The values of \var{start} and \var{end} must specify a
    well-defined interval in \var{str}, \ie, $0 \le \var{start} \le \var{end}
    \le \ex{(string-length \var{buf})}$.  

    It returns \var{nbytes}, the number of bytes read. If the buffer filled up
    without a delimiter character being found, \ex{\#f} is returned. If
    the port is at eof when the read starts, the eof object is returned.

    If an integer is returned (\ie, the read is successfully terminated by
    reading a delimiter character), then the \var{handle-delim} parameter
    determines how the terminating character is handled. 
    It is described above, and defaults to \ex{'trim}.
\end{desc}



\dfni{\%read-delimited!} {char-set buf gobble? [port start end]} 
        {[char-or-eof-or-\#f \integer]}{procedure}
        {\%read-delimited"!@\texttt{\%read-delimited"!}}
\begin{desc}
This low-level delimited reader uses an alternate interface.
It returns two values: \var{terminator} and \var{num-read}.
\begin{description}
\item [terminator]
        A value describing why the read was terminated:
        \begin{flushleft}
        \begin{tabular}{l@{\qquad$\Rightarrow$\qquad}l}
        Character or eof-object & Read terminated by this value. \\
        \ex{\#f}                & Filled buffer without finding a delimiter.
        \end{tabular}
        \end{flushleft}

\item [num-read]
        Number of characters read into \var{buf}.
\end{description}

If the read is successfully terminated by reading a delimiter character,
then the \var{gobble?} parameter determines what to do with the terminating
character.
If true, the character is removed from the input stream;
if false, the character is left in the input stream where a subsequent
read operation will retrieve it.
In either case, the character is also the first value returned by
the procedure call.
\end{desc}

%Note:
%- Invariant: TERMINATOR = #f  =>  NUM-READ = END - START.
%- Invariant: TERMINATOR = eof-object and NUM-READ = 0 => at EOF.
%- When determining the TERMINATOR return value, ties are broken
%  favoring character or the eof-object over #f. That is, if the buffer
%  fills up, %READ-DELIMITED! will peek at one more character from the
%  input stream to determine if it terminates the input. If so, that
%  is returned, not #f.

\begin{defundesc} {skip-char-set} {skip-chars [port]} {\integer}
    Skip characters occurring in the set \var{skip-chars};
    return the number of characters skipped.
    The \var{skip-chars} argument may be a charset, a string, a character, or a
    character predicate; it is coerced to a charset.
\end{defundesc}

%%% Local Variables: 
%%% mode: latex
%%% TeX-master: "man"
%%% End: 
